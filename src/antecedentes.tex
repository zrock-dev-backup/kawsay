\section{Antecedentes}

% The purpose of this section is to explore the importance and complexities of scheduling, particularly in educational institutions, and the technological solutions that have been developed to address scheduling challenges.

% Scheduling plays a critical role in the efficient operation of universities, as the Registrar Office is responsible for creating the schedule of classes and assigning students to these classes. Understanding the impact of scheduling decisions on the university's functioning is essential for optimizing resource allocation and ensuring student satisfaction. 

% The scheduling process can be broken down into two primary aspects:
% - The creation of the class schedule
% - The assignment of individual students to classes

% The scheduling process is not as simple as it may appear, especially when virtual classrooms are introduced. Virtual classrooms eliminate the need for physical facilities and time-based resource management, such as coordinating staff availability and room reservations. However, they come with their own set of constraints, primarily related to server capacity, scalability, and managing online interaction.

% Constraints in Virtual Classrooms
% While physical limitations (like room size) are irrelevant in virtual classrooms, other challenges must still be addressed. These include ensuring server performance, managing bandwidth, and maintaining effective communication tools for both students and instructors. 
%
% Moreover, making the most efficient use of available time is a critical goal of scheduling. In educational settings, ensuring that classes are scheduled at optimal times and that the system runs smoothly is key to maximizing student engagement and performance.

% Learning Management System (LMS)
% At Jala University (JalaU), the delivery of curricula is facilitated through an online Learning Management System (LMS). The LMS provides an organized platform where students and faculty can access course materials, interact, and participate in assessments. The system is integral to the university’s virtual scheduling, providing a centralized tool for scheduling and resource management.

% \subsection{Scheduling as a Key Function}
%
% Scheduling is broadly defined as a function in many industries, including education, where it ensures that events take place at the most opportune times. It is an essential aspect of personal time management and is crucial for enhancing productivity.
%
% \subsubsection{The Origins of Scheduling}
The science of scheduling has its roots in the industrial revolution. In 1874, Frederick Taylor, often referred to as the father of scientific management, began developing a discipline he called “Scientific Management” to improve the efficiency of machine shops. Taylor’s practices, including the creation of detailed schedules displayed on bulletin boards, helped ensure that time was used more effectively in the workplace.

Henry Gantt, a colleague of Taylor, built upon his ideas in the 1910s by creating Gantt charts, which provided a visual representation of project timelines. These charts are still widely used today, from large-scale projects such as Facebook, Google or Amazon.

% \subsection{General Scheduling Issues}

Various industries face scheduling issues due to the nature of their operations and business needs:

- **Healthcare**: Overlapping schedules, last-minute changes, and understaffing are common, particularly with the need for 24/7 coverage.
- **Retail**: Issues like understaffing, overstaffing, and poor shift management affect operations.
- **Hospitality**: Resource allocation and scheduling conflicts arise, especially during peak seasons.
- **Manufacturing and Warehousing**: Challenges include understaffing and equipment availability conflicts.
- **Call Centers**: Shifting schedules and fluctuating volumes complicate staffing.
- **Education**: Scheduling issues include staffing, withdraws, dissmission, and student availability.

% In all these sectors, businesses must efficiently manage people and resources to minimize disruptions, often utilizing sophisticated scheduling tools to address these challenges.

% \subsection{Technological Solutions to Scheduling Problems}

The development of computerized scheduling tools began in the mid-1950s, with E.I. du Pont de Nemours (Du Pont) experimenting with early computers like the UNIVAC1 to assist in planning and scheduling tasks. In the 1970s, more powerful scheduling systems were developed for mini-computers, further transforming the way scheduling was handled.

In our modern day ther is Teach 'n Go, modern school management software that integrates scheduling functionality with other features for managing school operations. Teach 'n Go is an LMS \footcite{LMS (Learning Management System)} designed to school management tasks, including scheduling.

% \subsubsection{Features of Teach 'n Go}
Teach 'n Go provides a wide range of features for school management:
\begin{itemize}
    \item \textbf{Class Color Coding}: Helps visually organize classes.
    \item \textbf{Classroom Size Management}: Restricts the number of students per class.
    \item \textbf{Quick Student Enrollment}: Allows for bulk enrollment or unenrollment of students.
    \item \textbf{Class Archiving}: Archives completed classes for record-keeping.
    \item \textbf{Quick Stats Dashboard}: Displays important metrics like attendance and grades.
    \item \textbf{Google Calendar Integration}: Syncs class schedules with Google Calendar for easy access.
\end{itemize}

The software also allows customization with school branding and provides integration with tools like Google Meet and Microsoft Teams, facilitating a seamless scheduling experience.

% \subsubsection{Registrar-Student Relationship}
The system facilitates better interaction between the registrar and students:
\begin{itemize}
    \item \textbf{Dashboard}: Provides an overview of the daily schedule, including teacher assignments and classroom availability.
    \item \textbf{Calendar View}: Displays courses per teacher, classroom, or other criteria, allowing users to create new meetings or print the calendar.
    \item \textbf{Error Handling}: Alerts users when scheduling conflicts arise, such as when a classroom is double-booked.
\end{itemize}

% \subsection{Dynamic Scheduling and Customization}
Dynamic scheduling involves the creation and management of flexible, real-time schedules that can adapt to changing conditions. In the context of educational institutions, dynamic scheduling is essential for handling unexpected changes, such as last-minute class cancellations or rescheduling.

Teach 'n Go allows users to set up recurring class schedules, either weekly or on custom dates, with customizable fields for tracking specific data such as class subject, level, and course codes.

Teach 'n Go integrates with HostHub Open API to talk with another systems. Making it possible to use it in an ecosystem of related software.


% \subsection{Spreadsheet as a Scheduling Solution}
% In addition to dedicated scheduling software like Teach 'n Go, spreadsheets are often used for dynamic scheduling. These tools offer a flexible way to organize and manipulate scheduling data, though they may lack some of the specialized features and integrations offered by purpose-built systems.
%
% \subsection{Conclusion}
% Scheduling is an essential but complex function in many industries, including education. As virtual classrooms and technology-driven solutions become more prevalent, it is important to understand the constraints and challenges that come with scheduling in this new environment. Tools like Teach 'n Go help address these challenges by providing customizable, dynamic scheduling solutions that integrate with existing systems, improving overall efficiency and resource management.

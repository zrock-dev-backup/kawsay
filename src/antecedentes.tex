\section{Antecedentes}

% > university administrators who are responsible for the creation of the schedule of classes and the assignment of students to these classes to understand the effects of their decisions on the operation of the university.
% Why?
%
% aspects of scheduling process:
%     - creation of the schedule of classes
%     - assigment of individual students ot classes
%
% importance of scheduling process.
%
% > The process starts with building a schedule of classes that helps expand the choice of courses available to the student.
% does a student has a say when picking schedule?
%
% virtual classes removes facility from the scheduling coordination of staff, facility and time resources.
%
% scheduling has constraints. Virtual classrom's physical limitation is irrelevant (related to server capacity and scalability) what are the constraints of a virtual classroom?
%
% > Making full use of available time is also critical
% The goal of scheduling.
%
% > Our online curricula are delivered via a learning management system (LMS)
% JalaU uses an LMS


% you have sell the fact that scheduling is no easy task

% Assignment to a schedule, or systematic arrangement.
% definition of the word
sheduling is defined as:  A function in many aspects of industry, commerce and computing in which events are timed to take place at the most opportune time. 

scheduling as personal time management for productive people.

% origins of scheduling
The science of scheduling began in the machine shops of the industrial revolution. In 1874, Frederick Taylor, the son of a wealthy lawyer, turned down his acceptance at Harvard to become an apprentice machinist at Enterprise Hydraulic Works in Philadelphia. Four years later, he completed his apprenticeship and began working at the Midvale Steel Works, where he rose through the ranks from lathe operator to machine shop foreman and ultimately to chief engineer. In the process, he came to believe that the time of the machines (and people) he oversaw was not being used very well, leading him to develop a discipline he called “Scientific Management.”
Taylor created a planning office, at the heart of which was a bulletin board displaying the shop’s schedule for all to see. The board depicted every machine in the shop, showing the task currently being carried out by that machine and all the tasks waiting for it. This practice would be built upon by Taylor’s colleague Henry Gantt, who in the 1910s developed the Gantt charts that would help organize many of the twentieth century’s most ambitious construction projects, from the Hoover Dam to the Interstate Highway System. A century later, Gantt charts still adorn the walls and screens of project managers at firms like Amazon, IKEA, and SpaceX.

% general scheduling issues
Various industries encounter scheduling issues due to the nature of their operations and workforce needs. In healthcare, overlapping schedules, last-minute changes, understaffing, and burnout are common due to the need for 24/7 coverage and unpredictable staff availability. Retail businesses face issues like understaffing or overstaffing, seasonal demand fluctuations, and employee dissatisfaction from poor shift management. Hospitality businesses struggle with similar challenges, along with resource allocation issues, especially during peak seasons. Manufacturing and warehousing often deal with understaffing, equipment availability conflicts, and unplanned disruptions, while call centers face overlapping shifts and fluctuating call volumes. In transportation and logistics, time zone conflicts and resource allocation issues complicate scheduling, leading to shipment delays. Event planning and entertainment sectors are affected by overlapping schedules and last-minute changes, while educational institutions face problems with staffing, cancellations, and student availability. Freelancers and consultants juggle multiple client schedules, leading to inconsistent prioritization, while tech and software development teams deal with shifting deadlines, time zone conflicts, and changes in project priorities. Construction and field services often struggle with resource allocation and seasonal demand fluctuations, and the fitness industry faces last-minute cancellations and overlapping sessions. In all these sectors, businesses must effectively manage people, resources, and client needs to minimize disruptions, often relying on scheduling tools and proactive planning to address these challenges.

% tech solutions
The development of CPM as computerised project management tool can be traced back to mid 1956. E.I. du Pont de Numours (Du Pont) was looking for useful things to do with its ‘UNIVAC1’ computer (this was one of the very first computers installed in a commercial business anywhere and only the third UNIVAC machine built).
Du Pont’s management felt that ‘planning, estimating and scheduling’ seemed like a good use of the computer! Morgan Walker was given the job of discovering if a computer could be programmed to help. Others had started studying the problem, including other researchers within
Du Pont but no one had achieved a commercially viable outcome

During the 1970s, the arrival of powerful project scheduling systems running on ‘Mini Computers’ caused the first major change. The lower operating cost of systems such as MAPPS on Wang and Artemis on HP and DEC hardware caused the rapid demise of mainframe scheduling systems. Apart from a few legacy systems the era of the mainframe was over by the
mid 1980s. The ‘mini systems’ retained many of the characteristics of the mainframes though and
required skilled schedulers to make efficient use of them. From the perspective of the people
working as schedulers all that changed was the hardware and maybe the software vendor.

Teach 'n Go - Modern school management software. https://www.teachngo.com/
Description: 
> Effectively manage and grow with our all-in-one school management software
Does it means is holistic

What does it do apart from scheduling?
These software products are typically designed for a broad audience or specific market and are intended to generate revenue through direct sales, subscriptions, or licensing agreements

It's an LMS that offers scheduling.

Functions
    application: token based authentication
        core requirements
            - Class colour coding
            - copy classes
            - clasroom size: limit the number of people in a class
            - Quickly Enrol Students: enrol or unenroll multiple students into a class
            - Archive Classes: archive school classes that have been completed to keep track of records
            - Quick stats: dashboard of important data. like attendance or grades?
            - Future Enrol Students: enrol students into future classes

        distinguishing:
            - school branding: Add your own unique touch to our product with our customizable branding options, custom url
            - Google Calendar Integration: So that students and other interested parties can have acess to the calendar. Do observe its populated with courses schedule

        registrar-student relationship
            - Dashboard: what teachers are scheduled in what classroom for the day ahead
            - Calendar: A view (per teacher, clasrooms, default) courses. Allowing interested parties to create new meetings.
                Print Calendar
                Add events
                calendar updates

    error handling:
        - Identifies and produces a warning notifications when there is a conflict. Notification indicates: Monday between 12:00 - 13:00; Computer Lab is being used by Programming CSS3; Donna Swanson is teaching Programming CSS3

    time-related:
        - creation of class


> customisable fields for tracking specific data in your school.
Data
    input:
        - create your class (clas details): class name, class subject, class level, awarding body, course code, teacher, classroom.
        - class schedule: start date, end date. Lessons days and times.

    output
        - class details: teacher and classroom fields are eligible. Is the product that makes those fields available prior input.

    preset
        - class schedule: is expected to have days of the week avaialble to choose.

    persistent:
        class details

    sequences/combinations
        - class schedule: class recurrence can be weekly or custom dates.

Operations
    environment: mobile app

Interfaces
    system interfaces
        multiple integrations: excel, google meet, google calendar, teams, skype, Hosthub Open API

%% Spreadsheet
Spreadsheet as a solution to the problem

What is dynamic scheduling.

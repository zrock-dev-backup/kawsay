\section{Planteamiento del problema}

The Registrar's Office needs to process an ever-growing amount of decentralized data, where business rules are constantly evolving. It must also generate reports and create schedules that need to be updated and adjusted in real-time as the academic calendar progresses and real-life situations arise.

A schedule must be updated for the following scenarios:

\begin{itemize}
    \item Student grades are not available yet. Since the grades are not in, the Registrar evaluates a student's eligibility based on the likelihood of passing. This analysis is performed considering the student's GPA and SAP status.
    \item The student failed a course.
    \item The student’s eligibility was evaluated as "No," but they were able to pass the course. The student was not considered for their next course and now needs to be assigned to it.
    \item A professor or faculty practitioner requests a schedule change. This request could be due to feedback from the class or external factors.
\end{itemize}

Currently, the Registrar's Office relies on spreadsheets, which presents several challenges.

Working with spreadsheets in an academic context is problematic due to the decentralized nature of the data, requiring manual input from multiple sources. This process is time-consuming and prone to errors, especially when real-time data updates are needed.

The spreadsheet setup itself requires considerable effort, involving defining cells, creating formulas and macros, and establishing conditional formatting to make the document resemble a functional dashboard. While this setup is necessary for organizing and interpreting large datasets, it can quickly become overwhelming without careful planning.

Spreadsheets also struggle with tasks like assessment and conflict detection. Manual calculations for student eligibility are time-consuming and error-prone, and spreadsheets lack automatic checks for issues such as double-booked students or course conflicts. Without built-in alerts or automated error detection, spreadsheets can lead to mistakes and inefficiencies in academic management.

\begin{figure}[H]
    \centering
    \includegraphics[width=0.8\textwidth]{"resources/images/ishikawa-spanish.png"}
    \caption{Diagrama de causa y efecto - Fuente: Elaboración propia}
    \label{fig:image}
\end{figure}

\subsection{Formulación del problema}
How an academic calendar software ETL \footnote{Extract, transform, load: data is extracted from one source, processed and delivered to another} system with machine learning-powered schedule forecasting can impact effiency and precision of the Registrar's Office?

Este capítulo establece el fundamento conceptual y técnico sobre el cual se desarrolla el presente proyecto de grado. Se abordan temas clave como Machine Learning, algoritmos de generación de horarios, arquitecturas de software backend y frontend, metodologías de diseño y gestión de proyectos, indicadores de rendimiento, técnicas de diagramado, tecnologías específicas y herramientas de soporte, culminando con el plan de pruebas basado en estándares de calidad. El objetivo es proporcionar el contexto necesario para comprender las decisiones de diseño e implementación del sistema web para la gestión del calendario académico en Jala University, incluyendo la predicción de elegibilidad de estudiantes mediante Machine Learning.

\subsection{Machine Learning}
El Machine Learning (ML), o Aprendizaje Automático, es una rama de la inteligencia artificial que se enfoca en el desarrollo de sistemas capaces de aprender y mejorar a partir de la experiencia, sin ser explícitamente programados para cada tarea específica \parencite{Samuel1959}. En el contexto de este proyecto, el ML se aplica para analizar datos históricos académicos y predecir la elegibilidad de los estudiantes para inscribirse en futuras asignaturas, buscando optimizar la asignación de cupos y la planificación académica, abordando así uno de los objetivos principales de entender el impacto del ML en este proceso.

\subsubsection{Supervised learning}
El aprendizaje supervisado es un paradigma del ML donde el algoritmo aprende a partir de un conjunto de datos previamente etiquetado, es decir, cada ejemplo de entrada está asociado a una salida correcta conocida \parencite{Bishop2006}. El objetivo es entrenar un modelo que pueda predecir la etiqueta de salida para nuevas entradas no vistas. Para la predicción de elegibilidad de estudiantes en Jala University, se utilizarán datos históricos (calificaciones, cursos aprobados, plan de estudios) como entradas etiquetadas (elegible/no elegible para un curso específico) para entrenar un modelo predictivo, como podría ser una regresión logística o una máquina de soporte vectorial.

\subsubsection{Unsupervised learning}
A diferencia del aprendizaje supervisado, el aprendizaje no supervisado trabaja con datos no etiquetados, buscando descubrir patrones, estructuras o relaciones inherentes en la información sin una guía previa sobre las salidas correctas \parencite{Hastie2009}. Técnicas comunes incluyen el clustering (agrupación) y la reducción de dimensionalidad. Aunque el enfoque principal del proyecto es supervisado para la predicción de elegibilidad, el análisis exploratorio de datos podría emplear técnicas no supervisadas para identificar grupos de estudiantes con perfiles académicos similares o detectar anomalías en los patrones de inscripción, complementando la comprensión del proceso actual.

\subsection{Timetable generation algorithms}
La generación de horarios académicos es un problema complejo de optimización combinatoria, clasificado como NP-difícil, que busca asignar recursos (profesores, aulas, horarios) a eventos (clases) respetando un conjunto de restricciones duras (inviolables) y blandas (deseables) \parencite{Schaerf1999}. El desarrollo de algoritmos eficientes para esta tarea es crucial para instituciones educativas como Jala University, ya que impacta directamente en la satisfacción de estudiantes y docentes, y en la utilización de recursos, siendo un pilar para mejorar el proceso actual de gestión del calendario.

\subsubsection{Evolutionary and genetic algorithms}
Los algoritmos evolutivos, y en particular los algoritmos genéticos, son metaheurísticas inspiradas en la evolución biológica que resultan efectivas para problemas de optimización complejos como la generación de horarios \parencite{Eiben2003}. Operan sobre una población de soluciones candidatas (horarios), aplicando operadores genéticos como selección, cruce y mutación para evolucionar hacia soluciones de mayor calidad (que satisfacen más restricciones o minimizan conflictos). Su capacidad para explorar amplios espacios de búsqueda los hace adecuados para encontrar horarios viables y optimizados en entornos con múltiples restricciones como el de Jala University.

\subsubsection{Constraint-based reasoning}
El razonamiento basado en restricciones (Constraint Satisfaction Problems - CSP) modela el problema de generación de horarios definiendo un conjunto de variables (clases), dominios (posibles asignaciones de tiempo/aula/profesor) y restricciones (reglas como no solapamiento, disponibilidad, capacidad) \parencite{Rossi2006}. Algoritmos como backtracking, forward checking o arc consistency se utilizan para encontrar asignaciones que satisfagan todas las restricciones. Este enfoque es útil para garantizar la viabilidad de los horarios generados, asegurando que se cumplan las reglas fundamentales del calendario académico de la universidad.

\subsubsection{Linear programming/Integer programming}
La programación lineal (LP) y la programación entera (IP) son técnicas de investigación de operaciones que permiten modelar problemas de optimización mediante funciones objetivo lineales y restricciones lineales (con variables continuas en LP y enteras o binarias en IP) \parencite{Winston2004}. La generación de horarios puede formularse como un problema de IP, donde las variables representan decisiones de asignación (e.g., si una clase se asigna a un horario específico) y el objetivo es optimizar alguna métrica (minimizar huecos, maximizar preferencias) sujeto a las restricciones académicas. Aunque computacionalmente intensivos, pueden garantizar soluciones óptimas para instancias de tamaño moderado.

\subsection{Backend architecture}
La arquitectura backend define la estructura interna del servidor, la lógica de negocio y la gestión de datos del sistema web, siendo fundamental para su escalabilidad, mantenibilidad y rendimiento. Una arquitectura bien diseñada facilita la evolución del sistema y la integración de nuevas funcionalidades, como el módulo de predicción de horarios con ML, asegurando que la gestión del calendario académico sea robusta y eficiente \parencite{Richards2015}.

\subsubsection{Microservices}
La arquitectura de microservicios estructura una aplicación como una colección de servicios pequeños, autónomos y débilmente acoplados, cada uno enfocado en una capacidad de negocio específica y comunicándose a través de APIs (\textit{Application Programming Interfaces}) ligeras, usualmente sobre \texttt{HTTP} \parencite{Newman2015}. Adoptar microservicios para el sistema de gestión académica permitiría desarrollar, desplegar y escalar independientemente componentes como la gestión de cursos, la generación de horarios, la predicción de elegibilidad y la gestión de usuarios, aumentando la resiliencia y flexibilidad del sistema global en Jala University.

\subsubsection{Clean Architecture}
La Arquitectura Limpia, propuesta por Robert C. Martin, es un conjunto de principios de diseño de software que promueve la separación de intereses y la independencia de frameworks, UI y bases de datos, organizando el código en capas concéntricas (Entidades, Casos de Uso, Adaptadores de Interfaz, Frameworks y Drivers) \parencite{Martin2017}. Aplicar Clean Architecture en el backend asegura que la lógica de negocio central (reglas académicas, algoritmos de predicción y generación de horarios) esté aislada de detalles externos, facilitando las pruebas, la mantenibilidad y la adaptabilidad a cambios tecnológicos futuros.

\subsubsection{Domain Driven Design}
El Diseño Guiado por el Dominio (DDD) es un enfoque para el desarrollo de software complejo que se centra en modelar el dominio del negocio (en este caso, la gestión del calendario académico y la predicción de elegibilidad en Jala University) y plasmar ese modelo en el código, utilizando un lenguaje ubicuo compartido entre expertos del dominio y desarrolladores \parencite{Evans2003}. DDD ayuda a gestionar la complejidad mediante conceptos como Entidades, Objetos Valor, Agregados, Repositorios y Servicios de Dominio, asegurando que el software refleje fielmente las reglas y procesos académicos.

\subsection{Frontend architecture}
La arquitectura frontend se ocupa de la estructura y organización del código que se ejecuta en el navegador del usuario, gestionando la interfaz de usuario (UI) y la interacción con el backend. Una buena arquitectura frontend es esencial para proporcionar una experiencia de usuario fluida, receptiva y mantenible para estudiantes y administradores que utilicen el sistema de gestión del calendario académico \parencite{Osmani2017}.

\subsubsection{MVM}
El patrón Model-View-ViewModel (MVVM) es un patrón de diseño arquitectónico para interfaces de usuario que facilita la separación entre la lógica de presentación (ViewModel), la interfaz de usuario (View) y los datos (Model) \parencite{Smith2005}. El ViewModel actúa como intermediario, exponiendo datos y comandos que la View puede enlazar (data binding), lo que simplifica el manejo del estado de la UI y mejora la testeabilidad. Utilizar MVVM en el frontend (posiblemente con frameworks como \texttt{React} o \texttt{Vue} adaptándolo) puede ayudar a gestionar la complejidad de las interfaces para visualizar horarios, configurar parámetros de predicción y mostrar resultados.

\subsection{Design Methodology}
La metodología de diseño guía el proceso de creación del sistema, desde la concepción de la idea hasta la implementación, enfocándose en comprender las necesidades del usuario y entregar valor de manera eficiente. La elección de una metodología adecuada es clave para asegurar que el sistema web desarrollado satisfaga los requerimientos de Jala University y sus usuarios \parencite{Cooper2014}.

\subsubsection{Lean thinking}
El pensamiento Lean, originado en el Sistema de Producción Toyota y adaptado al desarrollo de software, se centra en la eliminación de desperdicios (actividades que no agregan valor), la entrega rápida de valor al cliente y el aprendizaje continuo a través de ciclos de construir-medir-aprender \parencite{WomackJones2003}. Aplicar principios Lean en este proyecto implica enfocarse en las funcionalidades esenciales para la gestión del calendario y la predicción, validar hipótesis rápidamente con usuarios de Jala University y optimizar el flujo de desarrollo para entregar un producto mínimo viable (MVP) de forma temprana.

\subsubsection{Lean sprint design}
El Design Sprint, popularizado por Google Ventures, es un proceso intensivo de cinco días (o una versión adaptada) que comprime meses de trabajo de diseño y validación en una sola semana, utilizando principios Lean y Design Thinking para definir problemas, idear soluciones, prototipar y probar con usuarios reales \parencite{Knapp2016}. Emplear un enfoque similar a un Design Sprint al inicio o en fases clave del proyecto puede acelerar la toma de decisiones sobre el diseño de la interfaz, el flujo de interacción para la gestión de horarios y la presentación de las predicciones de ML, asegurando que la solución esté alineada con las necesidades de Jala University.

\subsection{Metodologías de trabajo}
Las metodologías de trabajo ágiles proporcionan marcos para organizar y gestionar el proceso de desarrollo de software de manera flexible, colaborativa e iterativa, permitiendo adaptarse a cambios y entregar valor de forma incremental. La elección de una metodología ágil es fundamental para la gestión eficiente de un proyecto de grado como este \parencite{Beck2001}.

\subsubsection{Scrum}
Scrum es un marco de trabajo ágil ampliamente utilizado que organiza el desarrollo en ciclos cortos llamados Sprints (usualmente de 2 a 4 semanas), con roles definidos (Product Owner, Scrum Master, Equipo de Desarrollo), artefactos (Product Backlog, Sprint Backlog, Incremento) y eventos (Sprint Planning, Daily Scrum, Sprint Review, Sprint Retrospective) \parencite{SchwaberSutherland2020}. Adoptar Scrum para este proyecto permitiría gestionar el desarrollo del sistema web de forma iterativa, priorizando funcionalidades, fomentando la colaboración y permitiendo la inspección y adaptación continua basada en el feedback y los avances.

\subsubsection{Scrumban}
Scrumban es un enfoque híbrido que combina elementos de Scrum (como los roles, las reuniones y el enfoque iterativo) con la visualización del flujo de trabajo y la gestión de límites de trabajo en progreso (WIP) de Kanban \parencite{Kniberg2010}. Podría ser útil para este proyecto si se busca la estructura de Scrum pero con una mayor flexibilidad en la gestión del flujo de tareas, permitiendo visualizar cuellos de botella en el desarrollo del sistema de gestión académica y optimizar la entrega continua de funcionalidades.

\subsection{KPI}
Los Indicadores Clave de Rendimiento (KPIs - Key Performance Indicators) son métricas cuantificables utilizadas para evaluar el éxito de una organización, proyecto o actividad específica en relación con sus objetivos estratégicos \parencite{Parmenter2015}. Para este proyecto, se definirán KPIs relevantes como la precisión del modelo de predicción de elegibilidad, el tiempo de generación de horarios, la reducción de conflictos en los horarios generados, la satisfacción del usuario (administradores, estudiantes) y la tasa de adopción del nuevo sistema en Jala University, permitiendo medir objetivamente el impacto y la mejora lograda.

\subsection{Diagramas}
Los diagramas son herramientas visuales esenciales en la ingeniería de software para comunicar la estructura, el comportamiento y la arquitectura de un sistema de manera clara y concisa. Utilizar diferentes tipos de diagramas ayuda a comprender y documentar distintos aspectos del sistema web de gestión académica \parencite{Fowler2003}.

\subsubsection{Diagrama de requerimientos}
Los diagramas de requerimientos, como los diagramas de casos de uso (\texttt{UML}), representan las interacciones entre los actores (usuarios como estudiantes, administradores académicos) y el sistema, describiendo las funcionalidades que este debe ofrecer (e.g., "Consultar horario", "Generar propuesta de horario", "Predecir elegibilidad de estudiante", "Administrar cursos") \parencite{Jacobson1992}. Estos diagramas son fundamentales en las etapas iniciales para definir el alcance del proyecto y asegurar que se comprendan y capturen las necesidades de los usuarios de Jala University.

\subsubsection{Diagrama C4}
El modelo C4 (Context, Containers, Components, Code) proporciona un marco para visualizar la arquitectura de software en diferentes niveles de abstracción, facilitando la comunicación entre distintos roles (desde negocio hasta desarrolladores) \parencite{BrownC4}. Es especialmente útil para describir sistemas complejos como el propuesto, permitiendo entender cómo encaja en el ecosistema de Jala University y cómo se estructura internamente.

\paragraph{System context}
El diagrama de Contexto (Nivel 1 de C4) muestra el sistema de software en su totalidad como una caja negra, identificando sus interacciones con los usuarios principales (estudiantes, administradores) y otros sistemas externos con los que se integra (e.g., sistema de registro de estudiantes de Jala University, sistema de autenticación). Este diagrama establece el alcance y los límites del sistema de gestión del calendario académico.

\paragraph{Containers}
El diagrama de Contenedores (Nivel 2 de C4) descompone el sistema en sus principales bloques ejecutables o desplegables, como aplicaciones web, APIs, bases de datos o microservicios (e.g., Web App Frontend, API Gateway, Servicio de Horarios, Servicio de Predicción ML, Base de Datos Académica). Muestra las responsabilidades de alto nivel de cada contenedor y las tecnologías principales utilizadas, así como las interacciones entre ellos.

\paragraph{Components}
El diagrama de Componentes (Nivel 3 de C4) detalla la estructura interna de un contenedor específico, mostrando los principales componentes (agrupaciones lógicas de código, como clases o módulos) y sus interacciones dentro de ese contenedor. Por ejemplo, podría mostrar los componentes dentro del "Servicio de Horarios", como el "Generador de Horarios", el "Validador de Restricciones" y el "Repositorio de Horarios".

\paragraph{Code}
El diagrama de Código (Nivel 4 de C4, opcional) ofrece una vista detallada a nivel de clases o entidades específicas dentro de un componente, utilizando notaciones como diagramas de clases \texttt{UML}. Este nivel es útil para desarrolladores que necesitan entender la implementación detallada de una parte específica del sistema, como las clases que implementan un algoritmo de predicción o las entidades del dominio académico.

\subsubsection{WebML}
Web Modeling Language (WebML) es un lenguaje de modelado visual específico para el diseño conceptual de aplicaciones web complejas, centrándose en la estructura de la información, la navegación y la composición de las páginas \parencite{Ceri2000}. Utilizar WebML podría ayudar a diseñar la estructura hipertextual del sistema de gestión académica, definiendo las unidades de contenido (e.g., información del curso, horario del estudiante), las páginas y los enlaces de navegación entre ellas de manera formal antes de la implementación del frontend.

\subsection{Metodología de investigación}
\subsubsection{Definición}
El caso de estudio es un método empírico cuyo objetivo es investigar fenómenos contemporáneos en su contexto.

El caso de estudio tiene cuatro tipos diferentes de metodologías de investigación, que son:
\begin{itemize}
    \item Exploratorio: Se trata de generar ideas para hipótesis; responde a la pregunta "¿Qué está sucediendo?".
    \item Descriptivo: Presenta una descripción exhaustiva del fenómeno.
    \item Explicativo: Es una explicación del problema. No siempre en forma de una relación causal.
    \item De mejora: Mejora un cierto aspecto del fenómeno estudiado.
\end{itemize}

Un caso de estudio de tipo \textit{Positivista} se centra en recopilar evidencia para proposiciones formales a partir de la medición de variables, la prueba de hipótesis y la extracción de inferencias de muestras para comprender un fenómeno, mientras que un estudio de caso de tipo \textit{Interpretativo} recopila información a través de la interpretación que hace el participante de su contexto.

Se espera que un caso de estudio tenga: (1) preguntas de investigación, establecidas desde el principio, (2) los datos se recopilen de manera planificada y consistente, (3) se realicen inferencias a partir de los datos para responder a las preguntas de investigación, (4) explore un fenómeno, (5) las amenazas a la validez del proyecto se aborden de manera sistemática.

\subsubsection{Protocolo del caso de estudio}
El protocolo del caso de estudio es un documento que contiene información sobre las decisiones de diseño e información sobre cómo llevar a cabo el proyecto.

\begin{longtable}{l|p{3in}}
\caption{Componentes del Protocolo del caso de estudio} \\
\hline
Sección & Contenido \\
\hline
\endfirsthead
Preámbulo & Información sobre el propósito del protocolo, directrices para el almacenamiento de datos y documentos, publicación \\ % Translated content
Procedimientos generales & Breve descripción general del proyecto de investigación y del método de investigación de caso \\ % Translated content
Instrumentos de investigación & Guías de entrevista, cuestionarios, etc., que se utilizarán para garantizar la recopilación coherente de datos. \\ % Translated content
Directrices para el análisis de datos & Descripción detallada de los procedimientos de análisis de datos, incluido el esquema de datos \\ % Translated content
\end{longtable}


\subsection{Tecnologías para Backend}
La elección de las tecnologías para el backend es crucial, ya que impacta directamente en el rendimiento, la escalabilidad, la seguridad y la facilidad de desarrollo y mantenimiento del servidor que aloja la lógica de negocio y gestiona los datos del sistema de gestión académica \parencite{FowlerBackend}.

\subsubsection{Programming languages}
La selección del lenguaje de programación para el backend depende de factores como el rendimiento requerido, el ecosistema de librerías disponibles (especialmente para ML y web), la experiencia del equipo y la compatibilidad con la infraestructura existente.

\paragraph{C\#}
\texttt{C\#} es un lenguaje de programación moderno, orientado a objetos y fuertemente tipado, desarrollado por Microsoft y ejecutado sobre la plataforma \texttt{.NET}. Ofrece un ecosistema robusto para el desarrollo web (\texttt{ASP.NET Core}), buen rendimiento y herramientas maduras, además de contar con librerías para ML (\texttt{ML.NET}), lo que lo convierte en una opción viable y productiva para construir los servicios backend del sistema \parencite{MicrosoftCSharp}.

\paragraph{Python}
Python es un lenguaje interpretado, dinámico y multipropósito, extremadamente popular en el ámbito de la ciencia de datos y el Machine Learning gracias a su sintaxis sencilla y a un vasto ecosistema de librerías especializadas (como \texttt{Scikit-learn}, \texttt{TensorFlow}, \texttt{PyTorch}) \parencite{PythonSoftwareFoundation}. Su facilidad de uso y las potentes capacidades para ML lo hacen ideal para desarrollar el servicio de predicción de elegibilidad, pudiendo integrarse con otros servicios backend desarrollados en \texttt{C\#} u otros lenguajes a través de APIs.

\subsubsection{Databases}
La elección de la base de datos adecuada es fundamental para almacenar y recuperar eficientemente la información académica, los horarios generados, los datos de entrenamiento para ML y la configuración del sistema.

\paragraph{Relational}
Las bases de datos relacionales (como \texttt{PostgreSQL}, \texttt{SQL Server}, \texttt{MySQL}) organizan los datos en tablas con esquemas predefinidos y utilizan \texttt{SQL} (Structured Query Language) para las consultas, garantizando la consistencia de los datos a través de transacciones \texttt{ACID} \parencite{Date2003}. Son ideales para almacenar datos estructurados con relaciones bien definidas, como la información de cursos, estudiantes, profesores y asignaciones académicas, que forman el núcleo del sistema de gestión de Jala University.

\paragraph{Non-Relational}
Las bases de datos \texttt{NoSQL} (Not Only SQL) ofrecen modelos de datos más flexibles (documental, clave-valor, columnar, grafo) y suelen priorizar la escalabilidad y la disponibilidad sobre la consistencia estricta (modelo \texttt{BASE}) \parencite{SadalegeFowler2012}. Podrían ser útiles para almacenar datos menos estructurados o de gran volumen, como logs del sistema, resultados intermedios de la generación de horarios, o quizás para perfiles de usuario o configuraciones flexibles, complementando a la base de datos relacional principal.

\subsection{Tecnologías para Frontend}
Las tecnologías frontend determinan cómo se construye la interfaz de usuario interactiva que los usuarios finales (estudiantes, administradores) utilizarán para interactuar con el sistema de gestión del calendario académico.

\subsubsection{TypeScript - React}
\texttt{TypeScript} es un superconjunto de \texttt{JavaScript} que añade tipado estático opcional, mejorando la robustez y mantenibilidad del código frontend, especialmente en proyectos grandes \parencite{MicrosoftTypeScript}. \texttt{React} es una popular biblioteca de \texttt{JavaScript} para construir interfaces de usuario declarativas y basadas en componentes \parencite{FacebookReact}. La combinación de \texttt{TypeScript} y \texttt{React} ofrece un entorno de desarrollo productivo y seguro para crear interfaces complejas y reactivas, adecuadas para visualizar horarios, formularios de gestión y resultados de predicciones de manera eficiente.

\subsection{Herramientas de diseño}
Las herramientas de diseño facilitan la creación de prototipos, wireframes y diseños visuales de la interfaz de usuario (UI) y la experiencia de usuario (UX), permitiendo iterar y validar ideas antes de escribir código.

\subsubsection{Figma}
Figma es una herramienta de diseño de interfaces basada en la web y colaborativa, que permite crear prototipos interactivos, sistemas de diseño y colaborar en tiempo real entre diseñadores y desarrolladores \parencite{Figma}. Es ideal para diseñar las pantallas del sistema de gestión académica, definir flujos de usuario y crear un lenguaje visual consistente para la aplicación de Jala University.

\subsubsection{Canvas}
El Business Model Canvas o herramientas similares de "canvas" (lienzo) como el Lean Canvas, son plantillas estratégicas que ayudan a visualizar y desarrollar modelos de negocio o propuestas de valor de forma estructurada y concisa \parencite{OsterwalderPigneur2010}. Aunque no es una herramienta de diseño de UI, puede usarse en las fases iniciales para definir la propuesta de valor del sistema, identificar segmentos de usuarios clave (estudiantes, administradores de Jala U) y alinear el proyecto con los objetivos institucionales.

\subsection{Organizador de tareas}
Las herramientas de gestión de tareas son esenciales para planificar, organizar y seguir el progreso del trabajo en un proyecto de desarrollo de software, especialmente cuando se utilizan metodologías ágiles.

\subsubsection{Clickup}
ClickUp es una plataforma de productividad y gestión de proyectos todo en uno que ofrece múltiples vistas (listas, tableros Kanban, calendarios, Gantt), personalización de flujos de trabajo y funcionalidades para la colaboración en equipo \parencite{ClickUp}. Podría utilizarse para gestionar el backlog del producto, planificar sprints (si se usa Scrum o Scrumban), asignar tareas y seguir el progreso general del desarrollo del sistema para Jala University.

\subsubsection{Taskwarrior}
Taskwarrior es una herramienta de gestión de tareas de código abierto y basada en línea de comandos, que permite organizar listas de tareas pendientes de forma eficiente y flexible \parencite{Taskwarrior}. Es una opción potente para desarrolladores que prefieren trabajar en la terminal, aunque requiere una curva de aprendizaje y es más adecuada para la gestión individual de tareas dentro del proyecto.

\subsubsection{Taiga}
Taiga es una plataforma de gestión de proyectos ágil, de código abierto y centrada en Scrum y Kanban, que ofrece tableros visuales, gestión de backlogs, seguimiento de issues y wikis \parencite{Taiga}. Representa una alternativa open-source a herramientas como Jira o ClickUp, adecuada para equipos que buscan una solución auto-alojada o gratuita para implementar Scrum o Scrumban en el desarrollo del proyecto.

\subsection{Herramientas de versionamiento}
El control de versiones es indispensable en el desarrollo de software para gestionar los cambios en el código fuente a lo largo del tiempo, facilitar la colaboración entre desarrolladores y permitir la reversión a estados anteriores.

\subsubsection{GitHub}
GitHub es una plataforma de desarrollo colaborativo basada en \texttt{Git} que ofrece hospedaje de repositorios, seguimiento de issues, revisión de código (Pull Requests), integración continua y otras herramientas para el ciclo de vida del desarrollo de software \parencite{GitHub}. Es la plataforma de facto para muchos proyectos de código abierto y empresariales, y sería una opción robusta para alojar el código del sistema de Jala University, gestionar la colaboración y automatizar flujos de trabajo.

\subsubsection{Sourcehut}
SourceHut es una suite de herramientas de desarrollo de software de código abierto, enfocada en la simplicidad, la estabilidad y la filosofía Unix, que ofrece hospedaje \texttt{Git}, seguimiento de tickets, listas de correo, CI/CD y otras funcionalidades \parencite{SourceHut}. Representa una alternativa más minimalista y centrada en el desarrollador a plataformas como GitHub o GitLab, atractiva para quienes valoran la transparencia y el control sobre sus herramientas de desarrollo.

\subsubsection{Nomenclatura de ramas, commits y pull requests}
Establecer una convención clara para nombrar ramas (e.g., \texttt{feature/nombre-funcionalidad}, \texttt{bugfix/descripcion-corta}, \texttt{release/v1.0}), escribir mensajes de commit significativos (e.g., siguiendo el formato Conventional Commits \parencite{ConventionalCommits}) y gestionar Pull Requests (PRs) de manera estructurada (con descripciones claras, revisiones obligatorias) es crucial para mantener un historial de cambios limpio, facilitar la revisión de código y mejorar la colaboración dentro del equipo de desarrollo del proyecto.

\subsection{Tecnología de diagramado}
Las herramientas de diagramado asistido por software permiten crear y mantener diagramas de arquitectura, diseño y procesos de manera eficiente y consistente, a menudo integrándose con el código o sistemas de control de versiones.

\subsubsection{Structurizr}
Structurizr es un conjunto de herramientas (bibliotecas de código abierto y una plataforma web) para crear diagramas de arquitectura de software basados en el modelo C4, utilizando el enfoque de "diagramas como código", donde los modelos se definen en código (e.g., Java, \texttt{C\#}, Python) y los diagramas se generan a partir de él \parencite{BrownStructurizr}. Esto asegura que la documentación arquitectónica se mantenga sincronizada con el código y facilita la automatización de la generación de diagramas C4 para el sistema de Jala University.

\subsubsection{PlantUML}
PlantUML es una herramienta de código abierto que permite generar diversos diagramas \texttt{UML} (secuencia, casos de uso, clases, actividad, componentes), así como otros tipos de diagramas (Arquitectura C4, ERD, Wireframe), a partir de una descripción textual simple \parencite{PlantUML}. Es muy útil para crear rápidamente diagramas técnicos y mantenerlos bajo control de versiones junto con el código fuente, facilitando la documentación visual del diseño del sistema.

\subsection{Plan de pruebas}
Un plan de pruebas sistemático es esencial para asegurar la calidad, fiabilidad y corrección del sistema web desarrollado, verificando que cumple con los requerimientos funcionales y no funcionales definidos.

\subsubsection{ISO 9126}
La norma ISO/IEC 9126 (reemplazada en parte por ISO/IEC 25010) define un modelo de calidad para el software, clasificando los atributos de calidad en seis características principales: Funcionalidad, Fiabilidad, Usabilidad, Eficiencia, Mantenibilidad y Portabilidad \parencite{ISO9126}. Utilizar este modelo como marco para el plan de pruebas del sistema de gestión académica permite definir criterios de aceptación claros y métricas específicas para evaluar cada aspecto de la calidad del software, asegurando una cobertura completa y sistemática de las pruebas.
\subsection{Metodología de investigación}
\subsubsection{Definición}
El caso de estudio es un método empírico cuyo objetivo es investigar fenómenos contemporáneos en su contexto.

El caso de estudio tiene cuatro tipos diferentes de metodologías de investigación, que son:
\begin{itemize}
    \item Exploratorio: Se trata de generar ideas para hipótesis; responde a la pregunta "¿Qué está sucediendo?".
    \item Descriptivo: Presenta una descripción exhaustiva del fenómeno.
    \item Explicativo: Es una explicación del problema.
    No siempre en forma de una relación causal.
    \item De mejora: Mejora un cierto aspecto del fenómeno estudiado.
\end{itemize}

Un caso de estudio de tipo \textit{Positivista} se centra en recopilar evidencia para proposiciones formales a partir de la medición de variables, la prueba de hipótesis y la extracción de inferencias de muestras para comprender un fenómeno, mientras que un estudio de caso de tipo \textit{Interpretativo} recopila información a través de la interpretación que hace el participante de su contexto.

Se espera que un caso de estudio tenga: (1) preguntas de investigación, establecidas desde el principio, (2) los datos se recopilen de manera planificada y consistente, (3) se realicen inferencias a partir de los datos para responder a las preguntas de investigación, (4) explore un fenómeno, (5) las amenazas a la validez\footnote{Las \textbf{amenazas a la validez} en investigación se refieren a factores o influencias que podrían llevar a conclusiones incorrectas sobre el estudio.} del proyecto se aborden de manera sistemática.

\subsubsection{Protocolo del caso de estudio}
El protocolo del caso de estudio es un documento que contiene información sobre las decisiones de diseño e información sobre cómo llevar a cabo el proyecto.

\begin{table}[h]
\caption{Componentes del Protocolo del caso de estudio}
\begin{tabularx}{\textwidth}{@{}lX@{}}
\toprule
Sección & Contenido \\
\midrule
Preámbulo & Información sobre el propósito del protocolo, directrices para el almacenamiento de datos y documentos, publicación \\
Procedimientos generales & Breve descripción general del proyecto de investigación y del método de investigación de caso \\
Instrumentos de investigación & Guías de entrevista, cuestionarios, etc., que se utilizarán para garantizar la recopilación coherente de datos. \\
Directrices para el análisis de datos & Descripción detallada de los procedimientos de análisis de datos, incluido el esquema de datos. \\
\bottomrule
\end{tabularx}
\end{table}

\subsection{Herramienta para gestionar tareas}
Las herramientas de gestión de tareas son esenciales para planificar, organizar y seguir el progreso del trabajo en un proyecto de desarrollo de software, especialmente cuando se utilizan metodologías ágiles.

\subsubsection{Clickup}
ClickUp es una plataforma de productividad y gestión de proyectos todo en uno que ofrece múltiples vistas (listas, tableros Kanban, calendarios, Gantt), personalización de flujos de trabajo y funcionalidades para la colaboración en equipo \parencite{ClickUp}.
Podría utilizarse para gestionar el backlog del producto, planificar sprints, asignar tareas y seguir el progreso general del desarrollo del sistema.

\subsubsection{Taskwarrior}
Taskwarrior es una herramienta de gestión de tareas de código abierto y basada en línea de comandos, que permite organizar listas de tareas pendientes de forma eficiente y flexible \parencite{Taskwarrior}.
Es una opción potente para desarrolladores que prefieren trabajar en la terminal, aunque requiere una curva de aprendizaje y es más adecuada para la gestión individual de tareas dentro del proyecto.

Como se observa en la Figura~\ref{fig:taskWarriorTaskOutput}, Taskwarrior mantiene metadatos de cada tarea, lo que permite el postprocesamiento\footnote{El \textbf{postprocesamiento} se refiere al procesamiento de datos que se realiza después de que han sido generados o recopilados.
En este caso, sería analizar los datos de Taskwarrior para crear visualizaciones o informes adicionales.} para crear gráficos de trabajo pendiente (burndown charts) y es posible agregar ADU \footnote{Un \textbf{Atributo Definido por el Usuario (ADU)} permite al usuario definir atributos.} para una mejor personalización.

\begin{figure}
	\caption{Ejemplo de salida de Taskwarrior}\label{fig:taskWarriorTaskOutput}
	\begin{verbatim}
Name               Value
------------------ ---------------------------------------------------------
Description        add figures to marco teorico
Status             Pending
Project            kawsay
Entered            2025-06-02 17:57:24 (45min)
Start              2025-06-02 18:09:17
Last modified      2025-06-02 18:09:17 (33min)
Tags               marco_teorico
Virtual tags       ACTIVE LATEST PENDING PROJECT READY TAGGED UDA UNBLOCKED
UUID               59a7c053-bc1d-4dcb-8b21-ea91646f6f41
Urgency            5.8
Estimate           45
	\end{verbatim}
\end{figure}

\subsubsection{Taiga}
Taiga es una plataforma de gestión de proyectos ágil, de código abierto y centrada en Scrum y Kanban, que ofrece tableros visuales, gestión de backlogs, seguimiento de issues\footnote{Un \textbf{issue} (incidencia o problema) en el contexto de la gestión de proyectos de software es una unidad de trabajo para rastrear una tarea, mejora, error o cualquier otro elemento que necesite ser abordado.} y wikis \parencite{Taiga}.
Representa una alternativa open-source a herramientas como Jira o ClickUp, adecuada para equipos que buscan una solución auto-alojada o gratuita para implementar metodologias agiles en el desarrollo del proyecto.

\paragraph{Conclusión:}
Para el desarrollo del sistema se utilizarán TaskWarrior y Taiga.
Porque ambas herramientas pueden combinarse para una gestión granular de tareas y también de alto nivel;
la API de Taiga abre posibilidades para otras automatizaciones relacionadas con la gestión de proyectos.

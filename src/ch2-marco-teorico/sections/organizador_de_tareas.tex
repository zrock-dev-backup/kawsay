\subsection{Organizador de tareas}
Las herramientas de gestión de tareas son esenciales para planificar, organizar y seguir el progreso del trabajo en un proyecto de desarrollo de software, especialmente cuando se utilizan metodologías ágiles.

\subsubsection{Clickup}
ClickUp es una plataforma de productividad y gestión de proyectos todo en uno que ofrece múltiples vistas (listas, tableros Kanban, calendarios, Gantt), personalización de flujos de trabajo y funcionalidades para la colaboración en equipo \parencite{ClickUp}.
Podría utilizarse para gestionar el backlog del producto, planificar sprints, asignar tareas y seguir el progreso general del desarrollo del sistema.

\subsubsection{Taskwarrior}
Taskwarrior es una herramienta de gestión de tareas de código abierto y basada en línea de comandos, que permite organizar listas de tareas pendientes de forma eficiente y flexible \parencite{Taskwarrior}.
Es una opción potente para desarrolladores que prefieren trabajar en la terminal, aunque requiere una curva de aprendizaje y es más adecuada para la gestión individual de tareas dentro del proyecto.

\subsubsection{Taiga}
Taiga es una plataforma de gestión de proyectos ágil, de código abierto y centrada en Scrum y Kanban, que ofrece tableros visuales, gestión de backlogs, seguimiento de issues y wikis \parencite{Taiga}.
Representa una alternativa open-source a herramientas como Jira o ClickUp, adecuada para equipos que buscan una solución auto-alojada o gratuita para implementar Scrum o Scrumban en el desarrollo del proyecto.

In \paragraph{conclusion} for the development of the system TaskWarrior and Taiga will be used.
Because both tools can be combined for a granular management of tasks and high level as well; Taiga's API opens possibilities for other automations related to project management.

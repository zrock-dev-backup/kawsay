\subsection{Tecnología de diagramado}
Las herramientas de diagramado asistido por software permiten crear y mantener diagramas de arquitectura, diseño y procesos de manera eficiente y consistente, a menudo integrándose con el código o sistemas de control de versiones.

\subsubsection{Structurizr}
Structurizr es un conjunto de herramientas (bibliotecas de código abierto y una plataforma web) para crear diagramas de arquitectura de software basados en el modelo C4, utilizando el enfoque de "diagramas como código", donde los modelos se definen en código (e.g., Java, Python) y los diagramas se generan a partir de él \parencite{BrownStructurizr}.
Esto asegura que la documentación arquitectónica se mantenga sincronizada con el código y facilita la automatización de la generación de diagramas C4 para el sistema de Jala University.

\subsubsection{PlantUML}
PlantUML es una herramienta de código abierto que permite generar diversos diagramas \texttt{UML} (secuencia, casos de uso, clases, actividad, componentes), así como otros tipos de diagramas (Arquitectura C4, ERD, Wireframe), a partir de una descripción textual simple \parencite{PlantUML}.
Es muy útil para crear rápidamente diagramas técnicos y mantenerlos bajo control de versiones junto con el código fuente, facilitando la documentación visual del diseño del sistema.


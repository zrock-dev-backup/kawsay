\subsection{Tecnologías para Backend}
La elección de las tecnologías para el backend es crucial, ya que impacta directamente en el rendimiento, la escalabilidad, la seguridad y la facilidad de desarrollo y mantenimiento del servidor que aloja la lógica de negocio y gestiona los datos del sistema de gestión académica \parencite{FowlerBackend}.

\subsubsection{Lenguajes de programación}
La selección del lenguaje de programación para el backend depende de factores como el rendimiento requerido, el ecosistema de librerías disponibles (especialmente para ML y web), la experiencia del equipo y la compatibilidad con la infraestructura existente.

\paragraph{C\#}
Es un lenguaje de programación moderno, orientado a objetos y fuertemente tipado, desarrollado por Microsoft y ejecutado sobre la plataforma \texttt{.NET}\footnote{La \textbf{plataforma .NET} es un marco de desarrollo de software gratuito y de código abierto para crear diferentes tipos de aplicaciones, como web, móviles, de escritorio, juegos e IoT.
Desarrollado por Microsoft, incluye lenguajes como C\#, F\# y VB.NET, y un amplio conjunto de bibliotecas y herramientas.}.
Ofrece un ecosistema robusto para el desarrollo web (\texttt{ASP.NET Core}\footnote{\textbf{ASP.NET Core} es un framework de código abierto y multiplataforma para crear aplicaciones web modernas, basadas en la nube y conectadas a Internet, desarrollado por Microsoft.
Es una reescritura de ASP.NET y funciona sobre .NET Core o .NET Framework.}), buen rendimiento y herramientas maduras.

\paragraph{Python}
es un lenguaje interpretado, dinámico y multipropósito, extremadamente popular en el ámbito de la ciencia de datos y el Machine Learning gracias a su sintaxis sencilla y a un vasto ecosistema de librerías especializadas como \texttt{Scikit-learn}\footnote{\textbf{Scikit-learn} es una popular biblioteca de machine learning de código abierto para Python.
Proporciona herramientas simples y eficientes para el análisis predictivo de datos, construida sobre NumPy, SciPy y matplotlib.}.
Su facilidad de uso y las potentes capacidades para ML lo hacen ideal para desarrollar el servicio de predicción de elegibilidad, pudiendo integrarse con otros servicios backend desarrollados en \texttt{C\#} u otros lenguajes a través de APIs.

\subsubsection{Bases de datos}
La elección de la base de datos adecuada es fundamental para almacenar y recuperar eficientemente la información académica, los horarios generados, los datos de entrenamiento para ML y la configuración del sistema.

\paragraph{Relacionales}
Las bases de datos relacionales (como \texttt{PostgreSQL}\footnote{\textbf{PostgreSQL} es un potente sistema de gestión de bases de datos relacionales de objetos de código abierto, conocido por su fiabilidad, robustez de características y rendimiento.}, \texttt{SQL Server}\footnote{\textbf{SQL Server} es un sistema de gestión de bases de datos relacionales desarrollado por Microsoft.
Ofrece una amplia gama de herramientas de análisis de datos, generación de informes e integración.}, \texttt{MySQL}\footnote{\textbf{MySQL} es un popular sistema de gestión de bases de datos relacionales de código abierto, ampliamente utilizado en aplicaciones web y como parte de la pila de software LAMP (Linux, Apache, MySQL, PHP/Python/Perl).}) organizan los datos en tablas con esquemas predefinidos y utilizan \texttt{SQL} (Structured Query Language)\footnote{\textbf{SQL (Structured Query Language)} es un lenguaje estándar utilizado para gestionar y manipular bases de datos relacionales.
Permite realizar consultas, insertar, actualizar y eliminar datos, así como definir y modificar la estructura de la base de datos.} para las consultas, garantizando la consistencia de los datos a través de transacciones \texttt{ACID} \parencite{Date2003}.
Son ideales para almacenar datos estructurados con relaciones bien definidas, como la información de cursos, estudiantes, profesores y asignaciones académicas, que forman el núcleo del sistema de gestión del calendario academíco.

\paragraph{No relacionales}
Las bases de datos \texttt{NoSQL} (Non SQL) ofrecen modelos de datos más flexibles que los relacionales tradicionales. Estos modelos incluyen:
el documental, donde una \textbf{base de datos documental} (un tipo de base de datos NoSQL) almacena datos en forma de documentos, a menudo en formato JSON o BSON, y cada documento es una estructura de datos auto-contenida;
el de clave-valor, en el que una \textbf{base de datos clave-valor} (un tipo simple de base de datos NoSQL) almacena datos como una colección de pares clave-valor, donde cada clave es única;
el columnar, mediante el cual una \textbf{base de datos columnar} almacena datos en columnas en lugar de filas, lo que puede ser más eficiente para ciertas cargas de trabajo analíticas donde se accede a subconjuntos de columnas;
y el de grafos, implementado por una \textbf{base de datos de grafos} que utiliza nodos, ejes y propiedades para representar y almacenar datos, resultando ideal para modelar relaciones complejas entre entidades.
Estos diversos modelos \texttt{NoSQL} suelen priorizar la escalabilidad y la disponibilidad sobre la consistencia estricta (modelo \texttt{BASE}) \parencite{SadalegeFowler2012}.
Podrían ser útiles para almacenar datos menos estructurados o de gran volumen, como logs del sistema, resultados intermedios de la generación de horarios, o quizás para perfiles de usuario o configuraciones flexibles, complementando a la base de datos relacional principal.

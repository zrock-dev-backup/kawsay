\subsection{Tecnologías para Backend}
La elección de las tecnologías para el backend es crucial, ya que impacta directamente en el rendimiento, la escalabilidad, la seguridad y la facilidad de desarrollo y mantenimiento del servidor que aloja la lógica de negocio y gestiona los datos del sistema de gestión académica \parencite{FowlerBackend}.

\subsubsection{Lenguajes de programación}
La selección del lenguaje de programación para el backend depende de factores como el rendimiento requerido, el ecosistema de librerías disponibles (especialmente para ML y web), la experiencia del equipo y la compatibilidad con la infraestructura existente.

\paragraph{C\#}
es un lenguaje de programación moderno, orientado a objetos y fuertemente tipado, desarrollado por Microsoft y ejecutado sobre la plataforma \texttt{.NET}.
Ofrece un ecosistema robusto para el desarrollo web (\texttt{ASP.NET Core}), buen rendimiento y herramientas maduras, además de contar con librerías para ML (\texttt{ML.NET}), lo que lo convierte en una opción viable y productiva para construir los servicios backend del sistema \parencite{MicrosoftCSharp}.

\paragraph{Python}
es un lenguaje interpretado, dinámico y multipropósito, extremadamente popular en el ámbito de la ciencia de datos y el Machine Learning gracias a su sintaxis sencilla y a un vasto ecosistema de librerías especializadas (como \texttt{Scikit-learn}, \texttt{TensorFlow}, \texttt{PyTorch}) \parencite{PythonSoftwareFoundation}.
Su facilidad de uso y las potentes capacidades para ML lo hacen ideal para desarrollar el servicio de predicción de elegibilidad, pudiendo integrarse con otros servicios backend desarrollados en \texttt{C\#} u otros lenguajes a través de APIs.

\subsubsection{Bases de datos}
La elección de la base de datos adecuada es fundamental para almacenar y recuperar eficientemente la información académica, los horarios generados, los datos de entrenamiento para ML y la configuración del sistema.

\paragraph{Relacionales}
Las bases de datos relacionales (como \texttt{PostgreSQL}, \texttt{SQL Server}, \texttt{MySQL}) organizan los datos en tablas con esquemas predefinidos y utilizan \texttt{SQL} (Structured Query Language) para las consultas, garantizando la consistencia de los datos a través de transacciones \texttt{ACID} \parencite{Date2003}.
Son ideales para almacenar datos estructurados con relaciones bien definidas, como la información de cursos, estudiantes, profesores y asignaciones académicas, que forman el núcleo del sistema de gestión del calendario academíco.

\paragraph{No relacionales}
Las bases de datos \texttt{NoSQL} (Not Only SQL) ofrecen modelos de datos más flexibles (documental, clave-valor, columnar, grafo) y suelen priorizar la escalabilidad y la disponibilidad sobre la consistencia estricta (modelo \texttt{BASE}) \parencite{SadalegeFowler2012}.
Podrían ser útiles para almacenar datos menos estructurados o de gran volumen, como logs del sistema, resultados intermedios de la generación de horarios, o quizás para perfiles de usuario o configuraciones flexibles, complementando a la base de datos relacional principal.

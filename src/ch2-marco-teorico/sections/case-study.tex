\subsection{Caso de estudio}
Los procesos de generación de horarios e inscripción de estudiantes se han identificado mediante el análisis del Apéndice~\ref{sec:appendixIStarAnalysis}.

\subsubsection{Procesos}
\paragraph{Proceso de generación de horarios}
Consiste en la producción del horario para un módulo que satisfaga el contenido académico de cada módulo definido en el Catálogo de Estudiantes de Jala University.

En este proceso, se produce un conflicto cuando un profesor solicita un cambio de horario por una razón externa; en esta situación, el registrador debe encontrar otra franja horaria disponible, teniendo en cuenta la disponibilidad de la clase y del profesor.
A veces, cuando esta operación no es posible, se reemplaza al profesor.

\paragraph{Proceso de inscripción de estudiantes}
El proceso de inscripción se realiza al inicio de cada módulo; los estudiantes se dividen en cohortes, las cuales se distribuyen en grupos y secciones.
En este proceso, se produce un conflicto de inscripción cuando un estudiante reprueba una asignatura y necesita repetirla; al inscribir a este estudiante en la clase, a veces también tendrá otra clase programada en el mismo período.

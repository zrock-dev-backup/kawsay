\subsection{Case study}
Timetable generation and student enrollment processes have been identified with the analysis of Appendix~\ref{sec:appendixIStarAnalysis}

\subsubsection{Processes}
\paragraph{Timetable generation process} Consist on the production of the timetable for a modulethat satisfies the academic content of each module in defined in the Jala University's Student Catalog.

In this process a clash is produced when a teacher requests a change of schedule because of an external reason, in this situation the registrar has to find another available timeslot taking into account the class availability and the teacher.
Sometimes when this operation is not possible then the teacher is replaced.

\paragraph{Student enrollment process} The enrollment process is performed at the start of each module, students are divided in cohorts which are distributed in groups and sections.
In this process a enrollment clash is produced when a student fails a course and needs to repeated, when enrolling this student to the class sometimes she will also have another class happening in the same period.

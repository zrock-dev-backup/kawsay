\subsection{Case study}
In JalaU the following actors play a role in the context of academic timetable generation and student enrollment.
\begin{itemize}
	\item Registrar: Enrolls students and generates timetable.
	\item Teacher: Teach students (Professor, Faculty praticioner)
	\item Student: A registered student that can be enrolled.
	\item Academic coordinator: In charge of hiring teaching staff.
	\item SIS: System holds student data.
	\item Student Services: Student helpers
\end{itemize}

Figure~\ref{fig:strategicDependenciesDiagram} displays the relationships among the identified actors.

\begin{figure}[htb]
	\caption{Strategic Dependencies Diagram - Fuente: (Elaboracion propia)}\label{fig:strategicDependenciesDiagram}
	\begin{center}
		\includegraphics[width=0.95\textwidth]{strategic-dependencies.pdf}
	\end{center}
\end{figure}

\begin{table}
	\caption{SD relationships description}\label{tab:strategicDependencies}
	\begin{center}
		\begin{tabularx}{\textwidth}{p{2in}X} \toprule
			\textbf{Relationship} &  \textbf{Description}\\
			\midrule
			
			Teacher- Registrar & A teacher needs to know his lecture timetables in order to deliver efective education. \\
			Registrar - SIS & Registrar needs student's information in order to enroll them\\
			Registrar - Teacher & Registrar needs to know a student grades in order to asses their elegibility to be enrolled in next courses or repeat one. \\
			Student services - Registrar & Student services wants to help students resolve enrollment and timetable issues which can be achieved through collaboratio with registrar \\
			Registrar - Academic coordinator & In order for registrar to select a suitable teacher for a class they need to have teacher data\\
			\bottomrule
		\end{tabularx}
	\end{center}
\end{table}

\subsubsection{Registrar}
Registrar is the actor that generates the timetable, for this purpose depends on Academic Coordinator for teaching staff data and SIS for student data.

Registrar is the main actor in the timetable and student enrollment management context.
Figure~\ref{fig:actorBoundaryRegistrar} is a representation of Registrar's strategic rationale.

Registrar office is concerned with student enrollment and timetable management.
Student depends upon Registrar to be enrolled in a course meaning Registrar has the task of enrolling students to class implying the generation of a timetable and planning ahead to have resources available for the next module or term.

This actor is also concerned with minimizing administrative overhead, because it needs to generate other kind of reports outside of the context of this case study, for which the softgoal "Minimize administrative overhead" has been identified.
She also has the softgoal of "Reducing student double booking" i.e. when a student is enrolled on two classes that happen at the same time.

\begin{landscape}
	\begin{figure}
		\caption{Registrar strategic rationale model - Fuente: Elaboracion propia}
		\includegraphics[width=1.4\textwidth]{registrar.pdf}
		\label{fig:actorBoundaryRegistrar}
	\end{figure}
\end{landscape}

\subsubsection{Student}
Figure~\ref{fig:actorBoundaryStudent} describes a Student's strategic rationale.
Student needs Registrar to be enrolled in a course, this actor peforms tasks to graduate from university and is interested in protecting their scolarship by following the university's guidelines.
\begin{landscape}
	\begin{figure}
		\caption{Student strategic rationale model - Fuente: Elaboracion propia}
		\includegraphics[width=1.4\textwidth]{student.pdf}
		\label{fig:actorBoundaryStudent}
	\end{figure}
\end{landscape}

\subsubsection{Teacher}
As we can see in Figure~\ref{fig:actorBoundaryTeacher} a teacher is interested in becoming \textit{Hired} to be able to teach a class.
Teacher needs \textit{Class schedule flexibility} to request schedule changes.

\begin{figure}
	\centering
	\caption{Teacher strategic rationale model - Fuente: Elaboracion propia}
	\label{fig:actorBoundaryTeacher}
	\includegraphics[width=.8\textwidth]{teacher.pdf}
\end{figure}

\subsubsection{Timetable clashes}
Timetable and student enrollment management is a task that faces two types of clashes.

\paragraph{Student double booking} Produced when a student fails a course and needs to repeated, when enrolling this student to the class sometimes she will also have another class happening in the same hour.
In this situation the registrar tries to find another class for when the student is available.

\paragraph{Teacher change of schedule} Produced when a teacher requests a change of schedule because of an external reason, in this situation the registrar has to find another available timeslot taking into account the class availability and the teacher.
Sometimes when this operation is not possible then the teacher is replaced.

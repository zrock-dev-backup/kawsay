\subsection{i* Framework}
El marco de trabajo i* \parencite{yu1995} es un marco de modelado utilizado en la ingeniería de requisitos y el desarrollo de software.
Ofrece una forma de modelar y analizar las relaciones intencionales de los interesados.
Las relaciones intencionales son relaciones que incluyen actores, objetivos, tareas, recursos y dependencias sociales dentro de un contexto organizacional.
Ayuda a comprender el porqué detrás de los requisitos del sistema e identificar posibles compensaciones entre los objetivos de los diferentes interesados.

El marco está orientado a actores, lo que significa que utiliza la noción de \textit{actor} como concepto primitivo.
Los actores pueden ser cualquier cosa, desde individuos hasta organizaciones, que tengan intereses estratégicos e intencionalidad.
La intencionalidad de un actor se captura a través de \textit{objetivos}.
Un objetivo representa una condición o estado de cosas que el actor desea alcanzar.
Los objetivos pueden ser objetivos duros u objetivos blandos.
Los objetivos duros son precisos y tienen criterios claros de satisfacción, mientras que los objetivos blandos son menos precisos y representan cualidades o consideraciones deseadas.

La Tabla~\ref{tab:istarDependencyTypes} presenta los tipos de dependencia junto con su ontología correspondiente.
\begin{table}
	\caption{Tipos de dependencia}\label{tab:istarDependencyTypes}
	\begin{tabularx}{\textwidth}{@{} llX @{}}
		\toprule
		\textbf{Ontología} & \textbf{Tipo} & \textbf{Descripción} \\
		\midrule
		Entidades & Recurso & Utilizado para representar el mundo como objetos. \\
		Actividades & Tarea & Producen un cambio en el mundo. \\
		Aserciones & Objetivos & Expresión de una condición o estado en el mundo. \\
		\bottomrule
	\end{tabularx}
\end{table}

\subsubsection{Modelos}
El marco de trabajo consta de dos modelos principales: el modelo de Dependencia Estratégica (DE) y el modelo de Racionalidad Estratégica (RE).

\begin{itemize}
    \item \textbf{Modelo de Dependencia Estratégica (DE):}
    Este modelo muestra la red de actores y sus dependencias.
    Ilustra cómo los actores dependen unos de otros para alcanzar objetivos, realizar tareas y obtener recursos.
    Este modelo es útil para identificar dependencias críticas y posibles vulnerabilidades en un sistema.
    \item \textbf{Modelo de Racionalidad Estratégica (RE):}
    Este modelo se centra en la racionalidad interna de los actores, mostrando cómo se alcanzan sus objetivos a través de tareas y cómo estas tareas utilizan recursos.
    Representa las razones detrás de las acciones de los actores y sus relaciones intencionales.
    Este modelo se utiliza para explorar formas alternativas de alcanzar objetivos y comprender las compensaciones involucradas.
\end{itemize}

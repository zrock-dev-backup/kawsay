\subsection{Algoritmos para la generacion de horarios}
La generación de horarios académicos es un problema complejo de optimización combinatoria, clasificado como NP-difícil, que busca asignar recursos (profesores, aulas, horarios) a eventos (clases) respetando un conjunto de restricciones duras (inviolables) y blandas (deseables) \parencite{Schaerf1999}.
El desarrollo de algoritmos eficientes para esta tarea es crucial para instituciones educativas como Jala University, ya que impacta directamente en la satisfacción de estudiantes y docentes, y en la utilización de recursos, siendo un pilar para mejorar el proceso actual de gestión del calendario.

\subsubsection{Constraint-based reasoning}
El razonamiento basado en restricciones (Constraint Satisfaction Problems - CSP) modela el problema de generación de horarios definiendo un conjunto de variables (clases), dominios (posibles asignaciones de tiempo/aula/profesor) y restricciones (reglas como no solapamiento, disponibilidad, capacidad) \parencite{Rossi2006}.
Algoritmos como backtracking, forward checking o arc consistency se utilizan para encontrar asignaciones que satisfagan todas las restricciones.
Este enfoque es útil para garantizar la viabilidad de los horarios generados, asegurando que se cumplan las reglas fundamentales del calendario académico de la universidad.

\subsubsection{Linear programming/Integer programming}
La programación lineal (LP) y la programación entera (IP) son técnicas de investigación de operaciones que permiten modelar problemas de optimización mediante funciones objetivo lineales y restricciones lineales (con variables continuas en LP y enteras o binarias en IP) \parencite{Winston2004}.
La generación de horarios puede formularse como un problema de IP, donde las variables representan decisiones de asignación (e.g., si una clase se asigna a un horario específico) y el objetivo es optimizar alguna métrica (minimizar huecos, maximizar preferencias) sujeto a las restricciones académicas.
Aunque computacionalmente intensivos, pueden garantizar soluciones óptimas para instancias de tamaño moderado.

\subsubsection{Heuristic algorithm}

In the context of faculty timetabling, a heuristic is a practical rule, strategy, or method used to find a feasible, though not necessarily optimal, schedule.
The problem of creating a university timetable is notoriously complex.
The number of possible schedules grows exponentially with the number of courses, faculty, rooms, and time slots.
Finding an optimal timetable (one that satisfies all constraints and maximizes some objective function like minimizing student conflicts or faculty preferences) is often computationally intractable, especially for large institutions.
Therefore, heuristic algorithms are crucial for tackling this challenge.

\paragraph{Conclusion}, timetable generation will be developed using Yule's \cite{Yule1969} heuristic algorithm. This algorithm has the following features:

\begin{enumerate}[label=\alph*.]
    \item \textbf{Constraint-Based Allocation}: Yule's algorithm emphasizes handling constraints effectively.
	These constraints can include:
    \begin{itemize}
        \item Faculty availability (days and times they are unavailable).
        \item Room availability (size, equipment, conflicts).
        \item Course prerequisites.
        \item Student course preferences (avoiding clashes).
        \item Institutional rules (e.g., limits on class sizes, specific times reserved for certain activities).
    \end{itemize}

    \item \textbf{Iterative Allocation:} The algorithm iteratively attempts to schedule classes one at a time, considering the current state of the timetable and the constraints associated with the class being scheduled.
	The order in which classes are scheduled is itself a heuristic; Yule mentions reordering classes that prove difficult to schedule.

    \item \textbf{Backtracking and Reordering:} If the algorithm reaches a point where it cannot schedule a class without violating constraints, it employs a form of backtracking.
	This may involve:
    \begin{itemize}
        \item Reordering the classes to be scheduled.
        \item Resetting the timetable to a previous state and trying a different allocation for an earlier class.
        \item Modifying "soft" constraints (preferences) to make a schedule feasible.
    \end{itemize}

    \item \textbf{Availability Matrix Representation:} Yule's algorithm uses availability matrices to efficiently represent the availability of faculty, rooms, and classes over time.
	These matrices allow the algorithm to quickly check for conflicts and determine feasible time slots.
	Formula 1 computes the line availability, Formulas 2 and 3 test for available times, and Formula 4 updates the matrices.

    \item \textbf{Line Availability Heuristic:} The concept of a "requirement line" (a single row in Yule's data structure representing a class, lecturer, and possibly a room) is key.
	The algorithm focuses on allocating entire requirement lines at once, which simplifies the problem and ensures that all components of a course (lecturer, class, room) are considered together.

    \item \textbf{Preference Handling:} Yule's algorithm incorporates faculty preferences for specific time slots or days off.
	These preferences are treated as "soft" constraints; the algorithm tries to satisfy them but will violate them if necessary to achieve a feasible timetable.
	Formula 4 handles preferences, marking times as "preferentially unavailable."

    \item \textbf{Dynamic Free Day Allocation:} The algorithm can dynamically adjust faculty free days based on scheduling difficulties.
	If a particular faculty member is causing a scheduling bottleneck, the algorithm may try changing their free day.
\end{enumerate}

Yule's algorithm makes several heuristic choices, including:

\begin{itemize}
    \item \textbf{Order of Class Scheduling:} The order in which classes are scheduled can significantly impact the outcome.
    \item \textbf{Free Day Allocation Strategy:} Deciding which faculty members to give free days to and when to adjust those days is a heuristic process.
    \item \textbf{Handling Preferences:} The balance between satisfying preferences and meeting hard constraints is a heuristic trade-off.
    \item \textbf{Backtracking Strategy:} Deciding when to backtrack, how far to backtrack, and which changes to make during backtracking are heuristic decisions.
\end{itemize}

Yule's algorithm has 4 basic formulas used for timetable generation.

Formula 1: \textbf{Line Availability Matrix} The first formula, shown in Equation \ref{eq:1}, calculates the elements of the line availability matrix \( _iE_{dp} \).
This matrix determines the overall availability of a particular lecture line \( i \) at a specific day \( d \) and period \( p \).

\begin{equation}
\label{eq:1}
_iE_{dp} = \left[\bigvee_{j \in S_i} (_jC_{dp})\right] \bigvee {}_zC'_{p}
\end{equation}

Where:
\begin{itemize}
    \item \( _iE_{dp} \) represents the availability of line \( i \) at day \( d \) and period \( p \).
    \item \( S_i \) is the set of classes, lecturers, and rooms involved in the \( i \)-th requirement.
    \item \( _jC_{dp} \) is an element of the availability matrix \( C \) for item \( j \) (class, lecturer, or room) indicating whether item \( j \) is unavailable (1) or available (0) in period \( p \) of day \( d \).
    \item The symbol \(\bigvee\) denotes the logical OR operation (union in the paper).
	The expression \(\bigvee_{j \in S_i} (_jC_{dp})\) calculates the union of availabilities for all items in set \( S_i \).
	If any item is unavailable, the result will be 1, meaning the time slot is generally unavailable.
    \item \( _zC'_{p} \) is an element of the vector \( C' \), which limits the times of day at which multi-period lectures may start. 
    \item The entire formula calculates whether a time slot is unavailable because either an element in the requirement \( S_i \) is unavailable or the period \( p \) is unsuitable for a lecture of length \( z_i \).
\end{itemize}

Formulas 2 and 3: \textbf{Testing for Available Place}, shown in Equations \ref{eq:2} and \ref{eq:3}, test whether a lecture can be scheduled in a specific time slot.
Formula 2 tests single period lectures, while formula 3 tests multi-period lectures.

\begin{equation}
\label{eq:2}
_iE_{dp} = 0 \quad \text{for a single period lecture}
\end{equation}

This formula simply states that for a single-period lecture to be scheduled at day \( d \) and period \( p \), the corresponding element in the line availability matrix \( _iE_{dp} \) must be 0, indicating the slot is available.

\begin{equation}
\label{eq:3}
_iE_{dp} \vee (_iE_{d,p+1} \wedge 1) \vee \dots \vee (_iE_{d,k} \wedge 1) = 0
\end{equation}

Where:
\begin{itemize}
    \item \( k = p + z_i - 1 \), where \( z_i \) is the length of the lecture (number of periods).
    \item The formula tests whether there is a continuous block of time slots available for a multi-period lecture of length \( z_i \) starting at period \( p \).
    \item  Each term \( _iE_{d,p+n} \) for \( n \) in the range \( 0 \) to \( z_i-1 \) represents the availability of the period \( p+n \).  The term \(  (_iE_{d,p+n} \wedge 1)  \) effectively masks any bits in  \( _iE_{d,p+n} \) except the least significant bit which represents availability in the most basic sense.
    \item If the logical OR of these terms equals 0, then all periods in the block are available, and the lecture can be scheduled.
\end{itemize}

Formula 4: \textbf{Updating Availability Matrices}, shown in Equation \ref{eq:4}, updates the availability matrices after a lecture has been scheduled.

\begin{equation}
\label{eq:4}
_jC_{d, p+k} (\text{updated}) = _jC_{d, p+k} (\text{old}) \vee 1
\end{equation}

Where:
\begin{itemize}
    \item \( _jC_{d, p+k} \) represents the availability matrix element for item \( j \) (class, lecturer, or room) at day \( d \) and period \( p+k \).
    \item \( k \) ranges from 0 to \( z_i - 1 \), covering all periods spanned by the scheduled lecture.
    \item The formula sets the availability element \( _jC_{d, p+k} \) to 1 (unavailable) by performing a logical OR with 1.
This ensures that the scheduled class, lecturer, or room is marked as unavailable for the duration of the lecture.
\end{itemize}

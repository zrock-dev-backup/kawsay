\subsection{Algoritmos para la generacion de horarios}
La generación de horarios académicos es un problema complejo de optimización combinatoria, que busca asignar recursos (profesores, aulas, horarios) a eventos (clases) respetando un conjunto de restricciones duras\footnote{Las \textbf{restricciones duras} son condiciones que deben ser satisfechas obligatoriamente en una solución válida.
Violar una restricción dura invalida la solución (por ejemplo, un profesor no puede estar en dos lugares al mismo tiempo).} (inviolables) y blandas\footnote{Las \textbf{restricciones blandas} son condiciones deseables pero no obligatorias.
Violar una restricción blanda no invalida la solución, pero disminuye su calidad o preferencia (por ejemplo, un profesor prefiere dar clases por la mañana).} (deseables) \parencite{Schaerf1999}.
El desarrollo de algoritmos eficientes para esta tarea es crucial para instituciones educativas como Jala University, ya que impacta directamente en la satisfacción de estudiantes y docentes, y en la utilización de recursos, siendo un pilar para mejorar el proceso actual de gestión del calendario.

\subsubsection{Razonamiento basado en restricciones}
El razonamiento basado en restricciones (Constraint Satisfaction Problems - CSP) modela el problema de generación de horarios definiendo un conjunto de variables (clases), dominios\footnote{En un CSP, el \textbf{dominio} de una variable es el conjunto de todos los valores posibles que esa variable puede tomar.} (posibles asignaciones de tiempo/aula/profesor) y restricciones (reglas como no solapamiento, disponibilidad, capacidad) \parencite{Rossi2006}.
Algoritmos como backtracking\footnote{El \textbf{backtracking} (vuelta atrás) es una técnica algorítmica general para encontrar todas (o algunas) soluciones a problemas computacionales}, forward checking o arc consistency\footnote{La \textbf{consistencia de arco (arc consistency)} Un arco entre dos variables es consistente si para cada valor en el dominio de la primera variable, existe algún valor en el dominio de la segunda variable tal que la restricción entre ellas se satisface.} se utilizan para encontrar asignaciones que satisfagan todas las restricciones.
Este enfoque es útil para garantizar la viabilidad de los horarios generados, asegurando que se cumplan las reglas fundamentales del calendario académico de la universidad.

\subsubsection{Programación lineal/Programación entera}
La programación lineal\footnote{La \textbf{programación lineal (PL)} es un método matemático para optimizar (maximizar o minimizar) una función objetivo lineal, sujeta a un conjunto de restricciones lineales (igualdades y desigualdades).} y la programación entera\footnote{La \textbf{programación entera (PE)} es un tipo de problema de optimización matemática en el que algunas o todas las variables están restringidas a ser números enteros.
Es una extensión de la programación lineal.} son técnicas de investigación de operaciones que permiten modelar problemas de optimización mediante funciones objetivo lineales y restricciones lineales (con variables continuas en PL y enteras o binarias en PE) \parencite{Winston2004}.
La generación de horarios puede formularse como un problema de PE, donde las variables representan decisiones de asignación (e.g., si una clase se asigna a un horario específico) y el objetivo es optimizar alguna métrica (minimizar huecos, maximizar preferencias) sujeto a las restricciones académicas.
Aunque computacionalmente intensivos, pueden garantizar soluciones óptimas para instancias de tamaño moderado.

\subsubsection{Algoritmo heurístico}

En el contexto de la generación de horarios académicos, una heurística es una regla práctica, estrategia o método utilizado para encontrar un horario factible, aunque no necesariamente óptimo.
El problema de crear un horario universitario es notoriamente complejo.
El número de horarios posibles crece exponencialmente con el número de cursos, profesores, aulas e intervalos de tiempo.
Encontrar un horario óptimo es a menudo computacionalmente intratable\footnote{\textbf{Computacionalmente intratable} se refiere a un problema que no puede ser resuelto por ningún algoritmo en un tiempo razonable (generalmente tiempo polinomial) a medida que el tamaño de la entrada del problema crece.}, especialmente para instituciones grandes.
Por lo tanto, los algoritmos heurísticos son cruciales para abordar este desafío.

\paragraph{Conclusión}
, la generación de horarios se desarrollará utilizando el algoritmo heurístico de Yule \cite{Yule1969}.
Este algoritmo tiene las siguientes características:

\begin{enumerate}[label=\alph*.]
    \item \textbf{Asignación Basada en Restricciones (Constraint-Based Allocation)}: El algoritmo de Yule enfatiza el manejo efectivo de las restricciones.
	Estas restricciones pueden incluir:
    \begin{itemize}
        \item Disponibilidad del profesorado (días y horas en que no están disponibles).
        \item Disponibilidad de aulas (tamaño, equipamiento, conflictos).
        \item Prerrequisitos de los cursos.
        \item Preferencias de cursos de los estudiantes (evitando conflictos).
        \item Reglas institucionales (p. ej., límites en el tamaño de las clases, horarios específicos reservados para ciertas actividades).
    \end{itemize}

    \item \textbf{Asignación Iterativa (Iterative Allocation):} El algoritmo intenta iterativamente programar las clases una por una, considerando el estado actual del horario y las restricciones asociadas con la clase que se está programando.
	El orden en que se programan las clases es en sí mismo una heurística; Yule menciona la reordenación de las clases que resultan difíciles de programar.

    \item \textbf{Retroceso y Reordenación (Backtracking and Reordering):} Si el algoritmo llega a un punto en el que no puede programar una clase sin violar las restricciones, emplea una forma de retroceso.
	Esto puede implicar:
    \begin{itemize}
        \item Reordenar las clases a programar.
        \item Restablecer el horario a un estado anterior e intentar una asignación diferente para una clase anterior.
        \item Modificar restricciones "blandas" (preferencias) para hacer factible un horario.
    \end{itemize}

    \item \textbf{Representación mediante Matriz de Disponibilidad (Availability Matrix Representation):} El algoritmo de Yule utiliza matrices de disponibilidad para representar eficientemente la disponibilidad del profesorado, las aulas y las clases a lo largo del tiempo.
	Estas matrices permiten al algoritmo verificar rápidamente los conflictos y determinar franjas horarias factibles.
	La Fórmula 1 calcula la disponibilidad de línea, las Fórmulas 2 y 3 prueban los tiempos disponibles, y la Fórmula 4 actualiza las matrices.

    \item \textbf{Heurística de Disponibilidad de Línea (Line Availability Heuristic):} El concepto de una "línea de requerimiento"\footnote{Una \textbf{"línea de requerimiento"} en el contexto del algoritmo de Yule es una estructura de datos que representa una unidad de programación, como una clase específica con su profesor y aula asignados (o por asignar).} es clave.
    El algoritmo se enfoca en asignar líneas de requerimiento completas de una vez, lo que simplifica el problema y asegura que todos los componentes de un curso (profesor, clase, aula) se consideren juntos.

    \item \textbf{Manejo de Preferencias (Preference Handling):} El algoritmo de Yule incorpora las preferencias del profesorado por franjas horarias o días libres específicos.
	Estas preferencias se tratan como restricciones "blandas"; el algoritmo intenta satisfacerlas, pero las violará si es necesario para lograr un horario factible.
	La Fórmula 4 maneja las preferencias, marcando los tiempos como "preferentemente no disponibles".

    \item \textbf{Asignación Dinámica de Días Libres (Dynamic Free Day Allocation):} El algoritmo puede ajustar dinámicamente los días libres del profesorado basándose en las dificultades de programación.
	Si un profesor en particular está causando un cuello de botella en la programación, el algoritmo puede intentar cambiar su día libre.
\end{enumerate}

El algoritmo de Yule toma varias decisiones heurísticas, incluyendo:

\begin{itemize}
    \item \textbf{Orden de Programación de Clases (Order of Class Scheduling):} El orden en que se programan las clases puede impactar significativamente el resultado.
    \item \textbf{Estrategia de Asignación de Días Libres (Free Day Allocation Strategy):} Decidir a qué miembros del profesorado dar días libres y cuándo ajustar esos días es un proceso heurístico.
    \item \textbf{Manejo de Preferencias (Handling Preferences):} El equilibrio entre satisfacer las preferencias y cumplir con las restricciones duras es una compensación heurística.
    \item \textbf{Estrategia de Retroceso (Backtracking Strategy):} Decidir cuándo retroceder, cuánto retroceder y qué cambios realizar durante el retroceso son decisiones heurísticas.
\end{itemize}

El algoritmo de Yule busca poblar una matriz de disponibilidad por línea de requerimiento, cada celda se ve como en la Figura~\ref{fig:yuleAlgorithm} donde la superposición de cada entidad definirá si la línea de requerimiento está disponible en la celda.
\begin{figure}[H]
    \centering
    \caption{Visualización de una celda de disponibilidad de línea de requerimiento}
    \includegraphics[width=.45\textwidth]{yule-algo.pdf}
    \label{fig:yuleAlgorithm}
\end{figure}

El algoritmo de Yule tiene 4 fórmulas básicas utilizadas para la generación de horarios.

Fórmula 1: \textbf{Matriz de Disponibilidad de Línea (Line Availability Matrix)} La primera fórmula, mostrada en la Ecuación \ref{eq:1}, calcula los elementos de la matriz de disponibilidad de línea \( _iE_{dp} \).
Esta matriz determina la disponibilidad general de una línea de clase particular \( i \) en un día específico \( d \) y período \( p \).

\begin{equation}
\label{eq:1}
_iE_{dp} = \left[\bigvee_{j \in S_i} (_jC_{dp})\right] \bigvee {}_zC'_{p}
\end{equation}

Donde:
\begin{itemize}
    \item \( _iE_{dp} \) representa la disponibilidad de la línea \( i \) en el día \( d \) y período \( p \).
    \item \( S_i \) es el conjunto de clases, profesores y aulas involucrados en el \( i \)-ésimo requerimiento.
    \item \( _jC_{dp} \) es un elemento de la matriz de disponibilidad \( C \) para el elemento \( j \) (clase, profesor o aula) que indica si el elemento \( j \) no está disponible (1) o está disponible (0) en el período \( p \) del día \( d \).
    \item El símbolo \(\bigvee\) denota la operación OR lógica (unión en el artículo original).
    La expresión \(\bigvee_{j \in S_i} (_jC_{dp})\) calcula la unión de las disponibilidades para todos los elementos en el conjunto \( S_i \).
    Si algún elemento no está disponible, el resultado será 1, lo que significa que la franja horaria generalmente no está disponible.
    \item \( _zC'_{p} \) es un elemento del vector \( C' \), que limita las horas del día en las que pueden comenzar las clases de múltiples períodos.
    \item La fórmula completa calcula si una franja horaria no está disponible porque un elemento en el requerimiento \( S_i \) no está disponible o el período \( p \) no es adecuado para una clase de duración \( z_i \).
\end{itemize}

Fórmulas 2 y 3: \textbf{Prueba de Lugar Disponible (Testing for Available Place)}, mostradas en las Ecuaciones \ref{eq:2} y \ref{eq:3}, prueban si una clase puede ser programada en una franja horaria específica.
La Fórmula 2 prueba clases de un solo período, mientras que la Fórmula 3 prueba clases de múltiples períodos.

\begin{equation}
\label{eq:2}
_iE_{dp} = 0 \quad \text{para una clase de un solo período}
\end{equation}

Esta fórmula simplemente establece que para que una clase de un solo período sea programada en el día \( d \) y período \( p \), el elemento correspondiente en la matriz de disponibilidad de línea \( _iE_{dp} \) debe ser 0, indicando que la franja está disponible.

\begin{equation}
\label{eq:3}
_iE_{dp} \vee (_iE_{d,p+1} \wedge 1) \vee \dots \vee (_iE_{d,k} \wedge 1) = 0
\end{equation}

Donde:
\begin{itemize}
    \item \( k = p + z_i - 1 \), donde \( z_i \) es la duración de la clase (número de períodos).
    \item La fórmula prueba si hay un bloque continuo de franjas horarias disponibles para una clase de múltiples períodos de duración \( z_i \) que comienza en el período \( p \).
    \item Cada término \( _iE_{d,p+n} \) para \( n \) en el rango de \( 0 \) a \( z_i-1 \) representa la disponibilidad del período \( p+n \).
    El término \( (_iE_{d,p+n} \wedge 1) \) enmascara eficazmente cualquier bit en \( _iE_{d,p+n} \) excepto el bit menos significativo, que representa la disponibilidad en el sentido más básico.
    \item Si el OR lógico de estos términos es igual a 0, entonces todos los períodos en el bloque están disponibles y la clase puede ser programada.
\end{itemize}

Fórmula 4: \textbf{Actualización de Matrices de Disponibilidad (Updating Availability Matrices)}, mostrada en la Ecuación \ref{eq:4}, actualiza las matrices de disponibilidad después de que una clase ha sido programada.

\begin{equation}
\label{eq:4}
_jC_{d, p+k} (\text{actualizado}) = _jC_{d, p+k} (\text{antiguo}) \vee 1
\end{equation}

Donde:
\begin{itemize}
    \item \( _jC_{d, p+k} \) representa el elemento de la matriz de disponibilidad para el ítem \( j \) (clase, profesor o aula) en el día \( d \) y período \( p+k \).
    \item \( k \) varía de 0 a \( z_i - 1 \), cubriendo todos los períodos abarcados por la clase programada.
    \item La fórmula establece el elemento de disponibilidad \( _jC_{d, p+k} \) en 1 (no disponible) realizando un OR lógico con 1.
    Esto asegura que la clase, profesor o aula programados se marquen como no disponibles durante la duración de la clase.
\end{itemize}

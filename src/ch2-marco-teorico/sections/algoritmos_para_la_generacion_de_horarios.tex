\subsection{Algoritmos para la generacion de horarios}
La generación de horarios académicos es un problema complejo de optimización combinatoria, clasificado como NP-difícil, que busca asignar recursos (profesores, aulas, horarios) a eventos (clases) respetando un conjunto de restricciones duras (inviolables) y blandas (deseables) \parencite{Schaerf1999}.
El desarrollo de algoritmos eficientes para esta tarea es crucial para instituciones educativas como Jala University, ya que impacta directamente en la satisfacción de estudiantes y docentes, y en la utilización de recursos, siendo un pilar para mejorar el proceso actual de gestión del calendario.

\subsubsection{Evolutionary and genetic algorithms}
Los algoritmos evolutivos, y en particular los algoritmos genéticos, son metaheurísticas inspiradas en la evolución biológica que resultan efectivas para problemas de optimización complejos como la generación de horarios \parencite{Eiben2003}.
Operan sobre una población de soluciones candidatas (horarios), aplicando operadores genéticos como selección, cruce y mutación para evolucionar hacia soluciones de mayor calidad (que satisfacen más restricciones o minimizan conflictos).
Su capacidad para explorar amplios espacios de búsqueda los hace adecuados para encontrar horarios viables y optimizados en entornos con múltiples restricciones como el de Jala University.

\subsubsection{Constraint-based reasoning}
El razonamiento basado en restricciones (Constraint Satisfaction Problems - CSP) modela el problema de generación de horarios definiendo un conjunto de variables (clases), dominios (posibles asignaciones de tiempo/aula/profesor) y restricciones (reglas como no solapamiento, disponibilidad, capacidad) \parencite{Rossi2006}.
Algoritmos como backtracking, forward checking o arc consistency se utilizan para encontrar asignaciones que satisfagan todas las restricciones.
Este enfoque es útil para garantizar la viabilidad de los horarios generados, asegurando que se cumplan las reglas fundamentales del calendario académico de la universidad.

\subsubsection{Linear programming/Integer programming}
La programación lineal (LP) y la programación entera (IP) son técnicas de investigación de operaciones que permiten modelar problemas de optimización mediante funciones objetivo lineales y restricciones lineales (con variables continuas en LP y enteras o binarias en IP) \parencite{Winston2004}.
La generación de horarios puede formularse como un problema de IP, donde las variables representan decisiones de asignación (e.g., si una clase se asigna a un horario específico) y el objetivo es optimizar alguna métrica (minimizar huecos, maximizar preferencias) sujeto a las restricciones académicas.
Aunque computacionalmente intensivos, pueden garantizar soluciones óptimas para instancias de tamaño moderado.

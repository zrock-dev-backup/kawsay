\subsection{Herramientas de versionamiento}
El control de versiones es indispensable en el desarrollo de software para gestionar los cambios en el código fuente a lo largo del tiempo, facilitar la colaboración entre desarrolladores y permitir la reversión a estados anteriores.

\subsubsection{GitHub}
GitHub es una plataforma de desarrollo colaborativo basada en \texttt{Git} que ofrece hospedaje de repositorios, seguimiento de issues, revisión de código (Pull Requests), integración continua y otras herramientas para el ciclo de vida del desarrollo de software \parencite{GitHub}.
Es la plataforma de facto para muchos proyectos de código abierto y empresariales, y sería una opción robusta para alojar el código del sistema de Jala University, gestionar la colaboración y automatizar flujos de trabajo.

\subsubsection{Sourcehut}
SourceHut es una suite de herramientas de desarrollo de software de código abierto, enfocada en la simplicidad, la estabilidad y la filosofía Unix, que ofrece hospedaje \texttt{Git}, seguimiento de tickets, listas de correo, CI/CD y otras funcionalidades \parencite{SourceHut}.
Representa una alternativa más minimalista y centrada en el desarrollador a plataformas como GitHub o GitLab, atractiva para quienes valoran la transparencia y el control sobre sus herramientas de desarrollo.

\subsubsection{Nomenclatura de ramas, commits y pull requests}
Establecer una convención clara para nombrar ramas (e.g., \texttt{feature/nombre-funcionalidad}, \texttt{bugfix/descripcion-corta}, \texttt{release/v1.0}), escribir mensajes de commit significativos (e.g., siguiendo el formato Conventional Commits \parencite{ConventionalCommits}) y gestionar Pull Requests (PRs) de manera estructurada (con descripciones claras, revisiones obligatorias) es crucial para mantener un historial de cambios limpio, facilitar la revisión de código y mejorar la colaboración dentro del equipo de desarrollo del proyecto.

\subsection{Arquitectura de Frontend}
La arquitectura frontend se ocupa de la estructura y organización del código que se ejecuta en el navegador del usuario, gestionando la interfaz de usuario (UI) y la interacción con el backend.
Una buena arquitectura frontend es esencial para proporcionar una experiencia de usuario fluida, receptiva y mantenible para estudiantes y administradores que utilicen el sistema de gestión del calendario académico \parencite{Osmani2017}.

\subsubsection{MVM}
El patrón Model-View-ViewModel (MVVM) es un patrón de diseño arquitectónico para interfaces de usuario que facilita la separación entre la lógica de presentación (ViewModel), la interfaz de usuario (View) y los datos (Model) \parencite{Smith2005}.
El ViewModel actúa como intermediario, exponiendo datos y comandos que la View puede enlazar (data binding), lo que simplifica el manejo del estado de la UI y mejora la testeabilidad.
Utilizar MVVM en el frontend (posiblemente con frameworks como \texttt{React} o \texttt{Vue} adaptándolo) puede ayudar a gestionar la complejidad de las interfaces para visualizar horarios, configurar parámetros de predicción y mostrar resultados.

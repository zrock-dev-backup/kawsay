\subsection{Backend architecture}
La arquitectura backend define la estructura interna del servidor, la lógica de negocio y la gestión de datos del sistema web, siendo fundamental para su escalabilidad, mantenibilidad y rendimiento.
Una arquitectura bien diseñada facilita la evolución del sistema y la integración de nuevas funcionalidades, como el módulo de predicción de horarios con ML, asegurando que la gestión del calendario académico sea robusta y eficiente \parencite{Richards2015}.

\subsubsection{Microservices}
La arquitectura de microservicios estructura una aplicación como una colección de servicios pequeños, autónomos y débilmente acoplados, cada uno enfocado en una capacidad de negocio específica y comunicándose a través de APIs (\textit{Application Programming Interfaces}) ligeras, usualmente sobre \texttt{HTTP} \parencite{Newman2015}.
Adoptar microservicios para el sistema de gestión académica permitiría desarrollar, desplegar y escalar independientemente componentes como la gestión de cursos, la generación de horarios, la predicción de elegibilidad y la gestión de usuarios, aumentando la resiliencia y flexibilidad del sistema global en Jala University.

\subsubsection{Clean Architecture}
La Arquitectura Limpia, propuesta por Robert C.
Martin, es un conjunto de principios de diseño de software que promueve la separación de intereses y la independencia de frameworks, UI y bases de datos, organizando el código en capas concéntricas (Entidades, Casos de Uso, Adaptadores de Interfaz, Frameworks y Drivers) \parencite{Martin2017}.
Aplicar Clean Architecture en el backend asegura que la lógica de negocio central (reglas académicas, algoritmos de predicción y generación de horarios) esté aislada de detalles externos, facilitando las pruebas, la mantenibilidad y la adaptabilidad a cambios tecnológicos futuros.

\subsubsection{Domain Driven Design}
El Diseño Guiado por el Dominio (DDD) es un enfoque para el desarrollo de software complejo que se centra en modelar el dominio del negocio (en este caso, la gestión del calendario académico y la predicción de elegibilidad en Jala University) y plasmar ese modelo en el código, utilizando un lenguaje ubicuo compartido entre expertos del dominio y desarrolladores \parencite{Evans2003}.
DDD ayuda a gestionar la complejidad mediante conceptos como Entidades, Objetos Valor, Agregados, Repositorios y Servicios de Dominio, asegurando que el software refleje fielmente las reglas y procesos académicos.

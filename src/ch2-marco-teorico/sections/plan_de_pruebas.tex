\subsection{Plan de pruebas}
Un plan de pruebas sistemático es esencial para asegurar la calidad, fiabilidad y corrección del sistema web desarrollado, verificando que cumple con los requerimientos funcionales\footnote{Los \textbf{requerimientos funcionales} definen comportamientos específicos del sistema.} y no funcionales\footnote{Los \textbf{requerimientos no funcionales} describen cómo un sistema de software debe ser, es decir, las cualidades o restricciones del sistema, como rendimiento, seguridad, usabilidad o mantenibilidad.} definidos.

\subsubsection{ISO 9126}
La norma ISO/IEC 9126 (reemplazada en parte por ISO/IEC 25010) define un modelo de calidad para el software, clasificando los atributos de calidad en seis características principales: Funcionalidad, Fiabilidad, Usabilidad, Eficiencia, Mantenibilidad y Portabilidad \parencite{ISO9126}.
Utilizar este modelo como marco para el plan de pruebas del sistema de gestión académica permite definir criterios de aceptación claros y métricas específicas para evaluar cada aspecto de la calidad del software, asegurando una cobertura completa y sistemática de las pruebas.

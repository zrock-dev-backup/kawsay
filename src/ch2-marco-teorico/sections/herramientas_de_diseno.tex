\subsection{Herramientas de diseño}
Las herramientas de diseño facilitan la creación de prototipos\footnote{Un \textbf{prototipo} en diseño es un modelo preliminar o maqueta de un producto (como una interfaz de usuario) que se utiliza para probar y evaluar conceptos, funcionalidades e interacciones antes del desarrollo completo. (Nota: Ya definido en `objetivos.tex` como "prototipos de interfaz de usuario", considerar glosario).}, wireframes y diseños visuales de la interfaz de usuario (UI) y la experiencia de usuario (UX)\footnote{La \textbf{Experiencia de Usuario (UX - User Experience)} se refiere a las percepciones y respuestas de una persona como resultado del uso o la anticipación del uso de un producto, sistema o servicio.
Abarca todos los aspectos de la interacción del usuario final con la empresa, sus servicios y sus productos.}, permitiendo iterar y validar ideas antes de escribir código.

\subsubsection{Figma}
Figma es una herramienta de diseño de interfaces basada en la web y colaborativa, que permite crear prototipos interactivos y colaborar en tiempo real entre diseñadores y desarrolladores \parencite{Figma}.
Es ideal para diseñar las pantallas del sistema de gestión académica, definir flujos de usuario\footnote{Un \textbf{flujo de usuario (user flow)} es la secuencia de pasos que un usuario sigue a través de una aplicación o sitio web para completar una tarea específica.
Se utiliza para analizar y optimizar la experiencia del usuario.} y crear un lenguaje visual consistente para la aplicación de Jala University.

\paragraph{Conclusión}
Esta herramienta se utilizará para diseñar prototipos iniciales del sistema con el fin de comprender los requisitos desde el punto de vista de los casos de uso.

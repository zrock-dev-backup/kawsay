\subsection{Metodologías de trabajo}
Las metodologías de trabajo ágiles proporcionan marcos para organizar y gestionar el proceso de desarrollo de software de manera flexible, colaborativa e iterativa, permitiendo adaptarse a cambios y entregar valor de forma incremental.
La elección de una metodología ágil es fundamental para la gestión eficiente de un proyecto de grado como este \parencite{Beck2001}.

\subsubsection{Scrum}
Scrum es un marco de trabajo ágil ampliamente utilizado que organiza el desarrollo en ciclos cortos llamados Sprints (usualmente de 2 a 4 semanas), con roles definidos (Product Owner, Scrum Master, Equipo de Desarrollo), artefactos (Product Backlog, Sprint Backlog, Incremento) y eventos (Sprint Planning, Daily Scrum, Sprint Review, Sprint Retrospective) \parencite{SchwaberSutherland2020}.
Adoptar Scrum para este proyecto permitiría gestionar el desarrollo del sistema web de forma iterativa, priorizando funcionalidades, fomentando la colaboración y permitiendo la inspección y adaptación continua basada en el feedback y los avances.

\subsubsection{eXtreme Programming (XP)}
eXtreme Programming (XP) es una metodología ágil centrada en la entrega continua de software de alta calidad y en la adaptación a los requisitos cambiantes.
Como se puede ver en la Figura~\ref{fig:xpWorkflowA}, se basa en un conjunto de valores (comunicación, simplicidad, feedback, coraje y respeto) y prácticas técnicas robustas como la programación en parejas (pair programming), el desarrollo guiado por pruebas (Test-Driven Development - TDD), la integración continua (Continuous Integration - CI), la refactorización y las pequeñas entregas \parencite{Beck2004}.

XP podría ser particularmente adecuada para este proyecto de grado si se busca un enfoque fuerte en la calidad técnica del código, una colaboración muy estrecha entre los miembros del equipo y una capacidad alta de respuesta a los cambios en los requisitos o el diseño a medida que el proyecto evoluciona.

\begin{figure}
    \centering
    \caption{Representación del ciclo de vida de XP}\label{fig:xpWorkflowA}
    \includegraphics[width=.75\textwidth]{xp-workflow.pdf}

    \vspace{0.5em}
    \begin{minipage}{\textwidth}
        \small\textit{Nota.} Fuente: \textcite{abrahamsson2017agile}.
    \end{minipage}
\end{figure}

\paragraph{Conclusión:}
Con base en la comparación de Scrum y XP y el análisis de viabilidad en el Apéndice \ref{sec:methodology-justification}, se ha elegido eXtreme Programming para el desarrollo del sistema.

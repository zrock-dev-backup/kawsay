\section{MARCO TEÓRICO}
\label{sec:marcoTeorico}
% \section{Marco teorico}
% scheduling engine takes chaos and delivers order by obeying constraints
%
% > If a standard set of patterns is chosen, with compatible starting and ending times, schedules will fit together more easily. If patterns are dissimilar, more conflicts will occur within a given academic week.
%
% How can the effectiveness of the system be measured?
%
% scheduling conflicts, time patterns , courses and periods.

\subsection{Machine Learning}

As \textcite{mahesh2020machine} says, the capacity of a computer system to learn how to perform an specific task without being explicitly programmed means the machine has learnt. Once a computer program has learnt what to do with data it can do its work automatically.

\paragraph {Decision Tree}

\textcite{navada2011overview} defines a decision tree as tree structure where internal nodes represent tests (on input data patterns), and leaf nodes represent categories of those patterns. These tests are passed down through the tree to yield the correct output for the input pattern. Decision tree algorithms can be applied in various fields. They can replace statistical procedures to find data, extract text, identify missing data in a class, improve search engines, and are even used in medical applications. Many decision tree algorithms have been developed, each varying in accuracy and cost-effectiveness. It is also important to choose the right algorithm for the task at hand.

\begin{figure}[H]
    \centering
    \includegraphics[width=0.8\textwidth]{"../resources/images/decision-tree.png"}
    \caption{Decision tree - \textcite{mahesh2020machine}}
    \label{fig:image}
\end{figure}

Decision trees are classified as:
\begin{itemize}
    \item Classification Tree: Applicated in probability and statistics.
    \item Regression Tree: Used in calculations for real estate. Determines one single predetermined outcome from different pieces of information
    \item Decision Tree Forests: Varied decision trees are created and then grouped together in order to accurately determine as to what will happen with a particular outcome.
\end{itemize}


\subsection{Case study}
Timetable generation and student enrollment processes have been identified with the analysis of Appendix~\ref{sec:appendixIStarAnalysis}

\subsubsection{Processes}
\paragraph{Timetable generation process} Consist on the production of the timetable for a modulethat satisfies the academic content of each module in defined in the Jala University's Student Catalog.

In this process a clash is produced when a teacher requests a change of schedule because of an external reason, in this situation the registrar has to find another available timeslot taking into account the class availability and the teacher.
Sometimes when this operation is not possible then the teacher is replaced.

\paragraph{Student enrollment process} The enrollment process is performed at the start of each module, students are divided in cohorts which are distributed in groups and sections.
In this process a enrollment clash is produced when a student fails a course and needs to repeated, when enrolling this student to the class sometimes she will also have another class happening in the same period.

\subsection{Machine Learning}
El Machine Learning (ML), es una rama de la inteligencia artificial que se enfoca en el desarrollo de sistemas capaces de aprender y mejorar a partir de la experiencia, sin ser explícitamente programados para cada tarea específica \parencite{Samuel1959}.
En el contexto de este proyecto, el ML se aplica para analizar datos históricos académicos y predecir la elegibilidad de los estudiantes para inscribirse en futuras asignaturas, buscando optimizar la asignación de cupos y la planificación académica, abordando así uno de los objetivos principales de entender el impacto del ML en este proceso.

\paragraph{Unsupervised learning} trabaja con datos no etiquetados, buscando descubrir patrones, estructuras o relaciones inherentes en la información sin una guía previa sobre las salidas correctas \parencite{Hastie2009}.
Técnicas comunes incluyen el clustering (agrupación) y la reducción de dimensionalidad.
Aunque el enfoque principal del proyecto es supervisado para la predicción de elegibilidad, el análisis exploratorio de datos podría emplear técnicas no supervisadas para identificar grupos de estudiantes con perfiles académicos similares o detectar anomalías en los patrones de inscripción, complementando la comprensión del proceso actual.

\paragraph{Supervised learning} es un paradigma del ML donde el algoritmo aprende a partir de un conjunto de datos previamente etiquetado, es decir, cada ejemplo de entrada está asociado a una salida correcta conocida \parencite{Bishop2006}.
El objetivo es entrenar un modelo que pueda predecir la etiqueta de salida para nuevas entradas no vistas.
Para la predicción de elegibilidad de estudiantes en Jala University, se utilizarán datos históricos (calificaciones, cursos aprobados, plan de estudios) como entradas etiquetadas (elegible/no elegible para un curso específico) para entrenar un modelo predictivo, como podría ser una regresión logística o una máquina de soporte vectorial.

Figure~\ref{fig:mlComparison} represents how an unsupervised model uses unlabeled data for classification and supervised learning uses labeled data.

\begin{figure}
	\caption{Schematic representation of learning paradigms, Fuente: \parencite{morimoto2021}} \label{fig:mlComparison}
	\begin{center}
		\includegraphics[width=0.45\textwidth]{comparison-supervised-unsupervised.pdf}
	\end{center}
\end{figure}


Decision Trees and Random Forest are two models that fit the requirements of the project.

\subsubsection{Decision Trees}
A decision tree is a supervised learning algorithm used for both classification and regression tasks.
It operates by recursively partitioning the feature space into distinct regions based on a series of decision rules inferred from the data \parencite{quinlan1986induction}.
Each internal node in the tree represents a test on an attribute (feature), each branch represents the outcome of the test, and each leaf node represents a class label (for classification) or a prediction (for regression).

The decision tree learning process involves selecting the most informative features to split the data at each node, aiming to create increasingly homogeneous subsets.
This process continues recursively until a stopping criterion is met, such as reaching a maximum tree depth, a minimum number of samples in a node, or achieving perfect classification (or prediction) within a node.

\paragraph{Splitting Criteria}
The selection of the best feature to split on is crucial in decision tree construction.
Common splitting criteria include:

\begin{itemize}
    \item \textbf{Gini Impurity (for classification):} Measures the impurity of a set of instances, with lower values indicating greater homogeneity.
    \item \textbf{Information Gain (for classification):} Measures the reduction in entropy after splitting on an attribute.
    \item \textbf{Mean Squared Error (MSE) (for regression):} Measures the average squared difference between predicted and actual values.
\end{itemize}

\subsubsection{Random Forest}
The Random Forest algorithm, introduced by Breiman \cite{breiman2001random}, is a supervised learning algorithm.
Is an ensemble of decision trees, where each tree is trained on a random subset of the data and a random subset of the features.
The final prediction is obtained by aggregating the predictions of all the individual trees.
For classification tasks, this is typically done by majority voting, while for regression tasks, the average prediction is used.

The Random Forest algorithm leverages the "wisdom of the crowd" principle, combining the diverse perspectives of multiple decision trees to yield more accurate and stable predictions than individual trees can provide.
The core steps involved in constructing a Random Forest are:

\begin{enumerate}
    \item \textbf{Bootstrap Sampling:} The algorithm randomly selects, with replacement, \textit{n} samples from the original training dataset to create a unique training set for each tree.
    This process, termed bootstrapping, ensures that each tree is trained on a slightly different subset of the data.
    \item \textbf{Feature Randomness:} For each node within a tree, a random subset of \textit{m} features is chosen from the total \textit{p} features available.
    The algorithm then identifies the optimal split for the node based on these \textit{m} features.
    The parameter \textit{m} is typically significantly smaller than \textit{p}, a crucial step for reducing correlation among the trees.
    \item \textbf{Tree Growth:} Each tree is grown to its maximum depth without pruning (or until a predefined minimum node size is reached).
    This allows each tree to capture complex relationships in its training data.
    \item \textbf{Prediction Aggregation:} For classification problems, the final class label is determined by a majority vote across all trees.
    In regression scenarios, the final prediction is the average of the predictions generated by the individual trees.
\end{enumerate}

\paragraph{Hyperparameter Tuning} is a critical phase in optimizing the performance of a Random Forest model.
Key hyperparameters that significantly impact the model's behavior include:

\begin{itemize}
    \item \texttt{n\_estimators}: The number of trees within the forest.
    \item \texttt{max\_features}: The number of features considered when searching for the best split at each node.
    \item \texttt{max\_depth}: The maximum permissible depth of the individual trees.
    \item \texttt{min\_samples\_split}: The minimum number of samples required to split an internal node.
    \item \texttt{min\_samples\_leaf}: The minimum number of samples mandated to reside in a leaf node.
\end{itemize}

Two widely used methodologies for hyperparameter tuning are Grid Search and Random Search.

\begin{itemize}
	\item \textbf{Grid search:} is an exhaustive search technique that systematically evaluates all possible combinations of hyperparameter values within a predefined grid \cite{bergstra2012random}.
	The user specifies a discrete set of values for each hyperparameter, and Grid Search meticulously explores every combination.

	\item \textbf{Random search:} is a stochastic search method that randomly samples hyperparameter combinations from specified distributions \cite{bergstra2012random}.
	In contrast to Grid Search's exhaustive exploration, Random Search probes a random subset of the hyperparameter space.
\end{itemize}

\paragraph{Conclusion:} For this project we're gonna use random forest.

\subsection{Algoritmos para la generacion de horarios}
La generación de horarios académicos es un problema complejo de optimización combinatoria, clasificado como NP-difícil\footnote{\textbf{NP-difícil} (Non-deterministic Polynomial-time hard) es una clase de problemas de decisión en teoría de la complejidad computacional que son "al menos tan difíciles como los problemas más difíciles en NP".}, que busca asignar recursos (profesores, aulas, horarios) a eventos (clases) respetando un conjunto de restricciones duras\footnote{Las \textbf{restricciones duras} son condiciones que deben ser satisfechas obligatoriamente en una solución válida.
Violar una restricción dura invalida la solución (por ejemplo, un profesor no puede estar en dos lugares al mismo tiempo).} (inviolables) y blandas\footnote{Las \textbf{restricciones blandas} son condiciones deseables pero no obligatorias.
Violar una restricción blanda no invalida la solución, pero disminuye su calidad o preferencia (por ejemplo, un profesor prefiere dar clases por la mañana).} (deseables) \parencite{Schaerf1999}.
El desarrollo de algoritmos eficientes para esta tarea es crucial para instituciones educativas como Jala University, ya que impacta directamente en la satisfacción de estudiantes y docentes, y en la utilización de recursos, siendo un pilar para mejorar el proceso actual de gestión del calendario.

\subsubsection{Razonamiento basado en restricciones}
El razonamiento basado en restricciones (Constraint Satisfaction Problems - CSP) modela el problema de generación de horarios definiendo un conjunto de variables (clases), dominios\footnote{En un CSP, el \textbf{dominio} de una variable es el conjunto de todos los valores posibles que esa variable puede tomar.} (posibles asignaciones de tiempo/aula/profesor) y restricciones (reglas como no solapamiento, disponibilidad, capacidad) \parencite{Rossi2006}.
Algoritmos como backtracking\footnote{El \textbf{backtracking} (vuelta atrás) es una técnica algorítmica general para encontrar todas (o algunas) soluciones a problemas computacionales}, forward checking o arc consistency\footnote{La \textbf{consistencia de arco (arc consistency)} Un arco entre dos variables es consistente si para cada valor en el dominio de la primera variable, existe algún valor en el dominio de la segunda variable tal que la restricción entre ellas se satisface.} se utilizan para encontrar asignaciones que satisfagan todas las restricciones.
Este enfoque es útil para garantizar la viabilidad de los horarios generados, asegurando que se cumplan las reglas fundamentales del calendario académico de la universidad.

\subsubsection{Programación lineal/Programación entera}
La programación lineal\footnote{La \textbf{programación lineal (PL)} es un método matemático para optimizar (maximizar o minimizar) una función objetivo lineal, sujeta a un conjunto de restricciones lineales (igualdades y desigualdades).} y la programación entera\footnote{La \textbf{programación entera (PE)} es un tipo de problema de optimización matemática en el que algunas o todas las variables están restringidas a ser números enteros.
Es una extensión de la programación lineal.} son técnicas de investigación de operaciones que permiten modelar problemas de optimización mediante funciones objetivo lineales y restricciones lineales (con variables continuas en PL y enteras o binarias en PE) \parencite{Winston2004}.
La generación de horarios puede formularse como un problema de PE, donde las variables representan decisiones de asignación (e.g., si una clase se asigna a un horario específico) y el objetivo es optimizar alguna métrica (minimizar huecos, maximizar preferencias) sujeto a las restricciones académicas.
Aunque computacionalmente intensivos, pueden garantizar soluciones óptimas para instancias de tamaño moderado.

\subsubsection{Algoritmo heurístico}

En el contexto de la generación de horarios académicos, una heurística es una regla práctica, estrategia o método utilizado para encontrar un horario factible, aunque no necesariamente óptimo.
El problema de crear un horario universitario es notoriamente complejo.
El número de horarios posibles crece exponencialmente\footnote{Un crecimiento \textbf{exponencial} significa que la cantidad aumenta multiplicándose por un factor constante en cada intervalo de tiempo o paso, llevando a un aumento muy rápido.} con el número de cursos, profesores, aulas e intervalos de tiempo.
Encontrar un horario óptimo es a menudo computacionalmente intratable\footnote{\textbf{Computacionalmente intratable} se refiere a un problema que no puede ser resuelto por ningún algoritmo en un tiempo razonable (generalmente tiempo polinomial) a medida que el tamaño de la entrada del problema crece.}, especialmente para instituciones grandes.
Por lo tanto, los algoritmos heurísticos son cruciales para abordar este desafío.

\paragraph{Conclusión}
, la generación de horarios se desarrollará utilizando el algoritmo heurístico de Yule \cite{Yule1969}.
Este algoritmo tiene las siguientes características:

\begin{enumerate}[label=\alph*.]
    \item \textbf{Asignación Basada en Restricciones (Constraint-Based Allocation)}: El algoritmo de Yule enfatiza el manejo efectivo de las restricciones.
	Estas restricciones pueden incluir:
    \begin{itemize}
        \item Disponibilidad del profesorado (días y horas en que no están disponibles).
        \item Disponibilidad de aulas (tamaño, equipamiento, conflictos).
        \item Prerrequisitos de los cursos.
        \item Preferencias de cursos de los estudiantes (evitando conflictos).
        \item Reglas institucionales (p. ej., límites en el tamaño de las clases, horarios específicos reservados para ciertas actividades).
    \end{itemize}

    \item \textbf{Asignación Iterativa (Iterative Allocation):} El algoritmo intenta iterativamente programar las clases una por una, considerando el estado actual del horario y las restricciones asociadas con la clase que se está programando.
	El orden en que se programan las clases es en sí mismo una heurística; Yule menciona la reordenación de las clases que resultan difíciles de programar.

    \item \textbf{Retroceso y Reordenación (Backtracking and Reordering):} Si el algoritmo llega a un punto en el que no puede programar una clase sin violar las restricciones, emplea una forma de retroceso.
	Esto puede implicar:
    \begin{itemize}
        \item Reordenar las clases a programar.
        \item Restablecer el horario a un estado anterior e intentar una asignación diferente para una clase anterior.
        \item Modificar restricciones "blandas" (preferencias) para hacer factible un horario.
    \end{itemize}

    \item \textbf{Representación mediante Matriz de Disponibilidad (Availability Matrix Representation):} El algoritmo de Yule utiliza matrices de disponibilidad para representar eficientemente la disponibilidad del profesorado, las aulas y las clases a lo largo del tiempo.
	Estas matrices permiten al algoritmo verificar rápidamente los conflictos y determinar franjas horarias factibles.
	La Fórmula 1 calcula la disponibilidad de línea, las Fórmulas 2 y 3 prueban los tiempos disponibles, y la Fórmula 4 actualiza las matrices.

    \item \textbf{Heurística de Disponibilidad de Línea (Line Availability Heuristic):} El concepto de una "línea de requerimiento"\footnote{Una \textbf{"línea de requerimiento"} en el contexto del algoritmo de Yule es una estructura de datos que representa una unidad de programación, como una clase específica con su profesor y aula asignados (o por asignar).} es clave.
    El algoritmo se enfoca en asignar líneas de requerimiento completas de una vez, lo que simplifica el problema y asegura que todos los componentes de un curso (profesor, clase, aula) se consideren juntos.

    \item \textbf{Manejo de Preferencias (Preference Handling):} El algoritmo de Yule incorpora las preferencias del profesorado por franjas horarias o días libres específicos.
	Estas preferencias se tratan como restricciones "blandas"; el algoritmo intenta satisfacerlas, pero las violará si es necesario para lograr un horario factible.
	La Fórmula 4 maneja las preferencias, marcando los tiempos como "preferentemente no disponibles".

    \item \textbf{Asignación Dinámica de Días Libres (Dynamic Free Day Allocation):} El algoritmo puede ajustar dinámicamente los días libres del profesorado basándose en las dificultades de programación.
	Si un profesor en particular está causando un cuello de botella en la programación, el algoritmo puede intentar cambiar su día libre.
\end{enumerate}

El algoritmo de Yule toma varias decisiones heurísticas, incluyendo:

\begin{itemize}
    \item \textbf{Orden de Programación de Clases (Order of Class Scheduling):} El orden en que se programan las clases puede impactar significativamente el resultado.
    \item \textbf{Estrategia de Asignación de Días Libres (Free Day Allocation Strategy):} Decidir a qué miembros del profesorado dar días libres y cuándo ajustar esos días es un proceso heurístico.
    \item \textbf{Manejo de Preferencias (Handling Preferences):} El equilibrio entre satisfacer las preferencias y cumplir con las restricciones duras es una compensación heurística.
    \item \textbf{Estrategia de Retroceso (Backtracking Strategy):} Decidir cuándo retroceder, cuánto retroceder y qué cambios realizar durante el retroceso son decisiones heurísticas.
\end{itemize}

El algoritmo de Yule busca poblar una matriz de disponibilidad por línea de requerimiento, cada celda se ve como en la Figura~\ref{fig:yuleAlgorithm} donde la superposición de cada entidad definirá si la línea de requerimiento está disponible en la celda.
\begin{figure}[H]
    \centering
    \caption{Visualización de una celda de disponibilidad de línea de requerimiento}
    \includegraphics[width=.45\textwidth]{yule-algo.pdf}
    \label{fig:yuleAlgorithm}
\end{figure}

El algoritmo de Yule tiene 4 fórmulas básicas utilizadas para la generación de horarios.

Fórmula 1: \textbf{Matriz de Disponibilidad de Línea (Line Availability Matrix)} La primera fórmula, mostrada en la Ecuación \ref{eq:1}, calcula los elementos de la matriz de disponibilidad de línea \( _iE_{dp} \).
Esta matriz determina la disponibilidad general de una línea de clase particular \( i \) en un día específico \( d \) y período \( p \).

\begin{equation}
\label{eq:1}
_iE_{dp} = \left[\bigvee_{j \in S_i} (_jC_{dp})\right] \bigvee {}_zC'_{p}
\end{equation}

Donde:
\begin{itemize}
    \item \( _iE_{dp} \) representa la disponibilidad de la línea \( i \) en el día \( d \) y período \( p \).
    \item \( S_i \) es el conjunto de clases, profesores y aulas involucrados en el \( i \)-ésimo requerimiento.
    \item \( _jC_{dp} \) es un elemento de la matriz de disponibilidad \( C \) para el elemento \( j \) (clase, profesor o aula) que indica si el elemento \( j \) no está disponible (1) o está disponible (0) en el período \( p \) del día \( d \).
    \item El símbolo \(\bigvee\) denota la operación OR lógica (unión en el artículo original).
    La expresión \(\bigvee_{j \in S_i} (_jC_{dp})\) calcula la unión de las disponibilidades para todos los elementos en el conjunto \( S_i \).
    Si algún elemento no está disponible, el resultado será 1, lo que significa que la franja horaria generalmente no está disponible.
    \item \( _zC'_{p} \) es un elemento del vector \( C' \), que limita las horas del día en las que pueden comenzar las clases de múltiples períodos.
    \item La fórmula completa calcula si una franja horaria no está disponible porque un elemento en el requerimiento \( S_i \) no está disponible o el período \( p \) no es adecuado para una clase de duración \( z_i \).
\end{itemize}

Fórmulas 2 y 3: \textbf{Prueba de Lugar Disponible (Testing for Available Place)}, mostradas en las Ecuaciones \ref{eq:2} y \ref{eq:3}, prueban si una clase puede ser programada en una franja horaria específica.
La Fórmula 2 prueba clases de un solo período, mientras que la Fórmula 3 prueba clases de múltiples períodos.

\begin{equation}
\label{eq:2}
_iE_{dp} = 0 \quad \text{para una clase de un solo período}
\end{equation}

Esta fórmula simplemente establece que para que una clase de un solo período sea programada en el día \( d \) y período \( p \), el elemento correspondiente en la matriz de disponibilidad de línea \( _iE_{dp} \) debe ser 0, indicando que la franja está disponible.

\begin{equation}
\label{eq:3}
_iE_{dp} \vee (_iE_{d,p+1} \wedge 1) \vee \dots \vee (_iE_{d,k} \wedge 1) = 0
\end{equation}

Donde:
\begin{itemize}
    \item \( k = p + z_i - 1 \), donde \( z_i \) es la duración de la clase (número de períodos).
    \item La fórmula prueba si hay un bloque continuo de franjas horarias disponibles para una clase de múltiples períodos de duración \( z_i \) que comienza en el período \( p \).
    \item Cada término \( _iE_{d,p+n} \) para \( n \) en el rango de \( 0 \) a \( z_i-1 \) representa la disponibilidad del período \( p+n \).
    El término \( (_iE_{d,p+n} \wedge 1) \) enmascara eficazmente cualquier bit en \( _iE_{d,p+n} \) excepto el bit menos significativo, que representa la disponibilidad en el sentido más básico.
    \item Si el OR lógico de estos términos es igual a 0, entonces todos los períodos en el bloque están disponibles y la clase puede ser programada.
\end{itemize}

Fórmula 4: \textbf{Actualización de Matrices de Disponibilidad (Updating Availability Matrices)}, mostrada en la Ecuación \ref{eq:4}, actualiza las matrices de disponibilidad después de que una clase ha sido programada.

\begin{equation}
\label{eq:4}
_jC_{d, p+k} (\text{actualizado}) = _jC_{d, p+k} (\text{antiguo}) \vee 1
\end{equation}

Donde:
\begin{itemize}
    \item \( _jC_{d, p+k} \) representa el elemento de la matriz de disponibilidad para el ítem \( j \) (clase, profesor o aula) en el día \( d \) y período \( p+k \).
    \item \( k \) varía de 0 a \( z_i - 1 \), cubriendo todos los períodos abarcados por la clase programada.
    \item La fórmula establece el elemento de disponibilidad \( _jC_{d, p+k} \) en 1 (no disponible) realizando un OR lógico con 1.
    Esto asegura que la clase, profesor o aula programados se marquen como no disponibles durante la duración de la clase.
\end{itemize}


\subsection{Metodologías de trabajo}
Las metodologías de trabajo ágiles proporcionan marcos para organizar y gestionar el proceso de desarrollo de software de manera flexible, colaborativa e iterativa, permitiendo adaptarse a cambios y entregar valor de forma incremental.
La elección de una metodología ágil es fundamental para la gestión eficiente de un proyecto de grado como este \parencite{Beck2001}.

\subsubsection{Scrum}
Scrum es un marco de trabajo ágil ampliamente utilizado que organiza el desarrollo en ciclos cortos llamados Sprints (usualmente de 2 a 4 semanas), con roles definidos (Product Owner, Scrum Master, Equipo de Desarrollo), artefactos (Product Backlog, Sprint Backlog, Incremento) y eventos (Sprint Planning, Daily Scrum, Sprint Review, Sprint Retrospective) \parencite{SchwaberSutherland2020}.
Adoptar Scrum para este proyecto permitiría gestionar el desarrollo del sistema web de forma iterativa, priorizando funcionalidades, fomentando la colaboración y permitiendo la inspección y adaptación continua basada en el feedback y los avances.

\subsubsection{eXtreme Programming (XP)}
eXtreme Programming (XP) es una metodología ágil centrada en la entrega continua de software de alta calidad y en la adaptación a los requisitos cambiantes.
Como se puede ver en Figure~\ref{fig:xpWorkflowA} se basa en un conjunto de valores (comunicación, simplicidad, feedback, coraje y respeto) y prácticas técnicas robustas como la programación en parejas (pair programming), el desarrollo guiado por pruebas (Test-Driven Development - TDD), la integración continua (Continuous Integration - CI), la refactorización y las pequeñas entregas \parencite{Beck2004}.

XP podría ser particularmente adecuada para este proyecto de grado si se busca un enfoque fuerte en la calidad técnica del código, una colaboración muy estrecha entre los miembros del equipo y una capacidad alta de respuesta a los cambios en los requisitos o el diseño a medida que el proyecto evoluciona

\begin{figure}
    \centering
    \caption{XP lifecylce representation}\label{fig:xpWorkflowA}
    \includegraphics[width=.75\textwidth]{xp-workflow.pdf}

    \vspace{0.5em}
    \begin{minipage}{\textwidth}
        \small\textit{Note.} Fuente: \textcite{abrahamsson2017agile}.
    \end{minipage}
\end{figure}

\paragraph{Conclusión:} Based on the comparison of Scrum and XP and feasibility analysis on Appendix \ref{sec:methodology-justification}, eXtreme Programming has been chosed for the development of the system.


\subsection{Metodología de investigación}
\subsubsection{Definición}
El caso de estudio es un método empírico cuyo objetivo es investigar fenómenos contemporáneos en su contexto.

El caso de estudio tiene cuatro tipos diferentes de metodologías de investigación, que son:
\begin{itemize}
    \item Exploratorio: Se trata de generar ideas para hipótesis; responde a la pregunta "¿Qué está sucediendo?".
    \item Descriptivo: Presenta una descripción exhaustiva del fenómeno.
    \item Explicativo: Es una explicación del problema.
    No siempre en forma de una relación causal.
    \item De mejora: Mejora un cierto aspecto del fenómeno estudiado.
\end{itemize}

Un caso de estudio de tipo \textit{Positivista} se centra en recopilar evidencia para proposiciones formales a partir de la medición de variables, la prueba de hipótesis y la extracción de inferencias de muestras para comprender un fenómeno, mientras que un estudio de caso de tipo \textit{Interpretativo} recopila información a través de la interpretación que hace el participante de su contexto.

Se espera que un caso de estudio tenga: (1) preguntas de investigación, establecidas desde el principio, (2) los datos se recopilen de manera planificada y consistente, (3) se realicen inferencias a partir de los datos para responder a las preguntas de investigación, (4) explore un fenómeno, (5) las amenazas a la validez\footnote{Las \textbf{amenazas a la validez} en investigación se refieren a factores o influencias que podrían llevar a conclusiones incorrectas sobre el estudio.} del proyecto se aborden de manera sistemática.

\subsubsection{Protocolo del caso de estudio}
El protocolo del caso de estudio es un documento que contiene información sobre las decisiones de diseño e información sobre cómo llevar a cabo el proyecto.

\begin{table}[h]
\caption{Componentes del Protocolo del caso de estudio}
\begin{tabularx}{\textwidth}{@{}lX@{}}
\toprule
Sección & Contenido \\
\midrule
Preámbulo & Información sobre el propósito del protocolo, directrices para el almacenamiento de datos y documentos, publicación \\
Procedimientos generales & Breve descripción general del proyecto de investigación y del método de investigación de caso \\
Instrumentos de investigación & Guías de entrevista, cuestionarios, etc., que se utilizarán para garantizar la recopilación coherente de datos. \\
Directrices para el análisis de datos & Descripción detallada de los procedimientos de análisis de datos, incluido el esquema de datos. \\
\bottomrule
\end{tabularx}
\end{table}


\subsection{Arquitectura de Backend}
La arquitectura backend define la estructura interna del servidor, la lógica de negocio y la gestión de datos del sistema web, siendo fundamental para su escalabilidad, mantenibilidad y rendimiento.
Una arquitectura bien diseñada facilita la evolución del sistema y la integración de nuevas funcionalidades, como el módulo de predicción de horarios con ML, asegurando que la gestión del calendario académico sea robusta y eficiente \parencite{Richards2015}.

\subsubsection{Microservices}
La arquitectura de microservicios estructura una aplicación como una colección de servicios pequeños, autónomos y débilmente acoplados, cada uno enfocado en una capacidad de negocio específica y comunicándose a través de APIs (\textit{Application Programming Interfaces}) ligeras, usualmente sobre \texttt{HTTP} \parencite{Newman2015}.
Adoptar microservicios para el sistema de gestión académica permitiría desarrollar, desplegar y escalar independientemente componentes como la gestión de cursos, la generación de horarios, la predicción de elegibilidad y la gestión de usuarios, aumentando la resiliencia y flexibilidad del sistema global en Jala University.

\subsubsection{Clean Architecture}
La Arquitectura Limpia, propuesta por Robert C.
Martin, es un conjunto de principios de diseño de software que promueve la separación de intereses y la independencia de frameworks, UI y bases de datos, organizando el código en capas concéntricas (Entidades, Casos de Uso, Adaptadores de Interfaz, Frameworks y Drivers) \parencite{Martin2017}.
Aplicar Clean Architecture en el backend asegura que la lógica de negocio central (reglas académicas, algoritmos de predicción y generación de horarios) esté aislada de detalles externos, facilitando las pruebas, la mantenibilidad y la adaptabilidad a cambios tecnológicos futuros.

\begin{figure}
	\centering
	\includegraphics[width=0.5\textwidth]{clean-architecture.png}
	\caption{Fuente: \parencite{CleanCodeBlog}}
	\label{fig:cleanCodeBlog}
\end{figure}

The layers in Figure~\ref{fig:cleanCodeBlog} clearly show separation of logic and separation of concerns which allow a highly cohesive and lowly coupled system design.


\subsubsection{Domain Driven Design}
El Diseño Guiado por el Dominio (DDD) es un enfoque para el desarrollo de software complejo que se centra en modelar el dominio del negocio (en este caso, la gestión del calendario académico y la predicción de elegibilidad en Jala University) y plasmar ese modelo en el código, utilizando un lenguaje ubicuo compartido entre expertos del dominio y desarrolladores \parencite{Evans2003}.
DDD ayuda a gestionar la complejidad mediante conceptos como Entidades, Objetos Valor, Agregados, Repositorios y Servicios de Dominio, asegurando que el software refleje fielmente las reglas y procesos académicos.

\subsection{Arquitectura de Frontend}
La arquitectura frontend se ocupa de la estructura y organización del código que se ejecuta en el navegador del usuario, gestionando la interfaz de usuario (UI) y la interacción con el backend.
Una buena arquitectura frontend es esencial para proporcionar una experiencia de usuario fluida, receptiva y mantenible para estudiantes y administradores que utilicen el sistema de gestión del calendario académico \parencite{Osmani2017}.

\subsubsection{MVM}
El patrón Model-View-ViewModel (MVVM) es un patrón de diseño arquitectónico para interfaces de usuario que facilita la separación entre la lógica de presentación (ViewModel), la interfaz de usuario (View) y los datos (Model) \parencite{Smith2005}.
El ViewModel actúa como intermediario, exponiendo datos y comandos que la View puede enlazar (data binding), lo que simplifica el manejo del estado de la UI y mejora la testeabilidad.
Utilizar MVVM en el frontend (posiblemente con frameworks como \texttt{React} o \texttt{Vue} adaptándolo) puede ayudar a gestionar la complejidad de las interfaces para visualizar horarios, configurar parámetros de predicción y mostrar resultados.

\subsection{Tecnologías para Backend}
La elección de las tecnologías para el backend es crucial, ya que impacta directamente en el rendimiento, la escalabilidad, la seguridad y la facilidad de desarrollo y mantenimiento del servidor que aloja la lógica de negocio y gestiona los datos del sistema de gestión académica \parencite{FowlerBackend}.

\subsubsection{Lenguajes de programación}
La selección del lenguaje de programación para el backend depende de factores como el rendimiento requerido, el ecosistema de librerías disponibles (especialmente para ML y web), la experiencia del equipo y la compatibilidad con la infraestructura existente.

\paragraph{C\#}
es un lenguaje de programación moderno, orientado a objetos y fuertemente tipado, desarrollado por Microsoft y ejecutado sobre la plataforma \texttt{.NET}.
Ofrece un ecosistema robusto para el desarrollo web (\texttt{ASP.NET Core}), buen rendimiento y herramientas maduras, además de contar con librerías para ML (\texttt{ML.NET}), lo que lo convierte en una opción viable y productiva para construir los servicios backend del sistema \parencite{MicrosoftCSharp}.

\paragraph{Python}
es un lenguaje interpretado, dinámico y multipropósito, extremadamente popular en el ámbito de la ciencia de datos y el Machine Learning gracias a su sintaxis sencilla y a un vasto ecosistema de librerías especializadas (como \texttt{Scikit-learn}, \texttt{TensorFlow}, \texttt{PyTorch}) \parencite{PythonSoftwareFoundation}.
Su facilidad de uso y las potentes capacidades para ML lo hacen ideal para desarrollar el servicio de predicción de elegibilidad, pudiendo integrarse con otros servicios backend desarrollados en \texttt{C\#} u otros lenguajes a través de APIs.

\subsubsection{Bases de datos}
La elección de la base de datos adecuada es fundamental para almacenar y recuperar eficientemente la información académica, los horarios generados, los datos de entrenamiento para ML y la configuración del sistema.

\paragraph{Relacionales}
Las bases de datos relacionales (como \texttt{PostgreSQL}, \texttt{SQL Server}, \texttt{MySQL}) organizan los datos en tablas con esquemas predefinidos y utilizan \texttt{SQL} (Structured Query Language) para las consultas, garantizando la consistencia de los datos a través de transacciones \texttt{ACID} \parencite{Date2003}.
Son ideales para almacenar datos estructurados con relaciones bien definidas, como la información de cursos, estudiantes, profesores y asignaciones académicas, que forman el núcleo del sistema de gestión del calendario academíco.

\paragraph{No relacionales}
Las bases de datos \texttt{NoSQL} (Not Only SQL) ofrecen modelos de datos más flexibles (documental, clave-valor, columnar, grafo) y suelen priorizar la escalabilidad y la disponibilidad sobre la consistencia estricta (modelo \texttt{BASE}) \parencite{SadalegeFowler2012}.
Podrían ser útiles para almacenar datos menos estructurados o de gran volumen, como logs del sistema, resultados intermedios de la generación de horarios, o quizás para perfiles de usuario o configuraciones flexibles, complementando a la base de datos relacional principal.

\subsection{Tecnologías para Frontend}
Las tecnologías frontend determinan cómo se construye la interfaz de usuario para interactuar con el sistema de gestión del calendario académico.

\subsubsection{TypeScript - React}
\texttt{TypeScript} es un superconjunto de \texttt{JavaScript} que añade tipado estático opcional, mejorando la robustez y mantenibilidad del código frontend, especialmente en proyectos grandes \parencite{MicrosoftTypeScript}. \texttt{React}\footnote{\textbf{React} (también conocido como React.js o ReactJS) es una biblioteca de JavaScript de código abierto para construir interfaces de usuario, especialmente aplicaciones de una sola página (SPA).
Se utiliza para manejar la capa de vista para aplicaciones web y móviles.
Es mantenida por Facebook y una comunidad de desarrolladores individuales y compañías.} es una popular biblioteca de \texttt{JavaScript} para construir interfaces de usuario declarativas y basadas en componentes \parencite{FacebookReact}.
La combinación de \texttt{TypeScript} y \texttt{React} ofrece un entorno de desarrollo productivo y seguro para crear interfaces complejas y reactivas, adecuadas para visualizar horarios, formularios de gestión y resultados de predicciones de manera eficiente.

\subsection{Diagramas}
Los diagramas son herramientas visuales esenciales en la ingeniería de software para comunicar la estructura, el comportamiento y la arquitectura de un sistema de manera clara y concisa.
Utilizar diferentes tipos de diagramas ayuda a comprender y documentar distintos aspectos del sistema web de gestión académica \parencite{Fowler2003}.

\subsubsection{Diagrama de requerimientos}
Los diagramas de requerimientos, como los diagramas de casos de uso (\texttt{UML}), representan las interacciones entre los actores\footnote{Un \textbf{actor} en el modelado de sistemas (especialmente en UML) es una entidad externa (persona, sistema o dispositivo) que interactúa con el sistema para lograr un objetivo.
Representa un rol, no una persona específica.} (usuarios como estudiantes, administradores académicos) y el sistema, describiendo las funcionalidades que este debe ofrecer (e.g., "Consultar horario", "Generar propuesta de horario", "Predecir elegibilidad de estudiante", "Administrar cursos") \parencite{Jacobson1992}.
Estos diagramas son fundamentales en las etapas iniciales para definir el alcance del proyecto y asegurar que se comprendan y capturen las necesidades de los usuarios de Jala University.

Systems Modeling Language (SysML) es una extensión de UML para aplicaciones de ingeniería de sistemas que proporciona un diagrama de requisitos que sirve como mecanismo de representación gráfica para capturar requisitos textuales y sus relaciones \parencite{Friedenthal2014}.
Este tipo de diagrama—único de SysML—permite a los ingenieros modelar jerarquías de requisitos, dependencias y métodos de verificación dentro de un marco de modelado unificado.

\subsubsection{Diagrama C4}
El modelo C4 (Context, Containers, Components, Code) proporciona un marco para visualizar la arquitectura de software en diferentes niveles de abstracción, facilitando la comunicación entre distintos roles (desde negocio hasta desarrolladores) \parencite{BrownC4}.
Es especialmente útil para describir sistemas complejos como el propuesto, permitiendo entender cómo encaja en el ecosistema de Jala University y cómo se estructura internamente.

\paragraph{System context}
El diagrama de Contexto (Nivel 1 de C4) muestra el sistema de software en su totalidad como una caja negra, identificando sus interacciones con los usuarios principales (estudiantes, administradores) y otros sistemas externos con los que se integra (e.g., sistema de registro de estudiantes de Jala University, sistema de autenticación).
Este diagrama establece el alcance y los límites del sistema de gestión del calendario académico.

\paragraph{Containers}
El diagrama de Contenedores (Nivel 2 de C4) descompone el sistema en sus principales bloques ejecutables o desplegables, como aplicaciones web, APIs, bases de datos o microservicios (e.g., Web App Frontend, API Gateway, Servicio de Horarios, Servicio de Predicción ML, Base de Datos Académica).
Muestra las responsabilidades de alto nivel de cada contenedor y las tecnologías principales utilizadas, así como las interacciones entre ellos.

\paragraph{Components}
El diagrama de Componentes (Nivel 3 de C4) detalla la estructura interna de un contenedor específico, mostrando los principales componentes (agrupaciones lógicas de código, como clases o módulos) y sus interacciones dentro de ese contenedor.

\paragraph{Code}
El diagrama de Código (Nivel 4 de C4, opcional) ofrece una vista detallada a nivel de clases o entidades específicas dentro de un componente, utilizando notaciones como diagramas de clases \texttt{UML}.
Este nivel es útil para desarrolladores que necesitan entender la implementación detallada de una parte específica del sistema, como las clases que implementan un algoritmo de predicción o las entidades del dominio académico.

\subsubsection{Lenguaje Unificado de Modelado (UML)}
Es un lenguaje de modelado\footnote{Un \textbf{lenguaje de modelado} es cualquier lenguaje artificial que se puede usar para expresar información o conocimiento o sistemas en una estructura que está definida por un conjunto consistente de reglas.
Estas reglas se utilizan para la interpretación del significado de los componentes en la estructura.} de propósito general estandarizado en el campo de la ingeniería de software.
Se utiliza principalmente para visualizar, especificar, construir y documentar los artefactos de los sistemas de software, así como para el modelado de negocios y otros sistemas que no son de software.
UML proporciona un conjunto de técnicas de notación gráfica para crear modelos visuales de sistemas intensivos en software.
No es un lenguaje de programación, sino un lenguaje visual para describir diseños y procesos de software.
UML ayuda a los equipos a comunicarse, explorar diseños potenciales y validar el diseño arquitectónico del software.
UML abarca una amplia gama de tipos de diagramas, incluyendo diagramas de clases, diagramas de casos de uso, diagramas de secuencia, diagramas de actividad y diagramas de estado, cada uno con un propósito distinto en el modelado de diferentes aspectos de un sistema \cite{OMG2017, Fowler2003}.

\subsection{Tecnología de diagramado}
Las herramientas de diagramado asistido por software permiten crear y mantener diagramas de arquitectura, diseño y procesos de manera eficiente y consistente, a menudo integrándose con el código o sistemas de control de versiones.

\subsubsection{Structurizr}
Structurizr es un conjunto de herramientas (bibliotecas de código abierto y una plataforma web) para crear diagramas de arquitectura de software basados en el modelo C4, utilizando el enfoque de "diagramas como código", donde los modelos se definen en código (e.g., Java, Python) y los diagramas se generan a partir de él \parencite{BrownStructurizr}.
Esto asegura que la documentación arquitectónica se mantenga sincronizada con el código y facilita la automatización de la generación de diagramas C4 para el sistema de Jala University.

\subsubsection{PlantUML}
PlantUML es una herramienta de código abierto que permite generar diversos diagramas \texttt{UML} (secuencia, casos de uso, clases, actividad, componentes), así como otros tipos de diagramas (Arquitectura C4, ERD\footnote{\textbf{ERD (Entity-Relationship Diagram o Diagrama Entidad-Relación)} es un tipo de diagrama conceptual que muestra la estructura de una base de datos, representando las entidades (tablas), sus atributos (columnas) y las relaciones entre ellas.}, Wireframe\footnote{Un \textbf{wireframe} es un esquema visual básico de una interfaz de usuario, generalmente en blanco y negro, que se centra en la estructura, el contenido y la funcionalidad, sin detalles de diseño gráfico.
Sirve como un esqueleto para el diseño de la interfaz.}), a partir de una descripción textual simple \parencite{PlantUML}.
Es muy útil para crear rápidamente diagramas técnicos y mantenerlos bajo control de versiones junto con el código fuente, facilitando la documentación visual del diseño del sistema.


\subsection{Herramienta para gestionar tareas}
Las herramientas de gestión de tareas son esenciales para planificar, organizar y seguir el progreso del trabajo en un proyecto de desarrollo de software, especialmente cuando se utilizan metodologías ágiles.

\subsubsection{Clickup}
ClickUp es una plataforma de productividad y gestión de proyectos todo en uno que ofrece múltiples vistas (listas, tableros Kanban, calendarios, Gantt), personalización de flujos de trabajo y funcionalidades para la colaboración en equipo \parencite{ClickUp}.
Podría utilizarse para gestionar el backlog del producto, planificar sprints, asignar tareas y seguir el progreso general del desarrollo del sistema.

\subsubsection{Taskwarrior}
Taskwarrior es una herramienta de gestión de tareas de código abierto y basada en línea de comandos, que permite organizar listas de tareas pendientes de forma eficiente y flexible \parencite{Taskwarrior}.
Es una opción potente para desarrolladores que prefieren trabajar en la terminal, aunque requiere una curva de aprendizaje y es más adecuada para la gestión individual de tareas dentro del proyecto.

Como se observa en la Figura~\ref{fig:taskWarriorTaskOutput}, Taskwarrior mantiene metadatos de cada tarea, lo que permite el postprocesamiento\footnote{El \textbf{postprocesamiento} se refiere al procesamiento de datos que se realiza después de que han sido generados o recopilados.
En este caso, sería analizar los datos de Taskwarrior para crear visualizaciones o informes adicionales.} para crear gráficos de trabajo pendiente (burndown charts) y es posible agregar ADU \footnote{Un \textbf{Atributo Definido por el Usuario (ADU)} permite al usuario definir atributos.} para una mejor personalización.

\begin{figure}
	\caption{Ejemplo de salida de Taskwarrior}\label{fig:taskWarriorTaskOutput}
	\begin{verbatim}
Name               Value
------------------ ---------------------------------------------------------
Description        add figures to marco teorico
Status             Pending
Project            kawsay
Entered            2025-06-02 17:57:24 (45min)
Start              2025-06-02 18:09:17
Last modified      2025-06-02 18:09:17 (33min)
Tags               marco_teorico
Virtual tags       ACTIVE LATEST PENDING PROJECT READY TAGGED UDA UNBLOCKED
UUID               59a7c053-bc1d-4dcb-8b21-ea91646f6f41
Urgency            5.8
Estimate           45
	\end{verbatim}
\end{figure}

\subsubsection{Taiga}
Taiga es una plataforma de gestión de proyectos ágil, de código abierto y centrada en Scrum y Kanban, que ofrece tableros visuales, gestión de backlogs, seguimiento de issues\footnote{Un \textbf{issue} (incidencia o problema) en el contexto de la gestión de proyectos de software es una unidad de trabajo para rastrear una tarea, mejora, error o cualquier otro elemento que necesite ser abordado.} y wikis \parencite{Taiga}.
Representa una alternativa open-source a herramientas como Jira o ClickUp, adecuada para equipos que buscan una solución auto-alojada o gratuita para implementar metodologias agiles en el desarrollo del proyecto.

\paragraph{Conclusión:}
Para el desarrollo del sistema se utilizarán TaskWarrior y Taiga.
Porque ambas herramientas pueden combinarse para una gestión granular de tareas y también de alto nivel;
la API de Taiga abre posibilidades para otras automatizaciones relacionadas con la gestión de proyectos.

\subsection{Herramientas de versionamiento}
El control de versiones es indispensable en el desarrollo de software para gestionar los cambios en el código fuente a lo largo del tiempo, facilitar la colaboración entre desarrolladores y permitir la reversión a estados anteriores.

\subsubsection{GitHub}
GitHub es una plataforma de desarrollo colaborativo basada en \texttt{Git}\footnote{\textbf{Git} es un sistema de control de versiones distribuido, de código abierto y gratuito, diseñado para manejar desde proyectos pequeños hasta muy grandes con velocidad y eficiencia.
Fue creado por Linus Torvalds en 2005.} que ofrece hospedaje de repositorios, seguimiento de issues, revisión de código (Pull Requests)\footnote{Un \textbf{Pull Request (PR)} o Solicitud de Integración es una característica de plataformas de control de versiones como GitHub que permite a los desarrolladores proponer cambios al código base.
Facilita la revisión de código y la discusión antes de integrar los cambios en la rama principal.}, integración continua y otras herramientas para el ciclo de vida del desarrollo de software \parencite{GitHub}.
Es la plataforma de facto para muchos proyectos de código abierto y empresariales, y sería una opción robusta para alojar el código del sistema de Jala University, gestionar la colaboración y automatizar flujos de trabajo.

\subsubsection{Sourcehut}
SourceHut es una suite de herramientas de desarrollo de software de código abierto, enfocada en la simplicidad, la estabilidad y la filosofía Unix, que ofrece hospedaje \texttt{Git}, seguimiento de tickets, listas de correo, CI/CD y otras funcionalidades \parencite{SourceHut}.
Representa una alternativa más minimalista y centrada en el desarrollador a plataformas como GitHub o GitLab, atractiva para quienes valoran la transparencia y el control sobre sus herramientas de desarrollo.

\subsubsection{Nomenclatura de ramas, commits y pull requests}
Establecer una convención clara para nombrar ramas (e.g., \texttt{feature/nombre-funcionalidad}, \texttt{bugfix/descripcion-corta}, \texttt{release/v1.0}), escribir mensajes de commit significativos (e.g., siguiendo el formato Conventional Commits \parencite{ConventionalCommits}) y gestionar Pull Requests (PRs) de manera estructurada (con descripciones claras, revisiones obligatorias) es crucial para mantener un historial de cambios limpio, facilitar la revisión de código y mejorar la colaboración dentro del equipo de desarrollo del proyecto.

\subsection{Herramientas de diseño}
Las herramientas de diseño facilitan la creación de prototipos, wireframes y diseños visuales de la interfaz de usuario (UI) y la experiencia de usuario (UX), permitiendo iterar y validar ideas antes de escribir código.

\subsubsection{Figma}
Figma es una herramienta de diseño de interfaces basada en la web y colaborativa, que permite crear prototipos interactivos, sistemas de diseño y colaborar en tiempo real entre diseñadores y desarrolladores \parencite{Figma}.
Es ideal para diseñar las pantallas del sistema de gestión académica, definir flujos de usuario y crear un lenguaje visual consistente para la aplicación de Jala University.

\paragraph{Conclusion} this tool will be used to design initial prototypes of the system as to understand requirements from the use case point of view.


\subsection{Plan de pruebas}
Un plan de pruebas sistemático es esencial para asegurar la calidad, fiabilidad y corrección del sistema web desarrollado, verificando que cumple con los requerimientos funcionales y no funcionales definidos.

\subsubsection{ISO 9126}
La norma ISO/IEC 9126 (reemplazada en parte por ISO/IEC 25010) define un modelo de calidad para el software, clasificando los atributos de calidad en seis características principales: Funcionalidad, Fiabilidad, Usabilidad, Eficiencia, Mantenibilidad y Portabilidad \parencite{ISO9126}.
Utilizar este modelo como marco para el plan de pruebas del sistema de gestión académica permite definir criterios de aceptación claros y métricas específicas para evaluar cada aspecto de la calidad del software, asegurando una cobertura completa y sistemática de las pruebas.

\subsection{KPI}
Los Indicadores Clave de Rendimiento (KPIs - Key Performance Indicators) son métricas cuantificables utilizadas para evaluar el éxito de una organización, proyecto o actividad específica en relación con sus objetivos estratégicos \parencite{Parmenter2015}.
Para este proyecto, se definirán KPIs relevantes como la precisión del modelo de predicción de elegibilidad, el tiempo de generación de horarios, la reducción de conflictos en los horarios generados permitiendo medir objetivamente el impacto y la mejora lograda.

\subsection{i* Framework}
El marco de trabajo i* \parencite{yu1995} es un marco de modelado utilizado en la ingeniería de requisitos y el desarrollo de software.
Ofrece una forma de modelar y analizar las relaciones intencionales de los interesados.
Las relaciones intencionales son relaciones que incluyen actores, objetivos, tareas, recursos y dependencias sociales dentro de un contexto organizacional.
Ayuda a comprender el porqué detrás de los requisitos del sistema e identificar posibles compensaciones entre los objetivos de los diferentes interesados.

El marco está orientado a actores, lo que significa que utiliza la noción de \textit{actor} como concepto primitivo.
Los actores pueden ser cualquier cosa, desde individuos hasta organizaciones, que tengan intereses estratégicos e intencionalidad.
La intencionalidad de un actor se captura a través de \textit{objetivos}.
Un objetivo representa una condición o estado de cosas que el actor desea alcanzar.
Los objetivos pueden ser objetivos duros u objetivos blandos.
Los objetivos duros son precisos y tienen criterios claros de satisfacción, mientras que los objetivos blandos son menos precisos y representan cualidades o consideraciones deseadas.

La Tabla~\ref{tab:istarDependencyTypes} presenta los tipos de dependencia junto con su ontología\footnote{Una \textbf{ontología} en ciencias de la información es una especificación formal y explícita de una conceptualización compartida.}
\begin{table}
	\caption{Tipos de dependencia}\label{tab:istarDependencyTypes}
	\begin{tabularx}{\textwidth}{@{} llX @{}}
		\toprule
		\textbf{Ontología} & \textbf{Tipo} & \textbf{Descripción} \\
		\midrule
		Entidades & Recurso & Utilizado para representar el mundo como objetos. \\
		Actividades & Tarea & Producen un cambio en el mundo. \\
		Aserciones & Objetivos & Expresión de una condición o estado en el mundo. \\
		\bottomrule
	\end{tabularx}
\end{table}

\subsubsection{Modelos}
El marco de trabajo consta de dos modelos principales: el modelo de Dependencia Estratégica (DE) y el modelo de Racionalidad Estratégica (RE).

\begin{itemize}
    \item \textbf{Modelo de Dependencia Estratégica (DE):}
    Este modelo muestra la red de actores y sus dependencias.
    Ilustra cómo los actores dependen unos de otros para alcanzar objetivos, realizar tareas y obtener recursos.
    Este modelo es útil para identificar dependencias críticas y posibles vulnerabilidades en un sistema.
    \item \textbf{Modelo de Racionalidad Estratégica (RE):}
    Este modelo se centra en la racionalidad interna de los actores, mostrando cómo se alcanzan sus objetivos a través de tareas y cómo estas tareas utilizan recursos.
    Representa las razones detrás de las acciones de los actores y sus relaciones intencionales.
    Este modelo se utiliza para explorar formas alternativas de alcanzar objetivos y comprender las compensaciones involucradas.
\end{itemize}


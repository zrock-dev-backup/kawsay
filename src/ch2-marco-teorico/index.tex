\section{MARCO TEÓRICO REFERENCIAL}
Este capítulo establece el fundamento conceptual y técnico sobre el cual se desarrolla el presente proyecto de grado.
Se abordan temas clave como Machine Learning, algoritmos de generación de horarios, arquitecturas de software backend y frontend, metodologías de diseño y gestión de proyectos, indicadores de rendimiento, técnicas de diagramado, tecnologías específicas y herramientas de soporte, culminando con el plan de pruebas basado en estándares de calidad. El objetivo es proporcionar el contexto necesario para comprender las decisiones de diseño e implementación del sistema web para la gestión del calendario académico en Jala University, incluyendo la predicción de elegibilidad de estudiantes mediante Machine Learning.

\subsection{Machine Learning}
El Machine Learning (ML), o Aprendizaje Automático, es una rama de la inteligencia artificial que se enfoca en el desarrollo de sistemas capaces de aprender y mejorar a partir de la experiencia, sin ser explícitamente programados para cada tarea específica \parencite{Samuel1959}.
En el contexto de este proyecto, el ML se aplica para analizar datos históricos académicos y predecir la elegibilidad de los estudiantes para inscribirse en futuras asignaturas, buscando optimizar la asignación de cupos y la planificación académica, abordando así uno de los objetivos principales de entender el impacto del ML en este proceso.

\subsubsection{Supervised learning}
El aprendizaje supervisado es un paradigma del ML donde el algoritmo aprende a partir de un conjunto de datos previamente etiquetado, es decir, cada ejemplo de entrada está asociado a una salida correcta conocida \parencite{Bishop2006}.
El objetivo es entrenar un modelo que pueda predecir la etiqueta de salida para nuevas entradas no vistas. Para la predicción de elegibilidad de estudiantes en Jala University, se utilizarán datos históricos (calificaciones, cursos aprobados, plan de estudios) como entradas etiquetadas (elegible/no elegible para un curso específico) para entrenar un modelo predictivo, como podría ser una regresión logística o una máquina de soporte vectorial.

\subsubsection{Unsupervised learning}
A diferencia del aprendizaje supervisado, el aprendizaje no supervisado trabaja con datos no etiquetados, buscando descubrir patrones, estructuras o relaciones inherentes en la información sin una guía previa sobre las salidas correctas \parencite{Hastie2009}.
Técnicas comunes incluyen el clustering (agrupación) y la reducción de dimensionalidad. Aunque el enfoque principal del proyecto es supervisado para la predicción de elegibilidad, el análisis exploratorio de datos podría emplear técnicas no supervisadas para identificar grupos de estudiantes con perfiles académicos similares o detectar anomalías en los patrones de inscripción, complementando la comprensión del proceso actual.

\subsection{Timetable generation algorithms}
La generación de horarios académicos es un problema complejo de optimización combinatoria, clasificado como NP-difícil, que busca asignar recursos (profesores, aulas, horarios) a eventos (clases) respetando un conjunto de restricciones duras (inviolables) y blandas (deseables) \parencite{Schaerf1999}.
El desarrollo de algoritmos eficientes para esta tarea es crucial para instituciones educativas como Jala University, ya que impacta directamente en la satisfacción de estudiantes y docentes, y en la utilización de recursos, siendo un pilar para mejorar el proceso actual de gestión del calendario.

\subsubsection{Evolutionary and genetic algorithms}
Los algoritmos evolutivos, y en particular los algoritmos genéticos, son metaheurísticas inspiradas en la evolución biológica que resultan efectivas para problemas de optimización complejos como la generación de horarios \parencite{Eiben2003}.
Operan sobre una población de soluciones candidatas (horarios), aplicando operadores genéticos como selección, cruce y mutación para evolucionar hacia soluciones de mayor calidad (que satisfacen más restricciones o minimizan conflictos). Su capacidad para explorar amplios espacios de búsqueda los hace adecuados para encontrar horarios viables y optimizados en entornos con múltiples restricciones como el de Jala University.

\subsubsection{Constraint-based reasoning}
El razonamiento basado en restricciones (Constraint Satisfaction Problems - CSP) modela el problema de generación de horarios definiendo un conjunto de variables (clases), dominios (posibles asignaciones de tiempo/aula/profesor) y restricciones (reglas como no solapamiento, disponibilidad, capacidad) \parencite{Rossi2006}.
Algoritmos como backtracking, forward checking o arc consistency se utilizan para encontrar asignaciones que satisfagan todas las restricciones. Este enfoque es útil para garantizar la viabilidad de los horarios generados, asegurando que se cumplan las reglas fundamentales del calendario académico de la universidad.

\subsubsection{Linear programming/Integer programming}
La programación lineal (LP) y la programación entera (IP) son técnicas de investigación de operaciones que permiten modelar problemas de optimización mediante funciones objetivo lineales y restricciones lineales (con variables continuas en LP y enteras o binarias en IP) \parencite{Winston2004}.
La generación de horarios puede formularse como un problema de IP, donde las variables representan decisiones de asignación (e.g., si una clase se asigna a un horario específico) y el objetivo es optimizar alguna métrica (minimizar huecos, maximizar preferencias) sujeto a las restricciones académicas. Aunque computacionalmente intensivos, pueden garantizar soluciones óptimas para instancias de tamaño moderado.

\subsection{Backend architecture}
La arquitectura backend define la estructura interna del servidor, la lógica de negocio y la gestión de datos del sistema web, siendo fundamental para su escalabilidad, mantenibilidad y rendimiento.
Una arquitectura bien diseñada facilita la evolución del sistema y la integración de nuevas funcionalidades, como el módulo de predicción de horarios con ML, asegurando que la gestión del calendario académico sea robusta y eficiente \parencite{Richards2015}.

\subsubsection{Microservices}
La arquitectura de microservicios estructura una aplicación como una colección de servicios pequeños, autónomos y débilmente acoplados, cada uno enfocado en una capacidad de negocio específica y comunicándose a través de APIs (\textit{Application Programming Interfaces}) ligeras, usualmente sobre \texttt{HTTP} \parencite{Newman2015}.
Adoptar microservicios para el sistema de gestión académica permitiría desarrollar, desplegar y escalar independientemente componentes como la gestión de cursos, la generación de horarios, la predicción de elegibilidad y la gestión de usuarios, aumentando la resiliencia y flexibilidad del sistema global en Jala University.

\subsubsection{Clean Architecture}
La Arquitectura Limpia, propuesta por Robert C.
Martin, es un conjunto de principios de diseño de software que promueve la separación de intereses y la independencia de frameworks, UI y bases de datos, organizando el código en capas concéntricas (Entidades, Casos de Uso, Adaptadores de Interfaz, Frameworks y Drivers) \parencite{Martin2017}. Aplicar Clean Architecture en el backend asegura que la lógica de negocio central (reglas académicas, algoritmos de predicción y generación de horarios) esté aislada de detalles externos, facilitando las pruebas, la mantenibilidad y la adaptabilidad a cambios tecnológicos futuros.

\subsubsection{Domain Driven Design}
El Diseño Guiado por el Dominio (DDD) es un enfoque para el desarrollo de software complejo que se centra en modelar el dominio del negocio (en este caso, la gestión del calendario académico y la predicción de elegibilidad en Jala University) y plasmar ese modelo en el código, utilizando un lenguaje ubicuo compartido entre expertos del dominio y desarrolladores \parencite{Evans2003}.
DDD ayuda a gestionar la complejidad mediante conceptos como Entidades, Objetos Valor, Agregados, Repositorios y Servicios de Dominio, asegurando que el software refleje fielmente las reglas y procesos académicos.

\subsection{Frontend architecture}
La arquitectura frontend se ocupa de la estructura y organización del código que se ejecuta en el navegador del usuario, gestionando la interfaz de usuario (UI) y la interacción con el backend.
Una buena arquitectura frontend es esencial para proporcionar una experiencia de usuario fluida, receptiva y mantenible para estudiantes y administradores que utilicen el sistema de gestión del calendario académico \parencite{Osmani2017}.

\subsubsection{MVM}
El patrón Model-View-ViewModel (MVVM) es un patrón de diseño arquitectónico para interfaces de usuario que facilita la separación entre la lógica de presentación (ViewModel), la interfaz de usuario (View) y los datos (Model) \parencite{Smith2005}.
El ViewModel actúa como intermediario, exponiendo datos y comandos que la View puede enlazar (data binding), lo que simplifica el manejo del estado de la UI y mejora la testeabilidad. Utilizar MVVM en el frontend (posiblemente con frameworks como \texttt{React} o \texttt{Vue} adaptándolo) puede ayudar a gestionar la complejidad de las interfaces para visualizar horarios, configurar parámetros de predicción y mostrar resultados.

\subsection{Design Methodology}
La metodología de diseño guía el proceso de creación del sistema, desde la concepción de la idea hasta la implementación, enfocándose en comprender las necesidades del usuario y entregar valor de manera eficiente.
La elección de una metodología adecuada es clave para asegurar que el sistema web desarrollado satisfaga los requerimientos de Jala University y sus usuarios \parencite{Cooper2014}.

\subsubsection{Lean thinking}
El pensamiento Lean, originado en el Sistema de Producción Toyota y adaptado al desarrollo de software, se centra en la eliminación de desperdicios (actividades que no agregan valor), la entrega rápida de valor al cliente y el aprendizaje continuo a través de ciclos de construir-medir-aprender \parencite{WomackJones2003}.
Aplicar principios Lean en este proyecto implica enfocarse en las funcionalidades esenciales para la gestión del calendario y la predicción, validar hipótesis rápidamente con usuarios de Jala University y optimizar el flujo de desarrollo para entregar un producto mínimo viable (MVP) de forma temprana.

\subsubsection{Lean sprint design}
El Design Sprint, popularizado por Google Ventures, es un proceso intensivo de cinco días (o una versión adaptada) que comprime meses de trabajo de diseño y validación en una sola semana, utilizando principios Lean y Design Thinking para definir problemas, idear soluciones, prototipar y probar con usuarios reales \parencite{Knapp2016}.
Emplear un enfoque similar a un Design Sprint al inicio o en fases clave del proyecto puede acelerar la toma de decisiones sobre el diseño de la interfaz, el flujo de interacción para la gestión de horarios y la presentación de las predicciones de ML, asegurando que la solución esté alineada con las necesidades de Jala University.

\subsection{Metodologías de trabajo}
Las metodologías de trabajo ágiles proporcionan marcos para organizar y gestionar el proceso de desarrollo de software de manera flexible, colaborativa e iterativa, permitiendo adaptarse a cambios y entregar valor de forma incremental.
La elección de una metodología ágil es fundamental para la gestión eficiente de un proyecto de grado como este \parencite{Beck2001}.

\subsubsection{Scrum}
Scrum es un marco de trabajo ágil ampliamente utilizado que organiza el desarrollo en ciclos cortos llamados Sprints (usualmente de 2 a 4 semanas), con roles definidos (Product Owner, Scrum Master, Equipo de Desarrollo), artefactos (Product Backlog, Sprint Backlog, Incremento) y eventos (Sprint Planning, Daily Scrum, Sprint Review, Sprint Retrospective) \parencite{SchwaberSutherland2020}.
Adoptar Scrum para este proyecto permitiría gestionar el desarrollo del sistema web de forma iterativa, priorizando funcionalidades, fomentando la colaboración y permitiendo la inspección y adaptación continua basada en el feedback y los avances.

\subsubsection{Scrumban}
Scrumban es un enfoque híbrido que combina elementos de Scrum (como los roles, las reuniones y el enfoque iterativo) con la visualización del flujo de trabajo y la gestión de límites de trabajo en progreso (WIP) de Kanban \parencite{Kniberg2010}.
Podría ser útil para este proyecto si se busca la estructura de Scrum pero con una mayor flexibilidad en la gestión del flujo de tareas, permitiendo visualizar cuellos de botella en el desarrollo del sistema de gestión académica y optimizar la entrega continua de funcionalidades.

\subsection{KPI}
Los Indicadores Clave de Rendimiento (KPIs - Key Performance Indicators) son métricas cuantificables utilizadas para evaluar el éxito de una organización, proyecto o actividad específica en relación con sus objetivos estratégicos \parencite{Parmenter2015}.
Para este proyecto, se definirán KPIs relevantes como la precisión del modelo de predicción de elegibilidad, el tiempo de generación de horarios, la reducción de conflictos en los horarios generados, la satisfacción del usuario (administradores, estudiantes) y la tasa de adopción del nuevo sistema en Jala University, permitiendo medir objetivamente el impacto y la mejora lograda.

\subsection{Diagramas}
Los diagramas son herramientas visuales esenciales en la ingeniería de software para comunicar la estructura, el comportamiento y la arquitectura de un sistema de manera clara y concisa.
Utilizar diferentes tipos de diagramas ayuda a comprender y documentar distintos aspectos del sistema web de gestión académica \parencite{Fowler2003}.

\subsubsection{Diagrama de requerimientos}
Los diagramas de requerimientos, como los diagramas de casos de uso (\texttt{UML}), representan las interacciones entre los actores (usuarios como estudiantes, administradores académicos) y el sistema, describiendo las funcionalidades que este debe ofrecer (e.g., "Consultar horario", "Generar propuesta de horario", "Predecir elegibilidad de estudiante", "Administrar cursos") \parencite{Jacobson1992}.
Estos diagramas son fundamentales en las etapas iniciales para definir el alcance del proyecto y asegurar que se comprendan y capturen las necesidades de los usuarios de Jala University.

\subsubsection{Diagrama C4}
El modelo C4 (Context, Containers, Components, Code) proporciona un marco para visualizar la arquitectura de software en diferentes niveles de abstracción, facilitando la comunicación entre distintos roles (desde negocio hasta desarrolladores) \parencite{BrownC4}.
Es especialmente útil para describir sistemas complejos como el propuesto, permitiendo entender cómo encaja en el ecosistema de Jala University y cómo se estructura internamente.

\paragraph{System context}
El diagrama de Contexto (Nivel 1 de C4) muestra el sistema de software en su totalidad como una caja negra, identificando sus interacciones con los usuarios principales (estudiantes, administradores) y otros sistemas externos con los que se integra (e.g., sistema de registro de estudiantes de Jala University, sistema de autenticación).
Este diagrama establece el alcance y los límites del sistema de gestión del calendario académico.

\paragraph{Containers}
El diagrama de Contenedores (Nivel 2 de C4) descompone el sistema en sus principales bloques ejecutables o desplegables, como aplicaciones web, APIs, bases de datos o microservicios (e.g., Web App Frontend, API Gateway, Servicio de Horarios, Servicio de Predicción ML, Base de Datos Académica).
Muestra las responsabilidades de alto nivel de cada contenedor y las tecnologías principales utilizadas, así como las interacciones entre ellos.

\paragraph{Components}
El diagrama de Componentes (Nivel 3 de C4) detalla la estructura interna de un contenedor específico, mostrando los principales componentes (agrupaciones lógicas de código, como clases o módulos) y sus interacciones dentro de ese contenedor.
Por ejemplo, podría mostrar los componentes dentro del "Servicio de Horarios", como el "Generador de Horarios", el "Validador de Restricciones" y el "Repositorio de Horarios".

\paragraph{Code}
El diagrama de Código (Nivel 4 de C4, opcional) ofrece una vista detallada a nivel de clases o entidades específicas dentro de un componente, utilizando notaciones como diagramas de clases \texttt{UML}.
Este nivel es útil para desarrolladores que necesitan entender la implementación detallada de una parte específica del sistema, como las clases que implementan un algoritmo de predicción o las entidades del dominio académico.

\subsubsection{WebML}
Web Modeling Language (WebML) es un lenguaje de modelado visual específico para el diseño conceptual de aplicaciones web complejas, centrándose en la estructura de la información, la navegación y la composición de las páginas \parencite{Ceri2000}.
Utilizar WebML podría ayudar a diseñar la estructura hipertextual del sistema de gestión académica, definiendo las unidades de contenido (e.g., información del curso, horario del estudiante), las páginas y los enlaces de navegación entre ellas de manera formal antes de la implementación del frontend.

\subsection{Metodología de investigación}
\subsubsection{Definición}
El caso de estudio es un método empírico cuyo objetivo es investigar fenómenos contemporáneos en su contexto.

El caso de estudio tiene cuatro tipos diferentes de metodologías de investigación, que son:
\begin{itemize}
    \item Exploratorio: Se trata de generar ideas para hipótesis; responde a la pregunta "¿Qué está sucediendo?".
    \item Descriptivo: Presenta una descripción exhaustiva del fenómeno.
    \item Explicativo: Es una explicación del problema.
No siempre en forma de una relación causal.
    \item De mejora: Mejora un cierto aspecto del fenómeno estudiado.
\end{itemize}

Un caso de estudio de tipo \textit{Positivista} se centra en recopilar evidencia para proposiciones formales a partir de la medición de variables, la prueba de hipótesis y la extracción de inferencias de muestras para comprender un fenómeno, mientras que un estudio de caso de tipo \textit{Interpretativo} recopila información a través de la interpretación que hace el participante de su contexto.

Se espera que un caso de estudio tenga: (1) preguntas de investigación, establecidas desde el principio, (2) los datos se recopilen de manera planificada y consistente, (3) se realicen inferencias a partir de los datos para responder a las preguntas de investigación, (4) explore un fenómeno, (5) las amenazas a la validez del proyecto se aborden de manera sistemática.

\subsubsection{Protocolo del caso de estudio}
El protocolo del caso de estudio es un documento que contiene información sobre las decisiones de diseño e información sobre cómo llevar a cabo el proyecto.

\begin{longtable}{l|p{3in}}
\caption{Componentes del Protocolo del caso de estudio} \\
\hline
Sección & Contenido \\
\hline
\endfirsthead
Preámbulo & Información sobre el propósito del protocolo, directrices para el almacenamiento de datos y documentos, publicación \\ % Translated content
Procedimientos generales & Breve descripción general del proyecto de investigación y del método de investigación de caso \\ % Translated content
Instrumentos de investigación & Guías de entrevista, cuestionarios, etc., que se utilizarán para garantizar la recopilación coherente de datos. \\ % Translated content
Directrices para el análisis de datos & Descripción detallada de los procedimientos de análisis de datos, incluido el esquema de datos \\ % Translated content
\end{longtable}


\subsection{Tecnologías para Backend}
La elección de las tecnologías para el backend es crucial, ya que impacta directamente en el rendimiento, la escalabilidad, la seguridad y la facilidad de desarrollo y mantenimiento del servidor que aloja la lógica de negocio y gestiona los datos del sistema de gestión académica \parencite{FowlerBackend}.

\subsubsection{Programming languages}
La selección del lenguaje de programación para el backend depende de factores como el rendimiento requerido, el ecosistema de librerías disponibles (especialmente para ML y web), la experiencia del equipo y la compatibilidad con la infraestructura existente.

\paragraph{C\#}
\texttt{C\#} es un lenguaje de programación moderno, orientado a objetos y fuertemente tipado, desarrollado por Microsoft y ejecutado sobre la plataforma \texttt{.NET}.
Ofrece un ecosistema robusto para el desarrollo web (\texttt{ASP.NET Core}), buen rendimiento y herramientas maduras, además de contar con librerías para ML (\texttt{ML.NET}), lo que lo convierte en una opción viable y productiva para construir los servicios backend del sistema \parencite{MicrosoftCSharp}.

\paragraph{Python}
Python es un lenguaje interpretado, dinámico y multipropósito, extremadamente popular en el ámbito de la ciencia de datos y el Machine Learning gracias a su sintaxis sencilla y a un vasto ecosistema de librerías especializadas (como \texttt{Scikit-learn}, \texttt{TensorFlow}, \texttt{PyTorch}) \parencite{PythonSoftwareFoundation}.
Su facilidad de uso y las potentes capacidades para ML lo hacen ideal para desarrollar el servicio de predicción de elegibilidad, pudiendo integrarse con otros servicios backend desarrollados en \texttt{C\#} u otros lenguajes a través de APIs.

\subsubsection{Databases}
La elección de la base de datos adecuada es fundamental para almacenar y recuperar eficientemente la información académica, los horarios generados, los datos de entrenamiento para ML y la configuración del sistema.

\paragraph{Relational}
Las bases de datos relacionales (como \texttt{PostgreSQL}, \texttt{SQL Server}, \texttt{MySQL}) organizan los datos en tablas con esquemas predefinidos y utilizan \texttt{SQL} (Structured Query Language) para las consultas, garantizando la consistencia de los datos a través de transacciones \texttt{ACID} \parencite{Date2003}.
Son ideales para almacenar datos estructurados con relaciones bien definidas, como la información de cursos, estudiantes, profesores y asignaciones académicas, que forman el núcleo del sistema de gestión de Jala University.

\paragraph{Non-Relational}
Las bases de datos \texttt{NoSQL} (Not Only SQL) ofrecen modelos de datos más flexibles (documental, clave-valor, columnar, grafo) y suelen priorizar la escalabilidad y la disponibilidad sobre la consistencia estricta (modelo \texttt{BASE}) \parencite{SadalegeFowler2012}.
Podrían ser útiles para almacenar datos menos estructurados o de gran volumen, como logs del sistema, resultados intermedios de la generación de horarios, o quizás para perfiles de usuario o configuraciones flexibles, complementando a la base de datos relacional principal.

\subsection{Tecnologías para Frontend}
Las tecnologías frontend determinan cómo se construye la interfaz de usuario interactiva que los usuarios finales (estudiantes, administradores) utilizarán para interactuar con el sistema de gestión del calendario académico.

\subsubsection{TypeScript - React}
\texttt{TypeScript} es un superconjunto de \texttt{JavaScript} que añade tipado estático opcional, mejorando la robustez y mantenibilidad del código frontend, especialmente en proyectos grandes \parencite{MicrosoftTypeScript}. \texttt{React} es una popular biblioteca de \texttt{JavaScript} para construir interfaces de usuario declarativas y basadas en componentes \parencite{FacebookReact}.
La combinación de \texttt{TypeScript} y \texttt{React} ofrece un entorno de desarrollo productivo y seguro para crear interfaces complejas y reactivas, adecuadas para visualizar horarios, formularios de gestión y resultados de predicciones de manera eficiente.

\subsection{Herramientas de diseño}
Las herramientas de diseño facilitan la creación de prototipos, wireframes y diseños visuales de la interfaz de usuario (UI) y la experiencia de usuario (UX), permitiendo iterar y validar ideas antes de escribir código.

\subsubsection{Figma}
Figma es una herramienta de diseño de interfaces basada en la web y colaborativa, que permite crear prototipos interactivos, sistemas de diseño y colaborar en tiempo real entre diseñadores y desarrolladores \parencite{Figma}.
Es ideal para diseñar las pantallas del sistema de gestión académica, definir flujos de usuario y crear un lenguaje visual consistente para la aplicación de Jala University.

\subsubsection{Canvas}
El Business Model Canvas o herramientas similares de "canvas" (lienzo) como el Lean Canvas, son plantillas estratégicas que ayudan a visualizar y desarrollar modelos de negocio o propuestas de valor de forma estructurada y concisa \parencite{OsterwalderPigneur2010}.
Aunque no es una herramienta de diseño de UI, puede usarse en las fases iniciales para definir la propuesta de valor del sistema, identificar segmentos de usuarios clave (estudiantes, administradores de Jala U) y alinear el proyecto con los objetivos institucionales.

\subsection{Organizador de tareas}
Las herramientas de gestión de tareas son esenciales para planificar, organizar y seguir el progreso del trabajo en un proyecto de desarrollo de software, especialmente cuando se utilizan metodologías ágiles.

\subsubsection{Clickup}
ClickUp es una plataforma de productividad y gestión de proyectos todo en uno que ofrece múltiples vistas (listas, tableros Kanban, calendarios, Gantt), personalización de flujos de trabajo y funcionalidades para la colaboración en equipo \parencite{ClickUp}.
Podría utilizarse para gestionar el backlog del producto, planificar sprints (si se usa Scrum o Scrumban), asignar tareas y seguir el progreso general del desarrollo del sistema para Jala University.

\subsubsection{Taskwarrior}
Taskwarrior es una herramienta de gestión de tareas de código abierto y basada en línea de comandos, que permite organizar listas de tareas pendientes de forma eficiente y flexible \parencite{Taskwarrior}.
Es una opción potente para desarrolladores que prefieren trabajar en la terminal, aunque requiere una curva de aprendizaje y es más adecuada para la gestión individual de tareas dentro del proyecto.

\subsubsection{Taiga}
Taiga es una plataforma de gestión de proyectos ágil, de código abierto y centrada en Scrum y Kanban, que ofrece tableros visuales, gestión de backlogs, seguimiento de issues y wikis \parencite{Taiga}.
Representa una alternativa open-source a herramientas como Jira o ClickUp, adecuada para equipos que buscan una solución auto-alojada o gratuita para implementar Scrum o Scrumban en el desarrollo del proyecto.

\subsection{Herramientas de versionamiento}
El control de versiones es indispensable en el desarrollo de software para gestionar los cambios en el código fuente a lo largo del tiempo, facilitar la colaboración entre desarrolladores y permitir la reversión a estados anteriores.

\subsubsection{GitHub}
GitHub es una plataforma de desarrollo colaborativo basada en \texttt{Git} que ofrece hospedaje de repositorios, seguimiento de issues, revisión de código (Pull Requests), integración continua y otras herramientas para el ciclo de vida del desarrollo de software \parencite{GitHub}.
Es la plataforma de facto para muchos proyectos de código abierto y empresariales, y sería una opción robusta para alojar el código del sistema de Jala University, gestionar la colaboración y automatizar flujos de trabajo.

\subsubsection{Sourcehut}
SourceHut es una suite de herramientas de desarrollo de software de código abierto, enfocada en la simplicidad, la estabilidad y la filosofía Unix, que ofrece hospedaje \texttt{Git}, seguimiento de tickets, listas de correo, CI/CD y otras funcionalidades \parencite{SourceHut}.
Representa una alternativa más minimalista y centrada en el desarrollador a plataformas como GitHub o GitLab, atractiva para quienes valoran la transparencia y el control sobre sus herramientas de desarrollo.

\subsubsection{Nomenclatura de ramas, commits y pull requests}
Establecer una convención clara para nombrar ramas (e.g., \texttt{feature/nombre-funcionalidad}, \texttt{bugfix/descripcion-corta}, \texttt{release/v1.0}), escribir mensajes de commit significativos (e.g., siguiendo el formato Conventional Commits \parencite{ConventionalCommits}) y gestionar Pull Requests (PRs) de manera estructurada (con descripciones claras, revisiones obligatorias) es crucial para mantener un historial de cambios limpio, facilitar la revisión de código y mejorar la colaboración dentro del equipo de desarrollo del proyecto.

\subsection{Tecnología de diagramado}
Las herramientas de diagramado asistido por software permiten crear y mantener diagramas de arquitectura, diseño y procesos de manera eficiente y consistente, a menudo integrándose con el código o sistemas de control de versiones.

\subsubsection{Structurizr}
Structurizr es un conjunto de herramientas (bibliotecas de código abierto y una plataforma web) para crear diagramas de arquitectura de software basados en el modelo C4, utilizando el enfoque de "diagramas como código", donde los modelos se definen en código (e.g., Java, \texttt{C\#}, Python) y los diagramas se generan a partir de él \parencite{BrownStructurizr}.
Esto asegura que la documentación arquitectónica se mantenga sincronizada con el código y facilita la automatización de la generación de diagramas C4 para el sistema de Jala University.

\subsubsection{PlantUML}
PlantUML es una herramienta de código abierto que permite generar diversos diagramas \texttt{UML} (secuencia, casos de uso, clases, actividad, componentes), así como otros tipos de diagramas (Arquitectura C4, ERD, Wireframe), a partir de una descripción textual simple \parencite{PlantUML}.
Es muy útil para crear rápidamente diagramas técnicos y mantenerlos bajo control de versiones junto con el código fuente, facilitando la documentación visual del diseño del sistema.

\subsection{Plan de pruebas}
Un plan de pruebas sistemático es esencial para asegurar la calidad, fiabilidad y corrección del sistema web desarrollado, verificando que cumple con los requerimientos funcionales y no funcionales definidos.

\subsubsection{ISO 9126}
La norma ISO/IEC 9126 (reemplazada en parte por ISO/IEC 25010) define un modelo de calidad para el software, clasificando los atributos de calidad en seis características principales: Funcionalidad, Fiabilidad, Usabilidad, Eficiencia, Mantenibilidad y Portabilidad \parencite{ISO9126}.
Utilizar este modelo como marco para el plan de pruebas del sistema de gestión académica permite definir criterios de aceptación claros y métricas específicas para evaluar cada aspecto de la calidad del software, asegurando una cobertura completa y sistemática de las pruebas.

\subsection{Case study}
\subsubsection{Context}
This are the actors
insert i* figure and the analysis of actors and their relationships


\begin{longtable}{p{3in}|p{3in}}
\caption{Actor} \label{tab:objectivesNactions} \\
\hline
\textbf{Actor} & \textbf{Definicion} \\
\hline
\endfirsthead
\hline
\textbf{Actor} & \textbf{Definicion} \\
\hline
\endhead
\hline
\endfoot

Teaching Staff &

It is defined in two branches
\begin{itemize}
	\item Professor
	\item Faculty practitioner
\end{itemize}
I want to use the system to be hired, I believe that by teaching students will achieve their dreams
\\\hline

Student &
I want to achieve comfort and understanding so that I geto to learing and enjoy not survive.
\\\hline

Registrar &
I want the student to go through the curriculum




\end{longtable}

\subsubsection{Timetable generation process}
Faculty timetable
Curriculum
A module lasts 8weeks, there are courses with 10weeks duration
This is how the periods look like
\subsubsection{Student enrollment process}
A student may not be available, not be eligible for enrollment.
\subsubsection{Clash management}
Handling a class migration clash, student enrollment clash, opening a new class clash.

\subsection{Machine Learning}
El Machine Learning (ML), es una rama de la inteligencia artificial que se enfoca en el desarrollo de sistemas capaces de aprender y mejorar a partir de la experiencia, sin ser explícitamente programados para cada tarea específica \parencite{Samuel1959}.
En el contexto de este proyecto, el ML se aplica para analizar datos históricos académicos y predecir la elegibilidad de los estudiantes para inscribirse en futuras asignaturas, buscando optimizar la asignación de cupos y la planificación académica, abordando así uno de los objetivos principales de entender el impacto del ML en este proceso.

\paragraph{Aprendizaje no supervisado}
trabaja con datos no etiquetados\footnote{Los \textbf{datos no etiquetados} en machine learning son datos de entrada que no tienen una salida o categoría predefinida asociada.}, buscando descubrir patrones, estructuras o relaciones inherentes en la información sin una guía previa sobre las salidas correctas \parencite{Hastie2009}.
Técnicas comunes incluyen el clustering\footnote{El \textbf{clustering} (agrupamiento) es una técnica de aprendizaje no supervisado que consiste en agrupar un conjunto de objetos de tal manera que los objetos en el mismo grupo (llamado clúster) sean más similares entre sí que con los de otros grupos.} (agrupación) y la reducción de dimensionalidad\footnote{La \textbf{reducción de dimensionalidad} es el proceso de reducir el número de variables (o características) consideradas en un conjunto de datos, mientras se conserva la información importante.
Se utiliza para simplificar modelos, mejorar el rendimiento y facilitar la visualización.}.
Aunque el enfoque principal del proyecto es supervisado para la predicción de elegibilidad, el análisis exploratorio de datos podría emplear técnicas no supervisadas para identificar grupos de estudiantes con perfiles académicos similares o detectar anomalías\footnote{Las \textbf{anomalías} (o valores atípicos) son puntos de datos que difieren significativamente de otras observaciones.
En el contexto de datos de inscripción, podrían ser patrones inusuales que merecen investigación.} en los patrones de inscripción, complementando la comprensión del proceso actual.

\paragraph{Aprendizaje supervisado}
es un paradigma del ML donde el algoritmo aprende a partir de un conjunto de datos previamente etiquetado\footnote{Los \textbf{datos etiquetados} en machine learning son datos de entrada que ya tienen una salida o categoría correcta conocida asociada.
El modelo aprende a mapear las entradas a las salidas correctas.}, es decir, cada ejemplo de entrada está asociado a una salida correcta conocida \parencite{Bishop2006}.
El objetivo es entrenar un modelo que pueda predecir la etiqueta de salida para nuevas entradas no vistas.
Para la predicción de elegibilidad de estudiantes en Jala University, se utilizarán datos históricos (calificaciones, cursos aprobados, plan de estudios) como entradas etiquetadas (elegible/no elegible para un curso específico) para entrenar un modelo predictivo, como podría ser una regresión logística\footnote{La \textbf{regresión logística} es un algoritmo de aprendizaje supervisado utilizado para problemas de clasificación binaria (predecir una de dos categorías).
Modela la probabilidad de que una instancia pertenezca a una clase particular.} o una máquina de soporte vectorial.

La Figura~\ref{fig:mlComparison} representa cómo un modelo no supervisado utiliza datos no etiquetados para la clasificación y el aprendizaje supervisado utiliza datos etiquetados.

\begin{figure}
    \centering
    \caption{Representación esquemática de los paradigmas de aprendizaje, } \label{fig:mlComparison}
    \includegraphics[width=0.45\textwidth]{comparison-supervised-unsupervised.pdf}

    \vspace{0.5em}
    \begin{minipage}{\textwidth}
        \small\textit{Nota.} Fuente: \textcite{morimoto2021}.
    \end{minipage}
\end{figure}

Los Árboles de Decisión y los Bosques Aleatorios son dos modelos que se ajustan a los requisitos del proyecto.

\subsubsection{Árboles de Decisión}
Un árbol de decisión es un algoritmo de aprendizaje supervisado utilizado tanto para tareas de clasificación como de regresión\footnote{La \textbf{regresión} en machine learning es un tipo de tarea de aprendizaje supervisado donde el objetivo es predecir un valor numérico continuo (por ejemplo, el precio de una casa, la temperatura).
La \textbf{clasificación}, por otro lado, predice una etiqueta categórica (por ejemplo, si un correo es spam o no).}.
Opera particionando recursivamente el espacio de características\footnote{El \textbf{espacio de características} (feature space) es el espacio n-dimensional definido por las características (variables) utilizadas para describir las instancias de datos.
Cada instancia de datos es un punto en este espacio.} en regiones distintas basadas en una serie de reglas de decisión inferidas de los datos \parencite{quinlan1986induction}.
Cada nodo interno en el árbol representa una prueba sobre un atributo (característica), cada rama representa el resultado de la prueba, y cada nodo hoja representa una etiqueta de clase (para clasificación) o una predicción (para regresión).

El proceso de aprendizaje del árbol de decisión implica seleccionar las características más informativas para dividir los datos en cada nodo, con el objetivo de crear subconjuntos cada vez más homogéneos.
Este proceso continúa recursivamente hasta que se cumple un criterio de parada, como alcanzar una profundidad máxima del árbol, un número mínimo de muestras en un nodo, o lograr una clasificación (o predicción) perfecta dentro de un nodo.

\paragraph{Criterios de División}
La selección de la mejor característica para realizar la división es crucial en la construcción de árboles de decisión.
Los criterios de división comunes incluyen:

\begin{itemize}
    \item \textbf{Impureza de Gini (para clasificación):} Mide la impureza de un conjunto de instancias, donde valores más bajos indican una mayor homogeneidad.
Es una medida de cuán a menudo un elemento elegido al azar del conjunto sería incorrectamente etiquetado si fuera etiquetado aleatoriamente según la distribución de etiquetas en el subconjunto.
    \item \textbf{Ganancia de Información (para clasificación):} Mide la reducción de la entropía\footnote{La \textbf{entropía} en teoría de la información es una medida de la incertidumbre o impureza en un conjunto de datos.
En los árboles de decisión, se utiliza para cuantificar la homogeneidad de las etiquetas de clase en un nodo.} después de dividir por un atributo.
    \item \textbf{Error Cuadrático Medio (ECM) (para regresión):} Mide la diferencia cuadrática promedio entre los valores predichos y los reales.
\end{itemize}

\subsubsection{Bosques Aleatorios}
El algoritmo de Bosques Aleatorios (Random Forest), introducido por Breiman \cite{breiman2001random}, es un algoritmo de aprendizaje supervisado.
Es un conjunto (ensemble)\footnote{Un \textbf{método de ensamble} (ensemble method) en machine learning combina las predicciones de múltiples modelos base (como árboles de decisión) para producir una predicción final mejor y más robusta que la de cualquier modelo individual.} de árboles de decisión, donde cada árbol se entrena con un subconjunto aleatorio de los datos y un subconjunto aleatorio de las características.
La predicción final se obtiene agregando las predicciones de todos los árboles individuales.
Para tareas de clasificación, esto se hace típicamente por votación mayoritaria\footnote{La \textbf{votación mayoritaria} en un ensamble de clasificadores significa que la clase predicha es la que recibe más "votos" (predicciones) de los modelos individuales que componen el ensamble.}, mientras que para tareas de regresión, se utiliza la predicción promedio.

El algoritmo de Bosques Aleatorios aprovecha el principio de la "sabiduría de la multitud", combinando las diversas perspectivas de múltiples árboles de decisión para producir predicciones más precisas y estables que las que pueden proporcionar los árboles individuales.
Los pasos fundamentales involucrados en la construcción de un Bosque Aleatorio son:

\begin{enumerate}
    \item \textbf{Muestreo Bootstrap (Bootstrap Sampling):} El algoritmo selecciona aleatoriamente, con reemplazo, \textit{n} muestras del conjunto de datos de entrenamiento original para crear un conjunto de entrenamiento único para cada árbol.
    Este proceso, denominado bootstrapping, asegura que cada árbol se entrene con un subconjunto de datos ligeramente diferente.
    \item \textbf{Aleatoriedad de Características (Feature Randomness):} Para cada nodo dentro de un árbol, se elige un subconjunto aleatorio de \textit{m} características del total de \textit{p} características disponibles.
    Luego, el algoritmo identifica la división óptima para el nodo basándose en estas \textit{m} características.
    El parámetro \textit{m} es típicamente significativamente más pequeño que \textit{p}, un paso crucial para reducir la correlación\footnote{La \textbf{correlación} entre árboles en un bosque aleatorio se refiere a la tendencia de diferentes árboles a cometer errores similares en las mismas instancias.
Reducir esta correlación es clave para la efectividad del ensamble.} entre los árboles.
    \item \textbf{Crecimiento del Árbol (Tree Growth):} Cada árbol se cultiva hasta su máxima profundidad sin poda\footnote{La \textbf{poda} (pruning) en árboles de decisión es una técnica para reducir el tamaño del árbol eliminando secciones que proporcionan poco poder predictivo, con el fin de evitar el sobreajuste (overfitting).} (o hasta que se alcanza un tamaño mínimo de nodo predefinido).
    Esto permite que cada árbol capture relaciones complejas en sus datos de entrenamiento.
    \item \textbf{Agregación de Predicciones (Prediction Aggregation):} Para problemas de clasificación, la etiqueta de clase final se determina mediante votación mayoritaria entre todos los árboles.
    En escenarios de regresión, la predicción final es el promedio de las predicciones generadas por los árboles individuales.
\end{enumerate}

\paragraph{Ajuste de Hiperparámetros}
es una fase crítica en la optimización del rendimiento de un modelo de Bosques Aleatorios.
Los hiperparámetros clave que impactan significativamente el comportamiento del modelo incluyen:

\begin{itemize}
    \item \texttt{n\_estimators}: El número de árboles dentro del bosque.
    \item \texttt{max\_features}: El número de características consideradas al buscar la mejor división en cada nodo.
    \item \texttt{max\_depth}: La profundidad máxima permitida de los árboles individuales.
    \item \texttt{min\_samples\_split}: El número mínimo de muestras requerido para dividir un nodo interno.
    \item \texttt{min\_samples\_leaf}: El número mínimo de muestras obligatorio para residir en un nodo hoja.
\end{itemize}

Dos metodologías ampliamente utilizadas para el ajuste de hiperparámetros son la Búsqueda en Rejilla (Grid Search) y la Búsqueda Aleatoria (Random Search).

\begin{itemize}
	\item \textbf{Búsqueda en Rejilla (Grid Search):} es una técnica de búsqueda exhaustiva que evalúa sistemáticamente todas las combinaciones posibles de valores de hiperparámetros dentro de una rejilla predefinida \cite{bergstra2012random}.
	El usuario especifica un conjunto discreto de valores para cada hiperparámetro, y la Búsqueda en Rejilla explora meticulosamente cada combinación.

	\item \textbf{Búsqueda Aleatoria (Random Search):} es un método de búsqueda estocástica\footnote{Un \textbf{método estocástico} es aquel que incorpora algún elemento de aleatoriedad en su proceso.
	En la búsqueda aleatoria de hiperparámetros, las combinaciones se eligen al azar.} que muestrea aleatoriamente combinaciones de hiperparámetros de distribuciones especificadas \cite{bergstra2012random}.
	A diferencia de la exploración exhaustiva de la Búsqueda en Rejilla, la Búsqueda Aleatoria sondea un subconjunto aleatorio del espacio de hiperparámetros.
\end{itemize}

\paragraph{Conclusión:}
Para este proyecto vamos a utilizar Bosques Aleatorios.

\subsection{Algoritmos para la generacion de horarios}
La generación de horarios académicos es un problema complejo de optimización combinatoria, que busca asignar recursos (profesores, aulas, horarios) a eventos (clases) respetando un conjunto de restricciones duras\footnote{Las \textbf{restricciones duras} son condiciones que deben ser satisfechas obligatoriamente en una solución válida.
Violar una restricción dura invalida la solución (por ejemplo, un profesor no puede estar en dos lugares al mismo tiempo).} (inviolables) y blandas\footnote{Las \textbf{restricciones blandas} son condiciones deseables pero no obligatorias.
Violar una restricción blanda no invalida la solución, pero disminuye su calidad o preferencia (por ejemplo, un profesor prefiere dar clases por la mañana).} (deseables) \parencite{Schaerf1999}.
El desarrollo de algoritmos eficientes para esta tarea es crucial para instituciones educativas como Jala University, ya que impacta directamente en la satisfacción de estudiantes y docentes, y en la utilización de recursos, siendo un pilar para mejorar el proceso actual de gestión del calendario.

\subsubsection{Razonamiento basado en restricciones}
El razonamiento basado en restricciones (Constraint Satisfaction Problems - CSP) modela el problema de generación de horarios definiendo un conjunto de variables (clases), dominios\footnote{En un CSP, el \textbf{dominio} de una variable es el conjunto de todos los valores posibles que esa variable puede tomar.} (posibles asignaciones de tiempo/aula/profesor) y restricciones (reglas como no solapamiento, disponibilidad, capacidad) \parencite{Rossi2006}.
Algoritmos como backtracking\footnote{El \textbf{backtracking} (vuelta atrás) es una técnica algorítmica general para encontrar todas (o algunas) soluciones a problemas computacionales}, forward checking o arc consistency\footnote{La \textbf{consistencia de arco (arc consistency)} Un arco entre dos variables es consistente si para cada valor en el dominio de la primera variable, existe algún valor en el dominio de la segunda variable tal que la restricción entre ellas se satisface.} se utilizan para encontrar asignaciones que satisfagan todas las restricciones.
Este enfoque es útil para garantizar la viabilidad de los horarios generados, asegurando que se cumplan las reglas fundamentales del calendario académico de la universidad.

\subsubsection{Programación lineal/Programación entera}
La programación lineal\footnote{La \textbf{programación lineal (PL)} es un método matemático para optimizar (maximizar o minimizar) una función objetivo lineal, sujeta a un conjunto de restricciones lineales (igualdades y desigualdades).} y la programación entera\footnote{La \textbf{programación entera (PE)} es un tipo de problema de optimización matemática en el que algunas o todas las variables están restringidas a ser números enteros.
Es una extensión de la programación lineal.} son técnicas de investigación de operaciones que permiten modelar problemas de optimización mediante funciones objetivo lineales y restricciones lineales (con variables continuas en PL y enteras o binarias en PE) \parencite{Winston2004}.
La generación de horarios puede formularse como un problema de PE, donde las variables representan decisiones de asignación (e.g., si una clase se asigna a un horario específico) y el objetivo es optimizar alguna métrica (minimizar huecos, maximizar preferencias) sujeto a las restricciones académicas.
Aunque computacionalmente intensivos, pueden garantizar soluciones óptimas para instancias de tamaño moderado.

\subsubsection{Algoritmo heurístico}

En el contexto de la generación de horarios académicos, una heurística es una regla práctica, estrategia o método utilizado para encontrar un horario factible, aunque no necesariamente óptimo.
El problema de crear un horario universitario es notoriamente complejo.
El número de horarios posibles crece exponencialmente con el número de cursos, profesores, aulas e intervalos de tiempo.
Encontrar un horario óptimo es a menudo computacionalmente intratable\footnote{\textbf{Computacionalmente intratable} se refiere a un problema que no puede ser resuelto por ningún algoritmo en un tiempo razonable (generalmente tiempo polinomial) a medida que el tamaño de la entrada del problema crece.}, especialmente para instituciones grandes.
Por lo tanto, los algoritmos heurísticos son cruciales para abordar este desafío.

\paragraph{Conclusión}
, la generación de horarios se desarrollará utilizando el algoritmo heurístico de Yule \cite{Yule1969}.
Este algoritmo tiene las siguientes características:

\begin{enumerate}[label=\alph*.]
    \item \textbf{Asignación Basada en Restricciones (Constraint-Based Allocation)}: El algoritmo de Yule enfatiza el manejo efectivo de las restricciones.
	Estas restricciones pueden incluir:
    \begin{itemize}
        \item Disponibilidad del profesorado (días y horas en que no están disponibles).
        \item Disponibilidad de aulas (tamaño, equipamiento, conflictos).
        \item Prerrequisitos de los cursos.
        \item Preferencias de cursos de los estudiantes (evitando conflictos).
        \item Reglas institucionales (p. ej., límites en el tamaño de las clases, horarios específicos reservados para ciertas actividades).
    \end{itemize}

    \item \textbf{Asignación Iterativa (Iterative Allocation):} El algoritmo intenta iterativamente programar las clases una por una, considerando el estado actual del horario y las restricciones asociadas con la clase que se está programando.
	El orden en que se programan las clases es en sí mismo una heurística; Yule menciona la reordenación de las clases que resultan difíciles de programar.

    \item \textbf{Retroceso y Reordenación (Backtracking and Reordering):} Si el algoritmo llega a un punto en el que no puede programar una clase sin violar las restricciones, emplea una forma de retroceso.
	Esto puede implicar:
    \begin{itemize}
        \item Reordenar las clases a programar.
        \item Restablecer el horario a un estado anterior e intentar una asignación diferente para una clase anterior.
        \item Modificar restricciones "blandas" (preferencias) para hacer factible un horario.
    \end{itemize}

    \item \textbf{Representación mediante Matriz de Disponibilidad (Availability Matrix Representation):} El algoritmo de Yule utiliza matrices de disponibilidad para representar eficientemente la disponibilidad del profesorado, las aulas y las clases a lo largo del tiempo.
	Estas matrices permiten al algoritmo verificar rápidamente los conflictos y determinar franjas horarias factibles.
	La Fórmula 1 calcula la disponibilidad de línea, las Fórmulas 2 y 3 prueban los tiempos disponibles, y la Fórmula 4 actualiza las matrices.

    \item \textbf{Heurística de Disponibilidad de Línea (Line Availability Heuristic):} El concepto de una "línea de requerimiento"\footnote{Una \textbf{"línea de requerimiento"} en el contexto del algoritmo de Yule es una estructura de datos que representa una unidad de programación, como una clase específica con su profesor y aula asignados (o por asignar).} es clave.
    El algoritmo se enfoca en asignar líneas de requerimiento completas de una vez, lo que simplifica el problema y asegura que todos los componentes de un curso (profesor, clase, aula) se consideren juntos.

    \item \textbf{Manejo de Preferencias (Preference Handling):} El algoritmo de Yule incorpora las preferencias del profesorado por franjas horarias o días libres específicos.
	Estas preferencias se tratan como restricciones "blandas"; el algoritmo intenta satisfacerlas, pero las violará si es necesario para lograr un horario factible.
	La Fórmula 4 maneja las preferencias, marcando los tiempos como "preferentemente no disponibles".

    \item \textbf{Asignación Dinámica de Días Libres (Dynamic Free Day Allocation):} El algoritmo puede ajustar dinámicamente los días libres del profesorado basándose en las dificultades de programación.
	Si un profesor en particular está causando un cuello de botella en la programación, el algoritmo puede intentar cambiar su día libre.
\end{enumerate}

El algoritmo de Yule toma varias decisiones heurísticas, incluyendo:

\begin{itemize}
    \item \textbf{Orden de Programación de Clases (Order of Class Scheduling):} El orden en que se programan las clases puede impactar significativamente el resultado.
    \item \textbf{Estrategia de Asignación de Días Libres (Free Day Allocation Strategy):} Decidir a qué miembros del profesorado dar días libres y cuándo ajustar esos días es un proceso heurístico.
    \item \textbf{Manejo de Preferencias (Handling Preferences):} El equilibrio entre satisfacer las preferencias y cumplir con las restricciones duras es una compensación heurística.
    \item \textbf{Estrategia de Retroceso (Backtracking Strategy):} Decidir cuándo retroceder, cuánto retroceder y qué cambios realizar durante el retroceso son decisiones heurísticas.
\end{itemize}

El algoritmo de Yule busca poblar una matriz de disponibilidad por línea de requerimiento, cada celda se ve como en la Figura~\ref{fig:yuleAlgorithm} donde la superposición de cada entidad definirá si la línea de requerimiento está disponible en la celda.
\begin{figure}[H]
    \centering
    \caption{Visualización de una celda de disponibilidad de línea de requerimiento}
    \includegraphics[width=.45\textwidth]{yule-algo.pdf}
    \label{fig:yuleAlgorithm}
\end{figure}

El algoritmo de Yule tiene 4 fórmulas básicas utilizadas para la generación de horarios.

Fórmula 1: \textbf{Matriz de Disponibilidad de Línea (Line Availability Matrix)} La primera fórmula, mostrada en la Ecuación \ref{eq:1}, calcula los elementos de la matriz de disponibilidad de línea \( _iE_{dp} \).
Esta matriz determina la disponibilidad general de una línea de clase particular \( i \) en un día específico \( d \) y período \( p \).

\begin{equation}
\label{eq:1}
_iE_{dp} = \left[\bigvee_{j \in S_i} (_jC_{dp})\right] \bigvee {}_zC'_{p}
\end{equation}

Donde:
\begin{itemize}
    \item \( _iE_{dp} \) representa la disponibilidad de la línea \( i \) en el día \( d \) y período \( p \).
    \item \( S_i \) es el conjunto de clases, profesores y aulas involucrados en el \( i \)-ésimo requerimiento.
    \item \( _jC_{dp} \) es un elemento de la matriz de disponibilidad \( C \) para el elemento \( j \) (clase, profesor o aula) que indica si el elemento \( j \) no está disponible (1) o está disponible (0) en el período \( p \) del día \( d \).
    \item El símbolo \(\bigvee\) denota la operación OR lógica (unión en el artículo original).
    La expresión \(\bigvee_{j \in S_i} (_jC_{dp})\) calcula la unión de las disponibilidades para todos los elementos en el conjunto \( S_i \).
    Si algún elemento no está disponible, el resultado será 1, lo que significa que la franja horaria generalmente no está disponible.
    \item \( _zC'_{p} \) es un elemento del vector \( C' \), que limita las horas del día en las que pueden comenzar las clases de múltiples períodos.
    \item La fórmula completa calcula si una franja horaria no está disponible porque un elemento en el requerimiento \( S_i \) no está disponible o el período \( p \) no es adecuado para una clase de duración \( z_i \).
\end{itemize}

Fórmulas 2 y 3: \textbf{Prueba de Lugar Disponible (Testing for Available Place)}, mostradas en las Ecuaciones \ref{eq:2} y \ref{eq:3}, prueban si una clase puede ser programada en una franja horaria específica.
La Fórmula 2 prueba clases de un solo período, mientras que la Fórmula 3 prueba clases de múltiples períodos.

\begin{equation}
\label{eq:2}
_iE_{dp} = 0 \quad \text{para una clase de un solo período}
\end{equation}

Esta fórmula simplemente establece que para que una clase de un solo período sea programada en el día \( d \) y período \( p \), el elemento correspondiente en la matriz de disponibilidad de línea \( _iE_{dp} \) debe ser 0, indicando que la franja está disponible.

\begin{equation}
\label{eq:3}
_iE_{dp} \vee (_iE_{d,p+1} \wedge 1) \vee \dots \vee (_iE_{d,k} \wedge 1) = 0
\end{equation}

Donde:
\begin{itemize}
    \item \( k = p + z_i - 1 \), donde \( z_i \) es la duración de la clase (número de períodos).
    \item La fórmula prueba si hay un bloque continuo de franjas horarias disponibles para una clase de múltiples períodos de duración \( z_i \) que comienza en el período \( p \).
    \item Cada término \( _iE_{d,p+n} \) para \( n \) en el rango de \( 0 \) a \( z_i-1 \) representa la disponibilidad del período \( p+n \).
    El término \( (_iE_{d,p+n} \wedge 1) \) enmascara eficazmente cualquier bit en \( _iE_{d,p+n} \) excepto el bit menos significativo, que representa la disponibilidad en el sentido más básico.
    \item Si el OR lógico de estos términos es igual a 0, entonces todos los períodos en el bloque están disponibles y la clase puede ser programada.
\end{itemize}

Fórmula 4: \textbf{Actualización de Matrices de Disponibilidad (Updating Availability Matrices)}, mostrada en la Ecuación \ref{eq:4}, actualiza las matrices de disponibilidad después de que una clase ha sido programada.

\begin{equation}
\label{eq:4}
_jC_{d, p+k} (\text{actualizado}) = _jC_{d, p+k} (\text{antiguo}) \vee 1
\end{equation}

Donde:
\begin{itemize}
    \item \( _jC_{d, p+k} \) representa el elemento de la matriz de disponibilidad para el ítem \( j \) (clase, profesor o aula) en el día \( d \) y período \( p+k \).
    \item \( k \) varía de 0 a \( z_i - 1 \), cubriendo todos los períodos abarcados por la clase programada.
    \item La fórmula establece el elemento de disponibilidad \( _jC_{d, p+k} \) en 1 (no disponible) realizando un OR lógico con 1.
    Esto asegura que la clase, profesor o aula programados se marquen como no disponibles durante la duración de la clase.
\end{itemize}

\subsection{KPI}
Los Indicadores Clave de Rendimiento (KPIs - Key Performance Indicators) son métricas cuantificables utilizadas para evaluar el éxito de una organización, proyecto o actividad específica en relación con sus objetivos estratégicos \parencite{Parmenter2015}.
Para este proyecto, se definirán KPIs relevantes como la precisión del modelo de predicción de elegibilidad, el tiempo de generación de horarios, la reducción de conflictos en los horarios generados permitiendo medir objetivamente el impacto y la mejora lograda.


\subsection{Metodologías de trabajo}
Las metodologías de trabajo ágiles proporcionan marcos para organizar y gestionar el proceso de desarrollo de software de manera flexible, colaborativa e iterativa, permitiendo adaptarse a cambios y entregar valor de forma incremental.
La elección de una metodología ágil es fundamental para la gestión eficiente de un proyecto de grado como este \parencite{Beck2001}.

\subsubsection{Scrum}
Scrum es un marco de trabajo ágil ampliamente utilizado que organiza el desarrollo en ciclos cortos llamados Sprints (usualmente de 2 a 4 semanas), con roles definidos (Product Owner, Scrum Master, Equipo de Desarrollo), artefactos (Product Backlog, Sprint Backlog, Incremento) y eventos (Sprint Planning, Daily Scrum, Sprint Review, Sprint Retrospective) \parencite{SchwaberSutherland2020}.
Adoptar Scrum para este proyecto permitiría gestionar el desarrollo del sistema web de forma iterativa, priorizando funcionalidades, fomentando la colaboración y permitiendo la inspección y adaptación continua basada en el feedback y los avances.

\subsubsection{eXtreme Programming (XP)}
eXtreme Programming (XP) es una metodología ágil centrada en la entrega continua de software de alta calidad y en la adaptación a los requisitos cambiantes.
Como se puede ver en Figure~\ref{fig:xpWorkflowA} se basa en un conjunto de valores (comunicación, simplicidad, feedback, coraje y respeto) y prácticas técnicas robustas como la programación en parejas (pair programming), el desarrollo guiado por pruebas (Test-Driven Development - TDD), la integración continua (Continuous Integration - CI), la refactorización y las pequeñas entregas \parencite{Beck2004}.

XP podría ser particularmente adecuada para este proyecto de grado si se busca un enfoque fuerte en la calidad técnica del código, una colaboración muy estrecha entre los miembros del equipo y una capacidad alta de respuesta a los cambios en los requisitos o el diseño a medida que el proyecto evoluciona

\begin{figure}
    \centering
    \caption{XP lifecylce representation}\label{fig:xpWorkflowA}
    \includegraphics[width=.75\textwidth]{xp-workflow.pdf}

    \vspace{0.5em}
    \begin{minipage}{\textwidth}
        \small\textit{Note.} Fuente: \textcite{abrahamsson2017agile}.
    \end{minipage}
\end{figure}

\paragraph{Conclusión:} Based on the comparison of Scrum and XP and feasibility analysis on Appendix \ref{sec:methodology-justification}, eXtreme Programming has been chosed for the development of the system.


\subsection{Metodología de investigación}
\subsubsection{Definición}
El caso de estudio es un método empírico cuyo objetivo es investigar fenómenos contemporáneos en su contexto.

El caso de estudio tiene cuatro tipos diferentes de metodologías de investigación, que son:
\begin{itemize}
    \item Exploratorio: Se trata de generar ideas para hipótesis; responde a la pregunta "¿Qué está sucediendo?".
    \item Descriptivo: Presenta una descripción exhaustiva del fenómeno.
    \item Explicativo: Es una explicación del problema.
    No siempre en forma de una relación causal.
    \item De mejora: Mejora un cierto aspecto del fenómeno estudiado.
\end{itemize}

Un caso de estudio de tipo \textit{Positivista} se centra en recopilar evidencia para proposiciones formales a partir de la medición de variables, la prueba de hipótesis y la extracción de inferencias de muestras para comprender un fenómeno, mientras que un estudio de caso de tipo \textit{Interpretativo} recopila información a través de la interpretación que hace el participante de su contexto.

Se espera que un caso de estudio tenga: (1) preguntas de investigación, establecidas desde el principio, (2) los datos se recopilen de manera planificada y consistente, (3) se realicen inferencias a partir de los datos para responder a las preguntas de investigación, (4) explore un fenómeno, (5) las amenazas a la validez\footnote{Las \textbf{amenazas a la validez} en investigación se refieren a factores o influencias que podrían llevar a conclusiones incorrectas sobre el estudio.} del proyecto se aborden de manera sistemática.

\subsubsection{Protocolo del caso de estudio}
El protocolo del caso de estudio es un documento que contiene información sobre las decisiones de diseño e información sobre cómo llevar a cabo el proyecto.

\begin{table}[h]
\caption{Componentes del Protocolo del caso de estudio}
\begin{tabularx}{\textwidth}{@{}lX@{}}
\toprule
Sección & Contenido \\
\midrule
Preámbulo & Información sobre el propósito del protocolo, directrices para el almacenamiento de datos y documentos, publicación \\
Procedimientos generales & Breve descripción general del proyecto de investigación y del método de investigación de caso \\
Instrumentos de investigación & Guías de entrevista, cuestionarios, etc., que se utilizarán para garantizar la recopilación coherente de datos. \\
Directrices para el análisis de datos & Descripción detallada de los procedimientos de análisis de datos, incluido el esquema de datos. \\
\bottomrule
\end{tabularx}
\end{table}


\subsection{Arquitectura de Backend}
La arquitectura backend define la estructura interna del servidor, la lógica de negocio y la gestión de datos del sistema web, siendo fundamental para su escalabilidad, mantenibilidad y rendimiento.
Una arquitectura bien diseñada facilita la evolución del sistema y la integración de nuevas funcionalidades, como el módulo de predicción de horarios con ML, asegurando que la gestión del calendario académico sea robusta y eficiente \parencite{Richards2015}.

\subsubsection{Microservicios}
La arquitectura de microservicios estructura una aplicación como una colección de servicios pequeños, autónomos y débilmente acoplados\footnote{El \textbf{acoplamiento débil} (loose coupling) en diseño de software es un enfoque donde los componentes de un sistema tienen poca o ninguna dependencia directa entre sí.
Esto permite que los cambios en un componente tengan un impacto mínimo en otros.}, cada uno enfocado en una capacidad de negocio específica y comunicándose a través de APIs (\textit{Application Programming Interfaces}) ligeras, usualmente sobre \texttt{HTTP} \parencite{Newman2015}.
Adoptar microservicios para el sistema de gestión académica permitiría desarrollar, desplegar y escalar independientemente componentes como la gestión de cursos, la generación de horarios, la predicción de elegibilidad y la gestión de usuarios, aumentando la resiliencia y flexibilidad del sistema global en Jala University.

\subsubsection{Arquitectura Limpia}
La Arquitectura Limpia, propuesta por Robert C.
Martin, es un conjunto de principios de diseño de software que promueve la separación de intereses\footnote{La \textbf{separación de intereses (Separation of Concerns - SoC)} es un principio de diseño para separar un programa informático en distintas secciones, de modo que cada sección aborde una preocupación o aspecto distinto.
Esto mejora la modularidad y la mantenibilidad.} y la independencia de frameworks, UI y bases de datos, organizando el código en capas concéntricas (Entidades, Casos de Uso, Adaptadores de Interfaz, Frameworks y Drivers) \parencite{Martin2017}.
Aplicar Clean Architecture en el backend asegura que la lógica de negocio central (reglas académicas, algoritmos de predicción y generación de horarios) esté aislada de detalles externos, facilitando las pruebas, la mantenibilidad y la adaptabilidad a cambios tecnológicos futuros.

\begin{figure}
    \centering
	\caption{Detalle de la Arquitectura Limpia} \label{fig:cleanCodeBlog}
	\includegraphics[width=0.6\textwidth]{clean-architecture.pdf}

    \vspace{0.5em}
    \begin{minipage}{\textwidth}
        \small\textit{Nota.} Fuente: \textcite{CleanCodeBlog}
    \end{minipage}
\end{figure}

Las capas en la Figura~\ref{fig:cleanCodeBlog} muestran claramente la separación de la lógica y la separación de responsabilidades, lo que permite un diseño de sistema altamente cohesivo y débilmente acoplado.


% \subsubsection{Diseño Guiado por el Dominio}
% El Diseño Guiado por el Dominio (DDD) es un enfoque para el desarrollo de software complejo que se centra en modelar el dominio del negocio (en este caso, la gestión del calendario académico y la predicción de elegibilidad en Jala University) y plasmar ese modelo en el código, utilizando un lenguaje ubicuo\footnote{El \textbf{lenguaje ubicuo (ubiquitous language)} en DDD es un lenguaje común y compartido desarrollado por el equipo (desarrolladores, expertos del dominio, usuarios) para describir todos los aspectos del dominio del software.
% Ayuda a evitar malentendidos y a asegurar que el modelo de software refleje fielmente el negocio.} compartido entre expertos del dominio y desarrolladores \parencite{Evans2003}.
% DDD ayuda a gestionar la complejidad mediante conceptos como Entidades, Objetos Valor, Agregados\footnote{Un \textbf{Agregado (Aggregate)} en DDD es un clúster de entidades y objetos valor que se tratan como una única unidad conceptual.
% Tiene una raíz (la Entidad Raíz del Agregado) que es el único punto de entrada para acceder o modificar el agregado.}, Repositorios\footnote{Un \textbf{Repositorio (Repository)} en DDD es un patrón de diseño que media entre el dominio y las capas de mapeo de datos, utilizando una interfaz similar a una colección para acceder a los objetos del dominio.} y Servicios de Dominio\footnote{Un \textbf{Servicio de Dominio (Domain Service)} en DDD encapsula lógica de dominio que no encaja naturalmente dentro de una entidad u objeto valor.
% Representa una operación o proceso significativo en el dominio.}, asegurando que el software refleje fielmente las reglas y procesos académicos.

\subsection{Arquitectura de Frontend}
La arquitectura frontend se ocupa de la estructura y organización del código que se ejecuta en el navegador del usuario, gestionando la interfaz de usuario (UI) y la interacción con el backend.
Una buena arquitectura frontend es esencial para proporcionar una experiencia de usuario fluida, receptiva y mantenible para estudiantes y administradores que utilicen el sistema de gestión del calendario académico \parencite{Osmani2017}.

\subsubsection{MVM}
El patrón Model-View-ViewModel (MVVM) es un patrón de diseño arquitectónico para interfaces de usuario que facilita la separación entre la lógica de presentación (ViewModel), la interfaz de usuario (View) y los datos (Model) \parencite{Smith2005}.
El ViewModel actúa como intermediario, exponiendo datos y comandos que la View puede enlazar (data binding), lo que simplifica el manejo del estado de la UI y mejora la testeabilidad.
Utilizar MVVM en el frontend (posiblemente con frameworks como \texttt{React} o \texttt{Vue} adaptándolo) puede ayudar a gestionar la complejidad de las interfaces para visualizar horarios, configurar parámetros de predicción y mostrar resultados.

\subsection{Tecnologías para Backend}
\subsubsection{Lenguajes de programación}

\paragraph{C\#}
De acuerdo a \parencite{MicrosoftCSharp} s un lenguaje de programación moderno, orientado a objetos y fuertemente tipado, desarrollado por Microsoft y ejecutado sobre la \textbf{plataforma .NET}.
Esta plataforma es un marco de desarrollo de software gratuito y de código abierto para crear diferentes tipos de aplicaciones, como web, móviles, de escritorio, juegos e IoT; desarrollado por Microsoft, incluye lenguajes como C\#, F\# y VB.NET, y un amplio conjunto de bibliotecas y herramientas.
C\# ofrece un ecosistema robusto para el desarrollo web a través de \textbf{ASP.NET Core}, un framework de código abierto y multiplataforma para crear aplicaciones web modernas, basadas en la nube y conectadas a Internet, también desarrollado por Microsoft; es una reescritura de ASP.NET y funciona sobre .NET Core o .NET Framework.
Además, C\# proporciona buen rendimiento y herramientas maduras.

\paragraph{Python}
es un lenguaje interpretado, dinámico y multipropósito, extremadamente popular en el ámbito de la ciencia de datos y el Machine Learning gracias a su sintaxis sencilla y a un vasto ecosistema de librerías especializadas.
Una de estas librerías es \textbf{Scikit-learn}, una popular biblioteca de machine learning de código abierto para Python que proporciona herramientas simples y eficientes para el análisis predictivo de datos, construida sobre NumPy, SciPy y matplotlib.
La facilidad de uso de Python y sus potentes capacidades para ML lo hacen ideal para desarrollar el servicio de predicción de elegibilidad, pudiendo integrarse con otros servicios backend desarrollados en \texttt{C\#} u otros lenguajes a través de APIs.

\subsubsection{Bases de datos}
La elección de la base de datos adecuada es fundamental para almacenar y recuperar eficientemente la información académica, los horarios generados, los datos de entrenamiento para ML y la configuración del sistema.

\paragraph{Relacionales}
Las bases de datos relacionales (como \texttt{PostgreSQL}\footnote{\textbf{PostgreSQL} es un potente sistema de gestión de bases de datos relacionales de objetos de código abierto, conocido por su fiabilidad, robustez de características y rendimiento.}, \texttt{SQL Server}\footnote{\textbf{SQL Server} es un sistema de gestión de bases de datos relacionales desarrollado por Microsoft.
Ofrece una amplia gama de herramientas de análisis de datos, generación de informes e integración.}, \texttt{MySQL}\footnote{\textbf{MySQL} es un popular sistema de gestión de bases de datos relacionales de código abierto, ampliamente utilizado en aplicaciones web y como parte de la pila de software LAMP (Linux, Apache, MySQL, PHP/Python/Perl).}) organizan los datos en tablas con esquemas predefinidos y utilizan \texttt{SQL} (Structured Query Language)\footnote{\textbf{SQL (Structured Query Language)} es un lenguaje estándar utilizado para gestionar y manipular bases de datos relacionales.
Permite realizar consultas, insertar, actualizar y eliminar datos, así como definir y modificar la estructura de la base de datos.} para las consultas, garantizando la consistencia de los datos a través de transacciones \texttt{ACID} \parencite{Date2003}.
Son ideales para almacenar datos estructurados con relaciones bien definidas, como la información de cursos, estudiantes, profesores y asignaciones académicas, que forman el núcleo del sistema de gestión del calendario academíco.

\paragraph{No relacionales}
Las bases de datos \texttt{NoSQL} (Non SQL) ofrecen modelos de datos más flexibles que los relacionales tradicionales.
Estos modelos incluyen:
el documental, donde una base de datos documental (un tipo de base de datos NoSQL) almacena datos en forma de documentos, a menudo en formato JSON o BSON, y cada documento es una estructura de datos auto-contenida;
el de clave-valor, en el que una base de datos clave-valor (un tipo simple de base de datos NoSQL) almacena datos como una colección de pares clave-valor, donde cada clave es única;
el columnar, mediante el cual una base de datos columnar almacena datos en columnas en lugar de filas, lo que puede ser más eficiente para ciertas cargas de trabajo analíticas donde se accede a subconjuntos de columnas;
y el de grafos, implementado por una base de datos de grafos que utiliza nodos, ejes y propiedades para representar y almacenar datos, resultando ideal para modelar relaciones complejas entre entidades.
Estos diversos modelos \texttt{NoSQL} suelen priorizar la escalabilidad y la disponibilidad sobre la consistencia estricta (modelo \texttt{BASE}) \parencite{SadalegeFowler2012}.
Podrían ser útiles para almacenar datos menos estructurados o de gran volumen, como logs del sistema, resultados intermedios de la generación de horarios, o quizás para perfiles de usuario o configuraciones flexibles, complementando a la base de datos relacional principal.

\subsection{Tecnologías para Frontend}
Las tecnologías frontend determinan cómo se construye la interfaz de usuario para interactuar con el sistema de gestión del calendario académico.

\subsubsection{TypeScript - React}
\texttt{TypeScript} es un superconjunto de \texttt{JavaScript} que añade tipado estático opcional, mejorando la robustez y mantenibilidad del código frontend, especialmente en proyectos grandes \parencite{MicrosoftTypeScript}. \texttt{React}\footnote{\textbf{React} (también conocido como React.js o ReactJS) es una biblioteca de JavaScript de código abierto para construir interfaces de usuario, especialmente aplicaciones de una sola página (SPA).
Se utiliza para manejar la capa de vista para aplicaciones web y móviles.
Es mantenida por Facebook y una comunidad de desarrolladores individuales y compañías.} es una popular biblioteca de \texttt{JavaScript} para construir interfaces de usuario declarativas y basadas en componentes \parencite{FacebookReact}.
La combinación de \texttt{TypeScript} y \texttt{React} ofrece un entorno de desarrollo productivo y seguro para crear interfaces complejas y reactivas, adecuadas para visualizar horarios, formularios de gestión y resultados de predicciones de manera eficiente.

\subsection{Diagramas}
Los diagramas son herramientas visuales esenciales en la ingeniería de software para comunicar la estructura, el comportamiento y la arquitectura de un sistema de manera clara y concisa.
Utilizar diferentes tipos de diagramas ayuda a comprender y documentar distintos aspectos del sistema web de gestión académica \parencite{Fowler2003}.

\subsubsection{Diagrama de requerimientos}
Los diagramas de requerimientos, como los diagramas de casos de uso (\texttt{UML}), representan las interacciones entre los actores (usuarios como estudiantes, administradores académicos) y el sistema, describiendo las funcionalidades que este debe ofrecer (e.g., "Consultar horario", "Generar propuesta de horario", "Predecir elegibilidad de estudiante", "Administrar cursos") \parencite{Jacobson1992}.
Estos diagramas son fundamentales en las etapas iniciales para definir el alcance del proyecto y asegurar que se comprendan y capturen las necesidades de los usuarios de Jala University.

Systems Modeling Language (SysML) es una extensión de UML para aplicaciones de ingeniería de sistemas que proporciona un diagrama de requisitos que sirve como mecanismo de representación gráfica para capturar requisitos textuales y sus relaciones \parencite{Friedenthal2014}.
Este tipo de diagrama—único de SysML—permite a los ingenieros modelar jerarquías de requisitos, dependencias y métodos de verificación dentro de un marco de modelado unificado.

\subsubsection{Diagrama C4}
El modelo C4 (Context, Containers, Components, Code) proporciona un marco para visualizar la arquitectura de software en diferentes niveles de abstracción, facilitando la comunicación entre distintos roles (desde negocio hasta desarrolladores) \parencite{BrownC4}.
Es especialmente útil para describir sistemas complejos como el propuesto, permitiendo entender cómo encaja en el ecosistema de Jala University y cómo se estructura internamente.

\paragraph{System context}
El diagrama de Contexto (Nivel 1 de C4) muestra el sistema de software en su totalidad como una caja negra, identificando sus interacciones con los usuarios principales (estudiantes, administradores) y otros sistemas externos con los que se integra (e.g., sistema de registro de estudiantes de Jala University, sistema de autenticación).
Este diagrama establece el alcance y los límites del sistema de gestión del calendario académico.

\paragraph{Containers}
El diagrama de Contenedores (Nivel 2 de C4) descompone el sistema en sus principales bloques ejecutables o desplegables, como aplicaciones web, APIs, bases de datos o microservicios (e.g., Web App Frontend, API Gateway, Servicio de Horarios, Servicio de Predicción ML, Base de Datos Académica).
Muestra las responsabilidades de alto nivel de cada contenedor y las tecnologías principales utilizadas, así como las interacciones entre ellos.

\paragraph{Components}
El diagrama de Componentes (Nivel 3 de C4) detalla la estructura interna de un contenedor específico, mostrando los principales componentes (agrupaciones lógicas de código, como clases o módulos) y sus interacciones dentro de ese contenedor.

\paragraph{Code}
El diagrama de Código (Nivel 4 de C4, opcional) ofrece una vista detallada a nivel de clases o entidades específicas dentro de un componente, utilizando notaciones como diagramas de clases \texttt{UML}.
Este nivel es útil para desarrolladores que necesitan entender la implementación detallada de una parte específica del sistema, como las clases que implementan un algoritmo de predicción o las entidades del dominio académico.

\subsubsection{Unified Modeling Language (UML)}
Is a standardized general-purpose modeling language in the field of software engineering.
It is primarily used to visualize, specify, construct, and document the artifacts of software systems, as well as for business modeling and other non-software systems.
UML provides a set of graphical notation techniques to create visual models of software-intensive systems.
It is not a programming language but a visual language for describing software designs and processes.
UML helps teams communicate, explore potential designs, and validate the architectural design of the software.
UML encompasses a wide array of diagram types, including class diagrams, use case diagrams, sequence diagrams, activity diagrams, and state diagrams, each serving a distinct purpose in modeling different aspects of a system \cite{OMG2017, Fowler2003}.

\subsection{Tecnología de diagramado}
Las herramientas de diagramado asistido por software permiten crear y mantener diagramas de arquitectura, diseño y procesos de manera eficiente y consistente, a menudo integrándose con el código o sistemas de control de versiones.

\subsubsection{Structurizr}
Structurizr es un conjunto de herramientas (bibliotecas de código abierto y una plataforma web) para crear diagramas de arquitectura de software basados en el modelo C4, utilizando el enfoque de "diagramas como código", donde los modelos se definen en código (e.g., Java, \texttt{C\#}, Python) y los diagramas se generan a partir de él \parencite{BrownStructurizr}.
Esto asegura que la documentación arquitectónica se mantenga sincronizada con el código y facilita la automatización de la generación de diagramas C4 para el sistema de Jala University.

\subsubsection{PlantUML}
PlantUML es una herramienta de código abierto que permite generar diversos diagramas \texttt{UML} (secuencia, casos de uso, clases, actividad, componentes), así como otros tipos de diagramas (Arquitectura C4, ERD, Wireframe), a partir de una descripción textual simple \parencite{PlantUML}.
Es muy útil para crear rápidamente diagramas técnicos y mantenerlos bajo control de versiones junto con el código fuente, facilitando la documentación visual del diseño del sistema.

\subsubsection{SysML}
La Systems Modeling Language (SysML) es una extensión de UML para aplicaciones de ingeniería de sistemas que proporciona un diagrama de requisitos que sirve como mecanismo de representación gráfica para capturar requisitos textuales y sus relaciones \parencite{Friedenthal2014}.
Este tipo de diagrama—único de SysML—permite a los ingenieros modelar jerarquías de requisitos, dependencias y métodos de verificación dentro de un marco de modelado unificado.

El fundamento teórico de los diagramas de requisitos de SysML puede formalizarse a través de la teoría de grafos y la teoría de sistemas.
Un diagrama de requisitos $G = (V, E)$ puede representarse como un grafo dirigido donde los vértices $V$ representan requisitos individuales, y las aristas $E$ representan relaciones entre requisitos \parencite{Weilkiens2011}.
Estas relaciones incluyen \textit{containment} (descomposición jerárquica), \textit{derive} (inferencia lógica), \textit{satisfy} (implementación), \textit{verify} (pruebas), \textit{refine} (elaboración) y \textit{trace} (dependencia general).

Los requisitos pueden definirse formalmente como tuplas:
$R = \langle id, texto, tipo, fuente, justificación, riesgo, estado, ...\rangle$
La representación estructurada que ofrecen los diagramas de requisitos facilita la descomposición jerárquica de requisitos complejos, el análisis de trazabilidad entre requisitos y otros artefactos del sistema, la planificación de verificación mediante la conexión de requisitos con casos de prueba, y la validación de requisitos a través de representaciones visuales que mejoran los procesos de revisión con las partes interesadas \parencite{INCOSE2015}.
Esta naturaleza dual posiciona a los diagramas de requisitos como herramientas mediadoras fundamentales dentro del proceso de ingeniería de sistemas, cerrando la brecha entre los espacios del problema y la solución.



\subsection{Organizador de tareas}
Las herramientas de gestión de tareas son esenciales para planificar, organizar y seguir el progreso del trabajo en un proyecto de desarrollo de software, especialmente cuando se utilizan metodologías ágiles.

\subsubsection{Clickup}
ClickUp es una plataforma de productividad y gestión de proyectos todo en uno que ofrece múltiples vistas (listas, tableros Kanban, calendarios, Gantt), personalización de flujos de trabajo y funcionalidades para la colaboración en equipo \parencite{ClickUp}.
Podría utilizarse para gestionar el backlog del producto, planificar sprints, asignar tareas y seguir el progreso general del desarrollo del sistema.

\subsubsection{Taskwarrior}
Taskwarrior es una herramienta de gestión de tareas de código abierto y basada en línea de comandos, que permite organizar listas de tareas pendientes de forma eficiente y flexible \parencite{Taskwarrior}.
Es una opción potente para desarrolladores que prefieren trabajar en la terminal, aunque requiere una curva de aprendizaje y es más adecuada para la gestión individual de tareas dentro del proyecto.

\subsubsection{Taiga}
Taiga es una plataforma de gestión de proyectos ágil, de código abierto y centrada en Scrum y Kanban, que ofrece tableros visuales, gestión de backlogs, seguimiento de issues y wikis \parencite{Taiga}.
Representa una alternativa open-source a herramientas como Jira o ClickUp, adecuada para equipos que buscan una solución auto-alojada o gratuita para implementar Scrum o Scrumban en el desarrollo del proyecto.

\subsection{Herramientas de versionamiento}
El control de versiones es indispensable en el desarrollo de software para gestionar los cambios en el código fuente a lo largo del tiempo, facilitar la colaboración entre desarrolladores y permitir la reversión a estados anteriores.

\subsubsection{GitHub}
GitHub es una plataforma de desarrollo colaborativo basada en \texttt{Git}\footnote{\textbf{Git} es un sistema de control de versiones distribuido, de código abierto y gratuito, diseñado para manejar desde proyectos pequeños hasta muy grandes con velocidad y eficiencia.
Fue creado por Linus Torvalds en 2005.} que ofrece hospedaje de repositorios, seguimiento de issues, revisión de código (Pull Requests)\footnote{Un \textbf{Pull Request (PR)} o Solicitud de Integración es una característica de plataformas de control de versiones como GitHub que permite a los desarrolladores proponer cambios al código base.
Facilita la revisión de código y la discusión antes de integrar los cambios en la rama principal.}, integración continua y otras herramientas para el ciclo de vida del desarrollo de software \parencite{GitHub}.
Es la plataforma de facto para muchos proyectos de código abierto y empresariales, y sería una opción robusta para alojar el código del sistema de Jala University, gestionar la colaboración y automatizar flujos de trabajo.

\subsubsection{Sourcehut}
SourceHut es una suite de herramientas de desarrollo de software de código abierto, enfocada en la simplicidad, la estabilidad y la filosofía Unix, que ofrece hospedaje \texttt{Git}, seguimiento de tickets, listas de correo, CI/CD y otras funcionalidades \parencite{SourceHut}.
Representa una alternativa más minimalista y centrada en el desarrollador a plataformas como GitHub o GitLab, atractiva para quienes valoran la transparencia y el control sobre sus herramientas de desarrollo.

\subsubsection{Nomenclatura de ramas, commits y pull requests}
Establecer una convención clara para nombrar ramas (e.g., \texttt{feature/nombre-funcionalidad}, \texttt{bugfix/descripcion-corta}, \texttt{release/v1.0}), escribir mensajes de commit significativos (e.g., siguiendo el formato Conventional Commits \parencite{ConventionalCommits}) y gestionar Pull Requests (PRs) de manera estructurada (con descripciones claras, revisiones obligatorias) es crucial para mantener un historial de cambios limpio, facilitar la revisión de código y mejorar la colaboración dentro del equipo de desarrollo del proyecto.

\subsection{Herramientas de diseño}
Las herramientas de diseño facilitan la creación de prototipos, wireframes y diseños visuales de la interfaz de usuario (UI) y la experiencia de usuario (UX), permitiendo iterar y validar ideas antes de escribir código.

\subsubsection{Figma}
Figma es una herramienta de diseño de interfaces basada en la web y colaborativa, que permite crear prototipos interactivos, sistemas de diseño y colaborar en tiempo real entre diseñadores y desarrolladores \parencite{Figma}.
Es ideal para diseñar las pantallas del sistema de gestión académica, definir flujos de usuario y crear un lenguaje visual consistente para la aplicación de Jala University.


\subsection{Plan de pruebas}
Un plan de pruebas sistemático es esencial para asegurar la calidad, fiabilidad y corrección del sistema web desarrollado, verificando que cumple con los requerimientos funcionales y no funcionales definidos.

\subsubsection{ISO 9126}
La norma ISO/IEC 9126 (reemplazada en parte por ISO/IEC 25010) define un modelo de calidad para el software, clasificando los atributos de calidad en seis características principales: Funcionalidad, Fiabilidad, Usabilidad, Eficiencia, Mantenibilidad y Portabilidad \parencite{ISO9126}.
Utilizar este modelo como marco para el plan de pruebas del sistema de gestión académica permite definir criterios de aceptación claros y métricas específicas para evaluar cada aspecto de la calidad del software, asegurando una cobertura completa y sistemática de las pruebas.


\section{I-star Framework Analysis}
\label{sec:appendixIStarAnalysis}

\subsection{Strategic Dependency}
Figure~\ref{fig:strategicDependenciesDiagram} is a model representing how actors interact with each other.
Table~\ref{tab:strategicDependencies} explains their relationships.
\begin{table}
\centering
\caption{SD relationships description} \label{tab:strategicDependencies}
	\begin{tabularx}{\linewidth}{@{} p{0.8in} p{1.2in} p{1.6in} X @{}}
	\toprule
	\multicolumn{2}{c}{\textbf{Relationship}} & \textbf{Dependum} & \textbf{Description} \\
	\cmidrule(lr){1-2}
	\textbf{Dependant} & \textbf{Dependee} & & \\
	\midrule
	% Student
	\multirow{4}{*}{Student} & Student Services & Enrollment compliance & To ensure appropiate module enrollment. \\
	& Registrar & Enrollment flexibility & LOA and course drop \\
	& Registrar & Enrolled & Effective enrollment to module courses \\
	& Registrar & Module timetable & Courses, time periods, section, groups \\
	\hline
	% Student Services
	Student Services & Registrar & Resolve student enrollment \& timetable issues & Handle a student's enrollment issues\\
	\hline
	% Teacher
	\multirow{3}{*}{Teacher} & Academic Coordinator & Hired & Effective assignation to class \\
	& Registrar & Class schedule flexibility & Move class timeslot \\
	& Registrar & Module timetable & Specific class timetable \\
	\hline
	% Academic Coordinator
	Academic Coordinator & Registrar & Teaching staff to be hired & Course and times \\
	\hline
	% Student Information System
	SIS & Teacher & Student grades & GPA and SAP\\
	\hline
	% Registrar
	\multirow{2}{*}{Registrar} & SIS & Student data & Student personal data and grades\\
	& Academic Coordinator & Hired teachers data & Names, availability and course\\
	\bottomrule
	\end{tabularx}
\end{table}

\begin{figure}
	\caption{Strategic Dependencies Diagram - Fuente: (Elaboracion propia)}\label{fig:strategicDependenciesDiagram}
	\centering
	\includegraphics[width=.70\textwidth]{strategic-dependencies.pdf}
\end{figure}

\subsection{Strategic Rationale}
\subsubsection{Registrar}
Strategic stakeholder concerned with student enrollment, timetable management and minimizing administrative overhead, because it needs to generate other kind of reports outside of the context of this case study, for which the softgoal "Minimize administrative overhead" has been identified.
She also has the softgoal of "Reducing student double booking" i.e. when a student is enrolled on two classes that happen at the same time.

Figure~\ref{fig:actorBoundaryRegistrar} is a representation of Registrar's strategic rationale.

\begin{landscape}
	\begin{figure}
		\centering
		\caption{Registrar strategic rationale model - Fuente: Elaboracion propia}
		\includegraphics[width=\textwidth]{registrar.pdf}
		\label{fig:actorBoundaryRegistrar}
	\end{figure}
\end{landscape}

\subsubsection{Student}
A student is an strategic stakeholder concerned with obtaining required credits.
Figure~\ref{fig:actorBoundaryStudent} describes their strategic rationale.
\begin{figure}
	\centering
	\caption{Student strategic rationale model - Fuente: Elaboracion propia}
	\includegraphics[width=\textwidth]{student.pdf}
	\label{fig:actorBoundaryStudent}
\end{figure}

\subsubsection{Teacher}
As we can see in Figure~\ref{fig:actorBoundaryTeacher} a teacher is interested in becoming \textit{Hired} to be able to teach a class.
Teacher needs \textit{Class schedule flexibility} to request schedule changes.

\begin{figure}
	\centering
	\caption{Teacher strategic rationale model - Fuente: Elaboracion propia}
	\label{fig:actorBoundaryTeacher}
	\includegraphics[width=\textwidth]{teacher.pdf}
\end{figure}


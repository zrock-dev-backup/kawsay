\section{Case study}

In the context of academic timetable generation and student enrollment, six actors have been identified.
\textit{Registrar}, \textit{Student} and \textit{Teacher} are strategic stakeholders because they directly require timetable and enrollment.
\textit{Academic coordinator} \footnote{In charge of hiring teaching staff}, \textit{Student Information System} \footnote{Internal system to handle student data} and \textit{Student Services} are actors whom require Registrar services to function properly.

Figure~\ref{fig:strategicDependenciesDiagram} is a model representing how this six actors interact with each other.
Table~\ref{tab:strategicDependencies} explains each relationship.
\begin{table}
\centering
\caption{SD relationships description} \label{tab:strategicDependencies}
	\begin{tabularx}{\linewidth}{@{} p{0.8in} p{1.2in} p{1.6in} X @{}}
	\toprule
	\multicolumn{2}{c}{\textbf{Relationship}} & \textbf{Dependum} & \textbf{Description} \\
	\cmidrule(lr){1-2}
	\textbf{Dependant} & \textbf{Dependee} & & \\
	\midrule
	% Student
	\multirow{4}{*}{Student} & Student Services & Enrollment compliance & To ensure appropiate module enrollment. \\
	& Registrar & Enrollment flexibility & LOA and course drop \\
	& Registrar & Enrolled & Effective enrollment to module courses \\
	& Registrar & Module timetable & Courses, time periods, section, groups \\
	\hline
	% Student Services
	Student Services & Registrar & Resolve student enrollment \& timetable issues & Handle a student's enrollment issues\\
	\hline
	% Teacher
	\multirow{3}{*}{Teacher} & Academic Coordinator & Hired & Effective assignation to class \\
	& Registrar & Class schedule flexibility & Move class timeslot \\
	& Registrar & Module timetable & Specific class timetable \\
	\hline
	% Academic Coordinator
	Academic Coordinator & Registrar & Teaching staff to be hired & Course and times \\
	\hline
	% Student Information System
	SIS & Teacher & Student grades & GPA and SAP\\
	\hline
	% Registrar
	\multirow{2}{*}{Registrar} & SIS & Student data & Student personal data and grades\\
	& Academic Coordinator & Hired teachers data & Names, availability and course\\
	\bottomrule
	\end{tabularx}
\end{table}

\begin{figure}
	\caption{Strategic Dependencies Diagram - Fuente: (Elaboracion propia)}\label{fig:strategicDependenciesDiagram}
	\centering
	\includegraphics[width=.70\textwidth]{strategic-dependencies.pdf}
\end{figure}

\subsubsection{Registrar}
Registrar is the actor that generates the timetable, for this purpose depends on Academic Coordinator for teaching staff data and SIS for student data.

Registrar is the main actor in the timetable and student enrollment management context.
Figure~\ref{fig:actorBoundaryRegistrar} is a representation of Registrar's strategic rationale.

Registrar office is concerned with student enrollment and timetable management.
Student depends upon Registrar to be enrolled in a course meaning Registrar has the task of enrolling students to class implying the generation of a timetable and planning ahead to have resources available for the next module or term.

This actor is also concerned with minimizing administrative overhead, because it needs to generate other kind of reports outside of the context of this case study, for which the softgoal "Minimize administrative overhead" has been identified.
She also has the softgoal of "Reducing student double booking" i.e. when a student is enrolled on two classes that happen at the same time.

\begin{landscape}
	\begin{figure}
		\centering
		\caption{Registrar strategic rationale model - Fuente: Elaboracion propia}
		\includegraphics[width=\textwidth]{registrar.pdf}
		\label{fig:actorBoundaryRegistrar}
	\end{figure}
\end{landscape}

\subsubsection{Student}
Student needs Registrar to be enrolled in a course, this actor peforms tasks to graduate from university and is interested in protecting their scolarship by following the university's guidelines.
Figure~\ref{fig:actorBoundaryStudent} describes a Student's strategic rationale.
\begin{figure}
	\centering
	\caption{Student strategic rationale model - Fuente: Elaboracion propia}
	\includegraphics[width=\textwidth]{student.pdf}
	\label{fig:actorBoundaryStudent}
\end{figure}

\subsubsection{Teacher}
As we can see in Figure~\ref{fig:actorBoundaryTeacher} a teacher is interested in becoming \textit{Hired} to be able to teach a class.
Teacher needs \textit{Class schedule flexibility} to request schedule changes.

\begin{figure}
	\centering
	\caption{Teacher strategic rationale model - Fuente: Elaboracion propia}
	\label{fig:actorBoundaryTeacher}
	\includegraphics[width=\textwidth]{teacher.pdf}
\end{figure}

\subsubsection{Timetable clashes}
Timetable and student enrollment management is a task that faces two types of clashes.

\paragraph{Student double booking} Produced when a student fails a course and needs to repeated, when enrolling this student to the class sometimes she will also have another class happening in the same hour.
In this situation the registrar tries to find another class for when the student is available.

\paragraph{Teacher change of schedule} Produced when a teacher requests a change of schedule because of an external reason, in this situation the registrar has to find another available timeslot taking into account the class availability and the teacher.
Sometimes when this operation is not possible then the teacher is replaced.

\section{Análisis del Marco de Trabajo i*}
\label{sec:appendixIStarAnalysis}

\subsection*{Diagrama strategico de dependencias}
La Figura~\ref{fig:strategicDependenciesDiagram} es un modelo que representa cómo los actores interactúan entre sí.
La Tabla~\ref{tab:strategicDependencies} explica sus relaciones.
\begin{table}
\centering
\caption{Detalle diagrama SD} \label{tab:strategicDependencies}
	\begin{tabularx}{\linewidth}{@{} p{0.8in} p{1.2in} p{1.6in} X @{}}
	\toprule
	\multicolumn{2}{c}{\textbf{Relación}} & \textbf{Dependum} & \textbf{Descripción} \\
	\cmidrule(lr){1-2}
	\textbf{Dependiente} & \textbf{Dependido} & & \\
	\midrule
	% Student
	\multirow{4}{*}{Student} & Student Services & Cumplimiento de la inscripción & Para asegurar la inscripción apropiada en el módulo. \\
	& Registrar & Flexibilidad de inscripción & Permiso de ausencia (LOA) y baja de curso \\
	& Registrar & Inscrito & Inscripción efectiva en los cursos del módulo \\
	& Registrar & Horario del módulo & Cursos, períodos, sección, grupos \\
	\hline
	% Student Services
	Student Services & Registrar & Resolver problemas de inscripción y horarios de los estudiantes & Gestionar los problemas de inscripción de un estudiante\\
	\hline
	% Teacher
	\multirow{3}{*}{Teacher} & Academic Coordinator & Contratado & Asignación efectiva a la clase \\
	& Registrar & Flexibilidad del horario de clase & Cambiar el horario de la clase \\
	& Registrar & Horario del módulo & Horario específico de la clase \\
	\hline
	% Academic Coordinator
	Academic Coordinator & Registrar & Personal docente a contratar & Curso y horarios \\
	\hline
	% Student Information System
	SIS & Teacher & Calificaciones del estudiante & GPA y SAP\\
	\hline
	% Registrar
	\multirow{2}{*}{Registrar} & SIS & Datos del estudiante & Datos personales y calificaciones del estudiante\\
	& Academic Coordinator & Datos de los profesores contratados & Nombres, disponibilidad y curso\\
	\bottomrule
	\end{tabularx}
\end{table}

\begin{figure}
	\caption{Diagrama de Dependencias Estratégicas}\label{fig:strategicDependenciesDiagram}
	\centering
	\includegraphics[width=.70\textwidth]{strategic-dependencies.pdf}
\end{figure}

\subsection*{Diagrama strategico de razones}
\subsubsection*{Oficina de registros}
Interesado en la inscripción de estudiantes, gestión de horarios y reducción de la sobrecarga administrativa para lo cual se ha identificado el objetivo flexible "Minimizar la sobrecarga administrativa".
Ella también tiene el objetivo flexible de "Reducir la doble inscripción de estudiantes", es decir, cuando un estudiante está inscrito en dos clases que ocurren al mismo tiempo.

La Figura~\ref{fig:actorBoundaryRegistrar} es una representación de la justificación estratégica de la Registrar.

\begin{landscape}
	\begin{figure}
		\centering
		\caption{Modelo de justificación estratégica de la Registrar}
		\includegraphics[width=1.2\textwidth]{registrar.pdf}
		\label{fig:actorBoundaryRegistrar}
	\end{figure}
\end{landscape}

\subsubsection*{Estudiante}
Un estudiante es un actor estratégico interesado en obtener los créditos requeridos.
La Figura~\ref{fig:actorBoundaryStudent} describe su justificación estratégica.
\begin{figure}
	\centering
	\caption{Modelo de justificación estratégica del Estudiante}
	\includegraphics[width=\textwidth]{student.pdf}
	\label{fig:actorBoundaryStudent}
\end{figure}

\subsubsection*{Profesor}
Como podemos ver en la Figura~\ref{fig:actorBoundaryTeacher}, un profesor está interesado en ser \textit{Contratado} para poder impartir una clase.
El profesor necesita \textit{Flexibilidad del horario de clase} para solicitar cambios de horario.

\begin{figure}
	\centering
	\caption{Modelo de justificación estratégica del Profesor}
	\label{fig:actorBoundaryTeacher}
	\includegraphics[width=\textwidth]{teacher.pdf}
\end{figure}

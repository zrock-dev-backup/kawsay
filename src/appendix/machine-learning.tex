\section{Validación del aprendizaje automático en el análisis de elegibilidad de estudiantes}

\subsection*{Objetivo}
El objetivo de esta Prueba de Concepto (PoC) es validar la usabilidad del análisis predictivo, como el aprendizaje automático, para evaluar si un estudiante aprobará un curso o no. La meta es explorar cómo diversos algoritmos de aprendizaje automático pueden ser utilizados para predecir el rendimiento estudiantil basado en datos históricos.

\subsection*{Requisitos}
\begin{itemize}
    \item Datos: Cada módulo debe implementar una función simple de "hello world" y contar con 4 tipos de datos disponibles (byte, entero, flotante y cadena) para representar diferentes atributos del rendimiento estudiantil, incluyendo pero no limitándose a:
    \begin{itemize}
        \item GPA (Promedio de Calificaciones)
        \item SAP (Progreso Académico Satisfactorio)
    \end{itemize}
    \item Algoritmos de Aprendizaje Automático: La PoC debe utilizar una variedad de algoritmos de aprendizaje automático para predecir la elegibilidad de los estudiantes, incluyendo:
    \item Métricas de Evaluación: El rendimiento del modelo será evaluado utilizando métricas estándar de evaluación como:
    \begin{itemize}
        \item Precisión
        \item Precisión (o Exactitud)
    \end{itemize}
    \item Preprocesamiento de Datos: Los datos deben ser limpiados y preprocesados antes del entrenamiento.
    \item Salida: La salida final del modelo será una puntuación de probabilidad que indique si un estudiante es probable que apruebe o no el curso.
\end{itemize}

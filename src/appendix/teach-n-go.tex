\section{Análisis de Teach N' Go}
\label{sec:teachNGo}

En la actualidad, existe Teach 'n Go, un software moderno de gestión escolar que integra funcionalidad de planificación con otras características para gestionar las operaciones escolares.
Teach 'n Go es un LMS \footnote{LMS (Learning Management System)} diseñado para tareas de gestión escolar, incluida la planificación.
\subsection*{Características}
Teach'n Go ofrece una amplia gama de características para la gestión escolar:
\begin{itemize}
    \item \textbf{Codificación por colores de las clases}: Ayuda a organizar visualmente las clases.
    \item \textbf{Gestión del tamaño del aula}: Restringe el número de estudiantes por clase.
    \item \textbf{Inscripción rápida de estudiantes}: Permite la inscripción o des inscripción masiva de estudiantes.
    \item \textbf{Archivado de clases}: Archiva las clases completadas para su registro.
    \item \textbf{Panel de estadísticas rápidas}: Muestra métricas importantes como la asistencia y las calificaciones.
    \item \textbf{Integración con Google Calendar}: Sincroniza los horarios de las clases con Google Calendar para un fácil acceso.
\end{itemize}

El software también permite personalización con la marca de la escuela e integración con herramientas como Google Meet y Microsoft Teams, facilitando una experiencia de planificación sin interrupciones.

\subsection*{Relación entre el Registrador y los Estudiantes}
El sistema facilita una mejor interacción entre el registrador y los estudiantes:
\begin{itemize}
    \item \textbf{Panel de control}: Proporciona una visión general del horario diario, incluidas las asignaciones de profesores y la disponibilidad de aulas.
    \item \textbf{Vista del calendario}: Muestra los cursos por profesor, aula u otros criterios, permitiendo a los usuarios crear nuevas reuniones o imprimir el calendario.
    \item \textbf{Manejo de errores}: Alerta a los usuarios cuando surgen conflictos de planificación, como cuando un aula está doblemente reservada.
\end{itemize}

Teach 'n Go permite a los usuarios configurar horarios de clases recurrentes, ya sea semanalmente o en fechas personalizadas, con campos personalizables para rastrear datos específicos como el tema de la clase, nivel y códigos de curso.

Teach 'n Go se integra con la API abierta de HostHub para comunicarse con otros sistemas, lo que permite usarlo en un ecosistema de software relacionado.


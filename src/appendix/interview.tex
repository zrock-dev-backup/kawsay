\section{Entrevistas}
\label{sec:appendixInterviews}

Se entrevistó al Registrador, un actor estratégico clave.
Su selección se basó en su participación directa en el proceso de generación de horarios y su conocimiento de los procedimientos de inscripción de estudiantes.

Un cuestionario de entrevista Tabla~\ref{tab:questionaree} sirvió como guía durante las entrevistas, asegurando que se abordaran los temas clave.

El cuestionario contiene principalmente preguntas abiertas diseñadas para obtener narrativas detalladas y perspectivas sobre los desafíos y procesos relacionados con la generación de horarios.
Las preguntas se derivaron directamente de la pregunta de investigación descrita en la Sección~\ref{sec:problemDefinition}, asegurando la alineación entre la recopilación de datos y los objetivos en la Sección~\ref{sec:objectives}.
Las entrevistas duraron aproximadamente [45 min] y fueron grabadas en video con el consentimiento informado de los participantes.

\begin{table}[h]
    \caption{Cuestionario}\label{tab:questionaree}
    \begin{tabularx}{\textwidth}{lX}
        \toprule
        \textbf{N.º} &  \textbf{Pregunta}\\
        \midrule
        1. & ¿Cómo se divide el alumnado entre las clases? \\
        2. & ¿Es la oficina del Registrador la única involucrada en el proceso de generación de horarios?\\
        3. & ¿Cómo se reincorpora un estudiante en licencia (LOA)?\\
        4. & ¿Por qué los datos están dispersos en diferentes hojas de cálculo?\\
        5. & ¿En qué situaciones ocurre un conflicto/choque (clash)?\\
        6. & ¿Cuáles son los desafíos de la generación de horarios?\\
        7. & ¿Cuáles son los artefactos que crea el Registrador en el contexto de los horarios y la inscripción de estudiantes?\\
        8. & ¿Cuáles son los beneficios especificados del sistema actual?\\
        9. & ¿Qué sucede si una estudiante no es elegible pero luego aprueba?\\
        10. & ¿Cuál es la cantidad de tiempo requerida para generar un horario?\\
        11. & ¿Cómo afecta la generación de horarios a la carga administrativa?\\
        \bottomrule
    \end{tabularx}
\end{table}

\subsection*{Análisis}

Además de las entrevistas, se analizaron documentos relevantes para proporcionar información contextual y triangular los hallazgos de las entrevistas.
Los documentos incluyeron: [Catálogo del Estudiante de la Universidad Jala].
El análisis de la documentación existente proporcionó una perspectiva complementaria sobre los procesos y procedimientos formales, permitiendo la comparación con las experiencias vividas descritas por los participantes de la entrevista.

El análisis de documentos se realizó simultáneamente con el análisis de las entrevistas, y los temas clave que surgieron de las entrevistas se compararon con la información contenida en los documentos.

Estas preguntas se utilizaron para elaborar el diagrama i* (Apéndice~\ref{sec:appendixIStarAnalysis}).

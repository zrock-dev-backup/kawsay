\section{Interviews}
\label{sec:appendixInterviews}

The key strategic stake holder Registrar was interviewed.
Selected based on their direct involvement in the timetable generation process and their knowledge of student enrollment procedures.

An interview questionnaire Table~\ref{tab:questionaree} served as a guide during the interviews, ensuring that key topics were addressed.

The questionnaire has primarily open-ended questions designed to elicit detailed narratives and perspectives on the challenges and processes related to timetable generation.
The questions were directly derived from the research question outlined in Section~\ref{sec:problemDefinition}, ensuring alignment between data collection and objectives in section Section~\ref{sec:objectives}.
Interviews lasted approximately [45mins] and were video with the participants' informed consent.

\begin{table}[h]
    \caption{Questionaree}\label{tab:questionaree}
    \begin{tabularx}{\textwidth}{lX}
        \toprule
        \textbf{\#} &  \textbf{Question}\\
        \midrule
        1. & How is the student body divided among lectures? \\
        2. & Is the registrar office the only one involved in the timetable generation process?\\
        3. & How is a LOA student reincorporated?\\
        4. & Why is data scattered among different spreadsheets?\\
        5. & Under which situations does a clash happens?\\
        6. & What are the timetable generation challenges?\\
        7. & What are the artifacts that registrar creates in the context of timetable and student enrollment?\\
        8. & What are the specified benefits of the current system?\\
        9. & What happens if a student is non-elegible but then she passes?\\
        10. & What's the amount of time required to generate a timetable?\\
        11. & How does timetable generation affects to administrative overhead?\\
        \bottomrule
    \end{tabularx}
\end{table}

\subsection*{Analysis}

In addition to interviews, relevant documents were analyzed to provide contextual information and triangulate interview findings.
Documents included: [Jala U Student Catalog].
Analyzing existing documentation provided a complementary perspective on the formal processes and procedures, allowing for comparison with the lived experiences described by interview participants.

The document analysis was conducted concurrently with the interview analysis, and key themes emerging from the interviews were compared with information contained in the documents.

These questions where used to elaborate the i-star diagram (Appendix~\ref{sec:appendixIStarAnalysis}).

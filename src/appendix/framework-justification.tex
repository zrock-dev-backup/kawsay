\section{Justificación de la metodología de trabajo}
\label{sec:methodology-justification}

El desarrollo del proyecto tiene las siguientes necesidades:

\begin{itemize}
    \item Plazos de entrega claros y el proyecto es de bajo costo.
    \item Durante el desarrollo de una sección del proyecto podrían surgir cambios en requisitos previamente acordados.
    Basándose en la madurez del sistema, los requisitos serán más estables.
    \item Se requiere que el usuario tenga una percepción real del software y permita una mejor comprensión de los requisitos del sistema.
    \item El software no es crítico, dado que la vida de las personas no depende del sistema.
    \item El proyecto tiene una complejidad media/baja.
    \item El tamaño del equipo de desarrollo es de 1 persona.
\end{itemize}

\begin{figure}
    \centering
    \caption{Comparación de RAD y RUD}
    \includegraphics[width=\textwidth]{comparison-rad-rud-xp.pdf}
    \label{fig:comparison-rad-rud-xp}

    \vspace{0.5em}
    \begin{minipage}{\textwidth}
        \small\textit{Nota.} Fuente: \textcite{geambasu2011influence}.
    \end{minipage}
\end{figure}

Basándose en este análisis, la metodología elegida será Programación Extrema (eXtreme Programming) porque las necesidades del proyecto se ajustan a esta metodología, y por su enfoque en las pruebas y los estándares de codificación.

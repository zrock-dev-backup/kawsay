\section{Work methodology justification}
\label{sec:methodology-justification}

The development of the project has the following needs:

\begin{itemize}
    \item Clear deadlines and the project is low cost.
    \item During the development of a a section of the project there could be changes to previously agreed requirements.
    Based on the maturity of the system, requirements will be more stable.
    \item It is required fot the  user to have a real sense of the software and allow a better understanding of system requirements.
    \item Software is not critical given that people's life don't depend on the system.
    \item The project has a medium/low complexity.
    \item The size of the development team is 1 person.
\end{itemize}

\begin{figure}
    \centering
    \caption{Comparison of RAD and RUD}
    \includegraphics[width=\textwidth]{comparison-rad-rud-xp.pdf}
    \label{fig:comparison-rad-rud-xp}

    \vspace{0.5em}
    \begin{minipage}{\textwidth}
        \small\textit{Note.} Fuente: \textcite{geambasu2011influence}.
    \end{minipage}
\end{figure}

Based on this analysis the chosen methodology will be eXtreme Programming because because the needs of the project fit in this methodology, and its approach to testing and coding standards.


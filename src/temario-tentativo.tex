\section{Temario tentativo}

CAPÍTULO 1. ESTRUCTURA PROPOSITIVA \\
1.1. INTRODUCCIÓN \\
1.2. ANTECEDENTES \\
1.3. PLANTEAMIENTO DEL PROBLEMA \\
1.3.1 Formulación del problema \\
1.4. OBJETIVOS \\
1.4.1 Objetivo general\\
1.4.2 Objetivos especificos \\
1.4.3 Objetivos y acciones \\
1.5. JUSTIFICACIÓN \\
1.5.1 Justificación tecnica\\
1.5.2 Justificación social \\
1.5.2 Justificación económica \\
1.6. ALCANCES Y LIMITACIONES \\
1.7. MARCO TEÓRICO \\
1.7.1 Metodología de trabajo \\
1.7.2 Herramientas de seguimiento  \\
1.7.3 Lenguajes de programacion  \\
1.7.3.1 Python  \\
1.7.3.2 C# \\
1.7.3.3 Typescript - React \\
1.7.4 Herramientas de prototipado  \\
1.8 MARCO PRACTICO \\
1.8.1 Sprint 0: Analisis, planeacion y diseño \\
1.8.2 Sprint 1: Student management, SIS integration \\
1.8.3 Sprint 2: Student management \\
1.8.4 Sprint 3: Teaching staff management \\
1.8.5 Sprint 4: Courses Management \\
1.8.6 Sprint 5: Schedule creation \\
1.8.7 Sprint 6: Schedule migration \\
1.8.8 Sprint 7: Evaluation \\
1.8.9 Sprint 8: Reports \\
1.8.10 Sprint 9: QA and bug fixing \\
1.9 TEMARIO TENTATIVO \\
1.10. BIBLIOGRAFÍA \\
1.11. CRONOGRAMA \\


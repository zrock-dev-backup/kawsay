\section{Justificación}

\subsection{Justificación Tecnológica}
% Course eligibility analysis
The Registrar's office evaluates a student's chances of passing the course based on GPA and SAP, so they can assess the size of a future class. This human interpretation of data is time-consuming and error-prone, and it depends on the talent of the person as to how much they improve in predicting based on mistakes.

A learning algorithm, from the field of machine learning, can be trained to predict whether a student will pass the course or not, and at the same time, learn from its mistakes, getting better and better over time. This technology reduces the amount of time required to assess how many students could be enrolled in a future class and increases the accuracy during each iteration.

\subsection{Justificación Social}

Currently, the Registrar's Office wastes time on digitalizable manual processes such as: centralization of data, student eligibility analysis, and production of reports. These processes take a lot of time because of the size of Jala University. All the dedicated time spent on such tasks will be automated by the machine.

The implementation of the proposed system means that the Registrar's Office will have all of its business rules related to building the academic schedule automated by the system, effectively reducing common errors and automating this time-consuming administrative task.

\subsection{Justificación Económica}
Scheduling systems are expensive because, most of the time, they represent a subsystem of a larger software. For instance, Teach 'n Go, as an LMS, also offers scheduling features among a range of others. Jala University already has an LMS but requires a scheduling system that does more than just serve as a calendar of activities.

Jala University's student body and teaching staff are constantly growing. Many applications, like Teach 'n Go, have a subscription-based revenue stream. For example, Teach 'n Go charges 239 USD monthly and offers support for only 1,000 students. Other well-established software solutions, like Class365, charge 1,000 USD monthly, offering the core modules for 751–1,000 students. For those prices, a scheduling system becomes expensive, and even more so if its features don't solve your business needs.

\subsection{Iteration 2}

\subsubsection{Initial approach}
The initial approach for this iteration was the development of basic features such as schedule generation and handling class timetable migration.  
The iteration backlog contained the following user stories:

\begin{table}[ht]
\centering
\caption{Iteration 2 Backlog}
\label{tab:iteration-2-backlog}
\begin{tabular}{l p{6cm} p{6cm} c}
\hline
\textbf{\#} &
\textbf{User story} &
\textbf{Description} &
\textbf{Score} \\
\hline
    7 & 
    Academic track creation &
    As a registrar, I want to create a track so that students can be managed under it.
    Track creation involves selecting cohorts; cohorts involve selecting groups; groups are composed of sections. &
    0 \\

    8 &
    Class timetable pre-population &
    When opening a module, it should come pre-populated with classes.
    Given that a module has a defined set of courses, the system should generate a timetable for a specific cohort. &
    0 \\

    6 &
    Class timeslot migration &
    As a registrar, I want to change the timeslot of a class so that I can handle the teacher’s request. &
    0 \\
\hline
\end{tabular}
\end{table}

\subsubsection{Iteration execution}
During iteration execution, one important challenge was the initial setup of the repository, along with writing the theoretical framework for this case-study research, resulting in less time for developing the user stories.
To adjust for these challenges, the researcher reduced quality in order to increase velocity.
Being a solo project, it required much more effort to apply the methodology.

Because of these challenges, acceptance tests and unit testing were not done. Additionally, user stories were not assigned velocity points.

\subsubsection{Retrospective}
Action items:
\begin{itemize}
    \item The researcher needs to read and understand better how to perform eXtreme Programming.
    \item Either take fewer user stories or invest more time in the project to include acceptance tests, quality assurance, and unit tests.
\end{itemize}

\subsubsection{Lessons learned}
Implementing “class timetable pre-population” (user story \#8) with a direct mapping approach was too simple and did not take into account the problem of distributing teaching staff among classes given their availabilities.
Therefore, it is necessary to research a new algorithm for timetable generation.

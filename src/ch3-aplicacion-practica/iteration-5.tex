\subsection{Iteración 5: Refactorización Arquitectónica y Estabilización}

\subsubsection{Descripción General del Sprint}

\textbf{Objetivo del Sprint}: Refactorizar la arquitectura del backend para implementar una separación clara de responsabilidades (Clean Architecture) y estabilizar el sistema resolviendo bugs críticos relacionados con la persistencia de datos y la lógica de negocio.

\textbf{Duración}: 10 de Mayo, 2025 - 24 de Mayo, 2025 (Aprox. 11 días hábiles)

\textbf{Definición de Terminado (Definition of Done)}: 
\begin{itemize}
    \item Pruebas unitarias y de integración actualizadas para reflejar la nueva arquitectura.
    \item Código revisado y refactorizado, eliminando el acceso directo al DbContext desde los controladores.
    \item Criterios de aceptación validados para los flujos de creación de clases.
    \item Documentación de la arquitectura (diagramas C4, SD) actualizada.
\end{itemize}

\subsubsection{Backlog del Sprint}

\begin{table}[H]
\caption{Backlog de la Iteración 5}
\label{tab:iteration-5-backlog}
\begin{tabularx}{\textwidth}{@{}llXrr@{}}
\toprule
\textbf{\#} & \textbf{Historia de Usuario} & \textbf{Descripción de la Tarea} & \textbf{Est. (min)} & \textbf{Real (min)} \\
\midrule
    1 & Refactorización & Refactorizar para añadir separaciones claras (DTO, DAO, Dominio). & 45 & 401 \\
    2 & Bug Fix & Analizar y arreglar problema al guardar clases planificadas en la BD. & 45 & 139 \\
    3 & Bug Fix & Arreglar bug en el frontend para la creación de clases (requiere frecuencia). & 45 & 248 \\
    4 & Documentación & Elaborar diagrama de Dependencias Estratégicas (SD). & 30 & 40 \\
    5 & Documentación & Documentar el flujo de trabajo del actor "Registrar". & 30 & 282 \\
\bottomrule
\end{tabularx}
\end{table}

\subsubsection{Análisis Técnico}

\textbf{Decisiones de Arquitectura y Diseño}:
\begin{itemize}
    \item \textbf{Adopción de Clean Architecture}: La decisión más importante del sprint fue refactorizar el backend para seguir los principios de Clean Architecture, introduciendo capas de `Api`, `Application`, `Domain`, e `Infrastructure`. Esto se evidencia en la tarea `b7e7f193` y múltiples commits que mueven entidades y lógica a sus respectivas capas. El objetivo era mejorar la mantenibilidad, testabilidad y desacoplamiento.
    \item \textbf{Introducción del Patrón Repositorio y Servicio}: Se eliminó el acceso directo al `DbContext` desde los controladores de la API, introduciendo una capa de servicios y repositorios para abstraer la lógica de acceso a datos.
\end{itemize}

\begin{table}[H]
    \caption{Trazabilidad de Decisiones de Arquitectura a Commits Relevantes}
    \label{tab:sprint-5-commit-traceability}
    \begin{tabularx}{\textwidth}{@{}lXl@{}}
        \toprule
        \textbf{Decisión de Arquitectura} & \textbf{Descripción del Commit} & \textbf{Hash del Commit} \\
        \midrule
        Separación de Capas & \texttt{refactor(all): separation of concerns} & \texttt{446b6c58...} \\
        Creación de Proyectos & \texttt{chore(all): create domain,infra and app projects} & \texttt{9d0358c1...} \\
        Migración de Entidades & \texttt{refactor(all): move Entities to Domain} & \texttt{bf0df724...} \\
        Uso del Patrón Repositorio & \texttt{refactor(Courses): update Controller to use repository} & \texttt{c0274ba1...} \\
        \bottomrule
    \end{tabularx}
\end{table}

\subsubsection{Análisis de la Ejecución del Sprint}

\textbf{Impedimentos y Resoluciones}:
\begin{itemize}
    \item \textbf{Impedimento}: La tarea de refactorización arquitectónica (`b7e7f193`) superó masivamente la estimación (45 min vs. 401 min), indicando una subestimación severa de la complejidad y el alcance del cambio.
    \item \textit{Resolución}: El equipo absorbió el impacto, reconociendo la importancia estratégica de la refactorización. Esto llevó a una recalibración del cronograma (`b499bd35`).
\end{itemize}

\begin{table}[H]
    \caption{Análisis de Precisión en la Estimación del Sprint 5}
    \label{tab:sprint-5-estimation-accuracy}
    \begin{tabularx}{\textwidth}{@{}Xrrr@{}}
        \toprule
        \textbf{Tarea} & \textbf{Estimado (min)} & \textbf{Real (min)} & \textbf{Varianza (\%)} \\
        \midrule
        Refactorización de arquitectura (DTO, DAO, Dominio) & 45 & 401 & +791\% \\
        Arreglar persistencia de clases en BD & 45 & 139 & +209\% \\
        Arreglar bug de creación de clases en frontend & 45 & 248 & +451\% \\
        Documentar flujo de trabajo del "Registrar" & 30 & 282 & +840\% \\
        \bottomrule
    \end{tabularx}
\end{table}

\begin{table}[H]
    \caption{Distribución del Esfuerzo por Temática en el Sprint 5}
    \label{tab:sprint-5-effort-distribution}
    \begin{tabularx}{\textwidth}{@{}Xrr@{}}
        \toprule
        \textbf{Temática (Tag)} & \textbf{Tiempo Total (min)} & \textbf{Porcentaje del Esfuerzo} \\
        \midrule
        Refactorización y Deuda Técnica & 835 & $\sim$54\% \\
        Documentación (marco\_teorico) & 403 & $\sim$26\% \\
        Análisis de Requisitos (i\_star) & 181 & $\sim$12\% \\
        Otros (QA, gestión) & 122 & $\sim$8\% \\
        \midrule
        \textbf{Total} & \textbf{1541} & \textbf{100\%} \\
        \bottomrule
    \end{tabularx}
\end{table}

\subsubsection{Retrospectiva del Sprint}

\textbf{Qué Salió Bien} (Mantener):
\begin{itemize}
    \item Se tomó la decisión valiente y correcta de invertir en una refactorización profunda, lo que mejorará la calidad del producto a largo plazo.
    \item La documentación del dominio del problema (i-star, flujos de trabajo) continuó madurando, proporcionando una base sólida para el desarrollo futuro.
\end{itemize}

\textbf{Qué No Salió Bien} (Problemas):
\begin{itemize}
    \item Las estimaciones para tareas de refactorización y documentación fueron extremadamente inexactas, lo que indica una brecha en la comprensión de la complejidad de estas actividades.
\end{itemize}

\textbf{Mejoras de Proceso} (Probar):
\begin{itemize}
    \item \textbf{Acción}: Para tareas de refactorización grandes, realizar un spike de investigación de 1-2 horas para descomponer la tarea en subtareas más pequeñas y estimables antes de comprometerse con el trabajo completo. \textit{Criterio de éxito}: Reducir la varianza de estimación para tareas de refactorización a menos del 100\% en el próximo sprint.
\end{itemize}

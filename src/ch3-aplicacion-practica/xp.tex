\subsection{Eleccion de metodología agil}

As stated on the Appendix~\ref{sec:methodology-justification} the chosen methodologies is XP.

\begin{figure}
    \includegraphics[width=0.45\textwidth]{xp-workflow.png}
    \caption{Fuente \textcite{abrahamsson2017agile}}\label{fig:xp-workflow}
\end{figure}

\paragraph{Adaption of methodology} XP will be the main development methodology and will be supported by Scrum's Sprint, Sprint Review and Sprint planning.
The idea of the adoption is to have the testing approach and coding standards of extreme programming while working in timeboxes with defined workloads.


\subsection{Roles y responsabilidades}

\paragraph{Programmer, Tester and Tracker, Manager} Tesista will asume those roles.

\paragraph{Consultant, Coach} Tutor will act as tehcnical reference regarding machine learning development.

\paragraph{Customer} The customer is the Registrar Office.

\subsection{Process workflow}

\begin{figure}
    \includegraphics[width=0.45\textwidth]{xp-project.png}
    \caption{Fuente \textcite{XpWeb}}\label{fig:xp-project}
\end{figure}

\paragraph{Release planning} Registrar office will pick a week's worth of user stories, based on velocity.

\paragraph{Stories board} The stories board will have the following columns:
\begin{itemize}
    \item Product Backlog, list of stories of the whole project; is infinite and ever growing.
    \item Iteration Backlog, list of stories to do during the iteration.
    \item Doing, stories under active development.
    \item Testing, stories on the acceptance testing.
    \item Done, stories deployed for the small release.
\end{itemize}

\begin{figure}
    \includegraphics[width=0.45\textwidth]{board-columns.png}
    \caption{Columnas, Fuente: Elaboracion propia}\label{fig:board-columns}
\end{figure}

As for the programming tasks those would be ordered based on the programmer's criteria.

\paragraph{Iteration}

\begin{figure}
    \includegraphics[width=0.45\textwidth]{xp-iteration.png}
    \caption{Fuente \textcite{XpWeb}}\label{fig:xp-iteration}
\end{figure}

\begin{itemize}
    \item Iteration planning, pick stories and produce programming tasks.
    \item Development, first unit tests and then coding.
\end{itemize}

During each iteration Tracker would produce the velocity metric.

\paragraph{Acceptance tests} Produced from user stories.
The customer will specify scenarios to test. This step produces bug-fixes and refactoring tasks to be taken into account for the next iteration

\subsection{Herramientas}
\subsubsection{Gestion del proyecto}
Tools to be used in order to manage user requirements and programming tasks.

\paragraph{Taiga} Will be used to manage user requirements.

\paragraph{TaskWarrior} Will be used to manage programming tasks, given that programming tasks are technical removing programming tasks from the user requirements view will declutter this view making it less technical for the client.

\subsubsection{Control de versiones}

The selected versioning tool is Git.

\paragraph{Branching model} The GitFlow branching model will be used.

\paragraph{GitHub} will be the remote codebase. This platform will also have CI/CD implementations.

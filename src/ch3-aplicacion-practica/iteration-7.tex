\subsection{Iteración 7: Motor de Planificación y Despliegue}

\subsubsection{Descripción General del Sprint}

\textbf{Objetivo del Sprint}: Implementar una nueva versión del motor de planificación de horarios, desarrollar la infraestructura para el despliegue continuo y construir las funcionalidades de gestión de la estructura académica y matriculación de estudiantes.

\textbf{Duración}: 4 de Junio, 2025 - 20 de Junio, 2025 (Aprox. 13 días hábiles)

\textbf{Definición de Terminado (Definition of Done)}: 
\begin{itemize}
    \item El nuevo motor de planificación está integrado y es funcional.
    \item La aplicación (frontend y backend) está desplegada en un entorno de pruebas a través de contenedores Docker.
    \item Los endpoints para la gestión de la estructura académica (cohortes, grupos, secciones) están implementados y probados.
    \item El flujo de matriculación de estudiantes está implementado y validado con pruebas E2E.
\end{itemize}

\subsubsection{Backlog del Sprint}

\begin{table}[H]
\caption{Backlog de la Iteración 7}
\label{tab:iteration-7-backlog}
\begin{tabularx}{\textwidth}{@{}llXrr@{}}
\toprule
\textbf{\#} & \textbf{Historia de Usuario} & \textbf{Descripción de la Tarea} & \textbf{Est. (min)} & \textbf{Real (min)} \\
\midrule
    1 & Despliegue & Desplegar Kawsay en el entorno de pruebas. & 45 & 610 \\
    2 & Motor de Planificación & Implementación del motor de planificación de horarios asistido en vivo. & 120 & 782 \\
    3 & Estructura Académica & Refactorización del backend para adaptarse a 'cohorte, grupo, sección'. & 120 & 112 \\
    4 & Matriculación & Implementación de la matriculación de estudiantes. & 300 & 304 \\
    5 & Integración ML & Serialización del modelo ONNX para PoC de ML. & 30 & 47 \\
\bottomrule
\end{tabularx}
\end{table}

\subsubsection{Análisis Técnico}

\textbf{Decisiones de Arquitectura y Diseño}:
\begin{itemize}
    \item \textbf{Reemplazo del Algoritmo de Planificación}: Se tomó la decisión estratégica de depreciar el algoritmo heurístico inicial (Yule) y reemplazarlo por un motor de planificación más robusto y preparado para el futuro. Esto se refleja en los commits `5a8ca55a` ("feat(scheduling): yule algorithm deprecation") y `8a0b92c6` ("feat(scheduling): implement schedule engine").
    \item \textbf{Contenerización para Despliegue}: Se adoptó Docker como estándar para el empaquetado y despliegue de la aplicación, utilizando Nix Flakes para asegurar compilaciones reproducibles. Esto se evidencia en la tarea `7e78d0b6` y los commits del frontend relacionados con Nix y Docker.
    \item \textbf{Modelo de Dominio para Estructura Académica}: Se implementó un modelo de dominio explícito para la estructura académica (cohorte, grupo, sección), lo que permite una gestión más granular y realista del alumnado.
\end{itemize}

\begin{table}[H]
    \caption{Trazabilidad de Decisiones de Arquitectura a Commits Relevantes}
    \label{tab:sprint-7-commit-traceability}
    \begin{tabularx}{\textwidth}{@{}lXl@{}}
        \toprule
        \textbf{Decisión de Arquitectura} & \textbf{Descripción del Commit} & \textbf{Hash del Commit} \\
        \midrule
        Nuevo Motor de Planificación & \texttt{feat(scheduling): implement schedule engine} & \texttt{8a0b92c6...} \\
        Modelo de Estructura Académica & \texttt{feat(academic structure): endpoints implementation} & \texttt{3f609a36...} \\
        Flujo de Matriculación & \texttt{feat(enrollment): implement student enrollment} & \texttt{f61440ca...} \\
        Infraestructura de Despliegue & \texttt{feat(nix): write nix deployment file} & \texttt{cd36a415...} \\
        Serialización de Modelo ML & \texttt{feat(all): append behavior tests (prediction-service)} & \texttt{8188f7c9...} \\
        \bottomrule
    \end{tabularx}
\end{table}

\subsubsection{Análisis de la Ejecución del Sprint}

\textbf{Impedimentos y Resoluciones}:
\begin{itemize}
    \item \textbf{Impedimento}: La tarea de despliegue (`7e78d0b6`) fue significativamente más compleja de lo estimado (45 min vs. 610 min) debido a problemas con las dependencias de compilación en el entorno de CI/CD y la configuración de la red.
    \item \textit{Resolución}: Se requirió una investigación profunda y múltiples iteraciones sobre los scripts de Nix y Docker para lograr una compilación y despliegue exitosos, lo que consumió una gran parte del tiempo del sprint.
\end{itemize}

\begin{table}[H]
    \caption{Análisis de Precisión en la Estimación del Sprint 7}
    \label{tab:sprint-7-estimation-accuracy}
    \begin{tabularx}{\textwidth}{@{}Xrrr@{}}
        \toprule
        \textbf{Tarea} & \textbf{Estimado (min)} & \textbf{Real (min)} & \textbf{Varianza (\%)} \\
        \midrule
        Desplegar Kawsay en entorno de pruebas & 45 & 610 & +1255\% \\
        Implementación del motor de planificación & 120 & 782 & +551\% \\
        Implementación de matriculación de estudiantes & 300 & 304 & +1.3\% \\
        Refactorización del backend para estructura académica & 120 & 112 & -6.7\% \\
        \bottomrule
    \end{tabularx}
\end{table}

\begin{table}[H]
    \caption{Distribución del Esfuerzo por Temática en el Sprint 7}
    \label{tab:sprint-7-effort-distribution}
    \begin{tabularx}{\textwidth}{@{}Xrr@{}}
        \toprule
        \textbf{Temática (Tag)} & \textbf{Tiempo Total (min)} & \textbf{Porcentaje del Esfuerzo} \\
        \midrule
        Desarrollo de Backend (refactor, engine) & 1002 & $\sim$44\% \\
        Despliegue e Infraestructura & 610 & $\sim$27\% \\
        Desarrollo de Frontend & 304 & $\sim$13\% \\
        Integración y Pruebas de ML & 129 & $\sim$6\% \\
        Documentación y Análisis & 245 & $\sim$10\% \\
        \midrule
        \textbf{Total} & \textbf{2290} & \textbf{100\%} \\
        \bottomrule
    \end{tabularx}
\end{table}

\subsubsection{Retrospectiva del Sprint}

\textbf{Qué Salió Bien} (Mantener):
\begin{itemize}
    \item Se lograron avances masivos en funcionalidades críticas (motor de planificación, matriculación, despliegue), demostrando una alta capacidad de ejecución.
    \item La implementación de la estructura académica en el dominio del backend proporcionó un modelo de datos mucho más robusto.
\end{itemize}

\textbf{Qué No Salió Bien} (Problemas):
\begin{itemize}
    \item La complejidad de las tareas de infraestructura y del nuevo motor de planificación fue severamente subestimada, lo que generó una gran presión y horas extra.
\end{itemize}

\textbf{Mejoras de Proceso} (Probar):
\begin{itemize}
    \item \textbf{Acción}: Asignar un "presupuesto de riesgo" de tiempo (ej. 20\% adicional) a tareas que involucren infraestructura de despliegue o la reescritura de componentes centrales. \textit{Criterio de éxito}: Mantener la varianza de estimación para tareas de infraestructura por debajo del 200\%.
\end{itemize}

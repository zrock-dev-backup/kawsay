\subsection{Exploration phase}
The exploration phase involved requirement gathering, getting familiar with C\# and React technologies and desigining the architecture.

\subsubsection{Requirement gathering}
\paragraph{Inteview} Through constant meetings with the client and by asking different questions based on the case study objectives it was possible to obtain important bussines data.

Table~\ref{tab:questionaree} is the questionaree formulated based on research questions and objectives.
\begin{table}[h]
    \caption{Questionaree}\label{tab:questionaree}
    \begin{tabularx}{\textwidth}{lX}
        \toprule
        \textbf{\#} &  \textbf{Question}\\
        \midrule
        1. & How is the student body divided among lectures? \\
        2. & Is the registrar office the only one involved in the timetable generation process?\\
        3. & How is a LOA student reincorporated?\\
        4. & Why is data scattered among different spreadsheets?\\
        5. & Under which situations does a clash happens?\\
        6. & What are the timetable generation challenges?\\
        7. & What are the artifacts that registrar creates in the context of timetable and student enrollment?\\
        8. & What are the specified benefits of the current system?\\
        9. & What happens if a student is non-elegible but then she passes?\\
        10. & What's the amount of time required to generate a timetable?\\
        11. & How does timetable generation affects to administrative overhead?\\
        \bottomrule
    \end{tabularx}
\end{table}

The followimg modeling techniques where used to represent system requirements. This tools helped understand the context of requirements and their relationships among themselves.
\paragraph{Strategic Dependency diagram} Used to model processes.
\paragraph{SysML requireents diagram} Used to represent requirements grouping them into requirement packages.
\paragraph{Story mapping} Used to represent scenarios.
\begin{figure}[H]
    \centering
    \includegraphics[width=0.8\textwidth]{story-map_1.png}
    \caption{Story map 1 - Fuente: Elaboración propia}
    \label{fig:story_map_1}
\end{figure}
\begin{figure}[H]
    \centering
    \includegraphics[width=0.8\textwidth]{story-map_2.png}
    \caption{Story map 2 - Fuente: Elaboración propia}
    \label{fig:story_map_2}
\end{figure}
\begin{figure}[H]
    \centering
    \includegraphics[width=0.8\textwidth]{story-map_3.png}
    \caption{Story map 3 - Fuente: Elaboración propia}
    \label{fig:story_map_3}
\end{figure}

% \subsection{Requirement gathering}
% Gathered requirements obtained through meetings with client are displayed in following SySML requirement diagrams.
%
% \begin{figure}
%     \includegraphics[width=0.45\textwidth]{req-academic.png}
%     \caption{diagrama de requerimientos - fuente: elaboracion propia}\label{fig:req-academic}
% \end{figure}
%
% \begin{figure}
%     \includegraphics[width=0.45\textwidth]{req-class.png}
%     \caption{diagrama de requerimientos - fuente: elaboracion propia}\label{fig:req-class}
% \end{figure}
% \begin{figure}[htbp]
%     \includegraphics[width=0.45\textwidth]{req-course.png}
%     \caption{diagrama de requerimientos - fuente: elaboracion propia}\label{fig:req-course}
% \end{figure}
% \begin{figure}
%     \includegraphics[width=0.45\textwidth]{req-enrollment-analysis.png}
%     \caption{diagrama de requerimientos - fuente: elaboracion propia}\label{fig:req-enrollment-analysis}
% \end{figure}
% \begin{figure}
%     \includegraphics[width=0.45\textwidth]{req-enrollment.png}
%     \caption{diagrama de requerimientos - fuente: elaboracion propia}\label{fig:req-enrollment}
% \end{figure}
% \begin{figure}
%     \includegraphics[width=0.45\textwidth]{req-reports.png}
%     \caption{diagrama de requerimientos - fuente: elaboracion propia}\label{fig:req-reports}
% \end{figure}
% \begin{figure}
%     \includegraphics[width=0.45\textwidth]{req-student-mgmt.png}
%     \caption{diagrama de requerimientos - fuente: elaboracion propia}\label{fig:req-student-mgmt}
% \end{figure}
% \begin{figure}
%     \includegraphics[width=0.45\textwidth]{req-student.png}
%     \caption{diagrama de requerimientos - fuente: elaboracion propia}\label{fig:req-student}
% \end{figure}
% \begin{figure}
%     \includegraphics[width=0.45\textwidth]{req-teaching-staff.png}
%     \caption{diagrama de requerimientos - fuente: elaboracion propia}\label{fig:req-teaching-staff}
% \end{figure}

\subsubsection{Diagrama C4}
The following diagram C4 diagram describes the representation of the system.
\begin{figure}[H]
    \centering
    \includegraphics[width=0.8\textwidth]{c4_lvl-1.png}
    \caption{C4, System context - Fuente: Elaboración propia}
    \label{fig:c4_lvl_1}
\end{figure}
\begin{figure}[H]
    \centering
    \includegraphics[width=0.8\textwidth]{c4_lvl-2.png}
    \caption{C4, Container view - Fuente: Elaboración propia}
    \label{fig:c4_lvl_2}
\end{figure}
\begin{figure}[H]
    \centering
    \includegraphics[width=0.8\textwidth]{c4_lvl-3-timetable_management.png}
    \caption{C4, System context - Fuente: Elaboración propia}
    \label{fig:c4_lvl_3a}
\end{figure}
\begin{figure}[H]
    \centering
    \includegraphics[width=0.8\textwidth]{c4_lvl-3-timetable_st-teaching-mgtm.png}
    \caption{C4, System context - Fuente: Elaboración propia}
    \label{fig:c4_lvl_3b}
\end{figure}

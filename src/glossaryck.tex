\section*{Glosario de términos}

\begin{itemize}
    \item A \\
    API (Interfaz de Programación de Aplicaciones): Un conjunto de reglas, protocolos y herramientas que permite que diferentes componentes o aplicaciones de software se comuniquen e intercambien datos entre sí.

    \item C \\
    Cohorte: Un grupo de estudiantes que inician sus estudios al mismo tiempo y avanzan juntos a través de un programa académico.
    \par
    Modelo C4 (Context, Containers, Components, Code): Enfoque para visualizar la arquitectura de software en diferentes niveles de abstracción (Contexto, Contenedores, Componentes y Código), facilitando la comunicación sobre el diseño del sistema.

    \item D \\
    Diagrama de Gantt: Herramienta de gestión de proyectos que ilustra el cronograma de un proyecto, mostrando las tareas, sus duraciones, fechas de inicio y fin en un formato de barras horizontales.

    \item G \\
    GPA (Grade Point Average): Representación numérica del rendimiento académico de un alumno.

    \item H \\
    Heurística: Regla práctica, atajo o método utilizado para resolver problemas de forma rápida y eficiente, especialmente en situaciones complejas donde no se garantiza una solución óptima pero se busca una suficientemente buena en un tiempo razonable.

    \item I \\
    Integración Continua (CI): Práctica de desarrollo de software donde los miembros del equipo integran su trabajo con frecuencia (generalmente varias veces al día) en un repositorio compartido.
    Cada integración se verifica mediante una construcción automatizada (incluyendo pruebas) para detectar errores de integración lo más rápido posible. (A menudo se complementa con el Despliegue Continuo o CD, que automatiza el lanzamiento de nuevas versiones).
    \par
    Interfaz de Usuario (UI - User Interface): Medio a través del cual un usuario interactúa con una máquina, dispositivo, aplicación de software o sitio web.

    \item L \\
    LMS (Learning Management System - Sistema de Gestión de Aprendizaje): Aplicación de software para la administración, documentación, seguimiento, generación de informes, automatización y entrega de cursos educativos, programas de capacitación o programas de aprendizaje y desarrollo.

    \item M \\
    Machine Learning (Aprendizaje Automático): Rama de la inteligencia artificial que se enfoca en el desarrollo de algoritmos y modelos estadísticos que permiten a los sistemas informáticos aprender de los datos y mejorar su rendimiento en tareas específicas sin ser programados explícitamente para cada caso.
    \par
    Mantenibilidad (del software): Facilidad con la que un producto de software puede ser modificado para corregir defectos, mejorar el rendimiento u otros atributos, o adaptarse a un entorno modificado.
    % Modelo C4 is already under C

    \item P \\
    Product Backlog (Pila del Producto): En metodologías ágiles como Scrum, es una lista priorizada y emergente de funcionalidades, mejoras, correcciones de errores y otros trabajos necesarios para desarrollar o mantener un producto.
    \par
    Prototipo (de interfaz de usuario): Modelo preliminar o maqueta de la interfaz con la que interactuarán los usuarios.
    Se utiliza para visualizar, probar y refinar el diseño antes del desarrollo completo.

    \item S \\
    SAP (Satisfactory Academic Progress): Conjunto de normas que utilizan las facultades y universidades para medir el rendimiento académico de un estudiante y garantizar que progresa hacia la obtención de su título o certificación en el plazo previsto.


\end{itemize}

\subsection{Alcances y limitaciones}

El alcance del proyecto de desarrollo del sistema de planificación académica abarca la creación de un sistema automatizado que permita la gestión eficiente de estudiantes, personal docente, cursos y horarios académicos, con integración al Sistema de Información Estudiantil (SIS). 

El sistema concierne de las siguientes funcionalidades:
\begin{itemize}
    \item \textbf{Gestión de estudiantes:} Registro, actualización y seguimiento de la información académica de los estudiantes, integrando datos del SIS como calificaciones y estado académico.
    \item \textbf{Gestión de personal docente:} Registro y administración de la disponibilidad de los profesores para la asignación de clases, así como la integración de horarios docentes.
    \item \textbf{Gestión de cursos:} Creación, asignación de prerrequisitos y gestión de cursos en función de la oferta académica y las necesidades del semestre.
    \item \textbf{Creación de horarios:} Funcionalidad para crear horarios académicos para el semestre, optimizando la asignación de cursos y la disponibilidad de los docentes.
    \item \textbf{Migración de horarios:} Capacidad para modificar los horarios de los cursos cuando el personal docente solicite cambios, sin crear conflictos con otros cursos.
    \item \textbf{Evaluación de conflictos de horarios:} Análisis automático para detectar posibles conflictos de horarios entre cursos y la disponibilidad de los docentes.
    \item \textbf{Generación de reportes:} Generación de reportes específicos sobre estudiantes, cursos y personal docente, necesarios para la toma de decisiones de la Oficina del Registro.
    \item \textbf{Predicción de aprobación de cursos:} Uso de algoritmos de machine learning para predecir la probabilidad de que un estudiante apruebe un curso basado en su historial académico.
    \item \textbf{Cumplimiento de normativas:} El sistema debe cumplir con regulaciones como FERPA \footnote{Family Educational Rights and Privacy Act: protege la privacidad de los expedientes académicos de los estudiantes}, garantizando la privacidad y seguridad de los datos de los estudiantes.
\end{itemize}

El proyecto no abarcará las siguientes áreas:
\begin{itemize}
    \item \textbf{Modificación de sistemas existentes:} No se realizarán cambios o mejoras en otros sistemas existentes fuera del sistema de planificación académica.
    \item \textbf{Gestión de recursos físicos:} El sistema no gestionará recursos físicos como aulas o materiales educativos, solo se centrará en la planificación académica.
    \item \textbf{Integración con otros sistemas externos:} El sistema se integrará únicamente con el Sistema de Información Estudiantil (SIS). No se contemplan otras integraciones con sistemas externos de la universidad en este proyecto.
    \item \textbf{Capacitación externa:} El proyecto no incluirá programas de capacitación para usuarios fuera de la Oficina del Registro o personal involucrado directamente con la gestión académica.
    \item \textbf{Desarrollo de nuevas funcionalidades fuera del alcance académico:} No se desarrollarán funcionalidades para la gestión de áreas fuera del contexto académico, como gestión de recursos humanos o financieros de la universidad.
\end{itemize}

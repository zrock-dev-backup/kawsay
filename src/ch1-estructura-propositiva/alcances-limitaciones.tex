\subsection{Alcances y limitaciones}
El alcance del proyecto de desarrollo del sistema de planificación académica abarca la creación de un sistema automatizado que permita la gestión del calendario académico y los procesos de inscripción y seguimiento modular de estudiantes.
El sistema concierne de las siguientes funcionalidades:

\begin{itemize}
\item \textbf{Gestión de Estudiantes (enfocada en inscripción y seguimiento modular):}
    \begin{itemize}
	\item Registro y gestión de la estructura académica de los estudiantes (tracks, cohortes, grupos y secciones).
	\item Inscripción de estudiantes en clases, incluyendo la capacidad de realizar inscripciones forzadas con justificación.
	\item Recepción de notificaciones durante el proceso de inscripción (e.g., si el estudiante no cumple prerrequisitos o tiene conflictos de horario).
	\item Visualización de listas de estudiantes agrupados por estado (aprobados/reprobados) al finalizar un módulo.
	\item Evaluación de la elegibilidad del estudiante para un curso, mostrando disponibilidad horaria y cumplimiento de prerrequisitos.
    \end{itemize}
\item \textbf{Gestión de Clases:}
    \begin{itemize}
	\item Creación, actualización y eliminación de clases (magistrales y prácticas).
	\item Selección de cursos (previamente existentes) para ser impartidos en las clases.
	\item Selección de docentes (previamente existentes y con disponibilidad definida) para impartir clases, con sugerencias de docentes disponibles según horarios preferentes.
	\item Definición de periodos (franjas horarias) preferentes para la impartición de clases.
	\item Creación de clases personalizadas para atender a estudiantes que necesiten recursar asignaturas.
    \end{itemize}
\item \textbf{Creación y Gestión de Horarios:}
    \begin{itemize}
	\item Creación de horarios académicos para los módulos.
	\item Funcionalidad para generar automáticamente horarios académicos, optimizando la asignación de cursos a clases y la disponibilidad de los docentes, y considerando preferencias de horario.
	\item Distribución de grupos y secciones en el horario para evitar que docentes y clases tengan lecciones simultáneas.
	\item Visualización de horarios en segmentos semanales, con capacidad para seleccionar rangos de fechas.
    \end{itemize}
\item \textbf{Migración de horarios de Clases:}
    \begin{itemize}
	\item Capacidad para modificar los horarios de las clases individuales cuando el personal docente solicite cambios, sin crear conflictos con otros cursos.
    \end{itemize}
\item \textbf{Evaluación de Conflictos de Horarios para Estudiantes:}
    \begin{itemize}
	\item Análisis automático para detectar posibles conflictos de horarios al intentar inscribir a un estudiante en múltiples clases.
	\item Emisión de alertas y sugerencias de horarios alternativos en caso de detectar un conflicto de inscripción para un estudiante.
    \end{itemize}
\item \textbf{Predicción de aprobación de cursos:}
    \begin{itemize}
	\item Uso de algoritmos de machine learning para predecir la probabilidad de que un estudiante apruebe un curso basado en su historial académico (GPA y SPA).
    \end{itemize}
\item \textbf{Gestión de Módulos Académicos:}
    \begin{itemize}
	\item Vista para la gestión de estudiantes dentro del contexto de un módulo académico (e.g., visualización de aprobados/reprobados).
	\end{itemize}
    \end{itemize}

El proyecto no abarcará las siguientes áreas:
\begin{itemize}
    \item \textbf{Gestión completa de Cursos:} No se incluye la creación, modificación detallada, asignación de prerrequisitos a nivel de la definición del curso, ni la gestión de la oferta académica general o el mapa académico gráfico.
    El sistema consumirá cursos ya definidos.
    \item \textbf{Gestión completa de Personal Docente:} No se incluye el registro, actualización o la gestión de la disponibilidad del personal docente.
    El sistema utilizará la información de disponibilidad existente para la asignación a clases.
    \item \textbf{Gestión detallada de Estudiantes más allá de la inscripción y el módulo:} No se incluyen funcionalidades completas de búsqueda y filtrado avanzado de estudiantes a nivel general del sistema, ni la visualización detallada de perfiles completos de estudiante (e.g., historial académico completo, datos personales extensos) más allá de lo necesario para la inscripción y el seguimiento modular.
    La integración con SIS para calificaciones y estado académico detallado es limitada.
    \item \textbf{Generación de Reportes Avanzados:} La generación de reportes se limitará a listas de estudiantes por módulo (aprobados/reprobados).
    No se generarán reportes específicos detallados sobre la carga docente total, horarios por profesor/curso para la Oficina del Registro, o informes de necesidades de personal docente para próximos periodos.
    \item \textbf{Modificación de sistemas existentes:} No se realizarán cambios o mejoras en otros sistemas existentes fuera del sistema de planificación académica.
    \item \textbf{Gestión de recursos físicos:} El sistema no gestionará recursos físicos como aulas o materiales educativos.
    \item \textbf{Integración extensiva con otros sistemas externos:} El sistema se integrará de forma limitada con el Sistema de Información Estudiantil (SIS) principalmente para verificar la elegibilidad básica y obtener datos para la predicción de aprobación.
    No se contemplan otras integraciones con sistemas externos de la universidad en este proyecto.
    \item \textbf{Capacitación externa:} El proyecto no incluirá programas de capacitación para usuarios fuera de la Oficina del Registro o personal involucrado directamente con la gestión académica cubierta por el sistema.
    \item \textbf{Desarrollo de nuevas funcionalidades fuera del alcance académico:} No se desarrollarán funcionalidades para la gestión de áreas fuera del contexto académico, como gestión de recursos humanos o financieros de la universidad.
    \item \textbf{Cumplimiento de normativas como funcionalidad explícita y auditable:} Si bien el sistema se desarrollará buscando proteger la privacidad de los datos estudiantiles, funcionalidades explícitas y auditables diseñadas específicamente para el cumplimiento de regulaciones como FERPA (más allá de las buenas prácticas generales de seguridad y privacidad) no están dentro del alcance del desarrollo activo.
\end{itemize}

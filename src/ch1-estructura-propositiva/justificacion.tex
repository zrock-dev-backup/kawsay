\subsection{Justificación}

\subsubsection{Justificación Tecnológica}
La generación de horarios universitarios es una tarea vital para el funcionamiento de la universidad.
Existen muchas variables y procesamiento de información que se deben considerar al momento de elaborar los horarios. Dado que actualmente esta tarea se hace de forma manual, se producen errores humanos que podrían haberse identificado fácilmente si hubiera un sistema.

Un algoritmo de aprendizaje, proveniente del campo del aprendizaje automático, puede ser entrenado para predecir si un estudiante aprobará o no el curso, y al mismo tiempo, aprender de sus errores, mejorando progresivamente con el tiempo.
Esta tecnología reduce el tiempo necesario para evaluar cuántos estudiantes podrían inscribirse en una futura clase y aumenta la precisión en cada iteración. Su aplicación en la generación automática de horarios significa una mejora significativa a la hora de planear horarios futuros.

\subsubsection{Justificación Social}
Jala University believes that only through a commited investmen in education they can transform the economies of disadvantaged regions by offering world class educational programs.
Currently they face an important challenge regarding timetable management; by using the system they get to deliver a better education experience because the system provides the following perks:

\begin{itemize}
    \item Automating tasks such as timetable generation, centralization of data reduces the amount of \textit{time} Registrar usually takes to create a working timetable for the module.
    \item Registar can generate different timetables and pick which fits their needs, allowing better \textit{planning}
    \item JalaU is constantly growing and it becomes difficult to manually verify rules compliance, the system automates them; therefore delivering \textit{quality} of service.
\end{itemize}

\subsubsection{Justificación Económica}
Los sistemas de planificación son costosos porque, la mayoría de las veces, representan un subsistema de un software más grande.
Por ejemplo, Teach 'n Go, como LMS, también ofrece funciones de planificación, entre otras. La Universidad Jala ya cuenta con un LMS, pero necesita un sistema de planificación que haga más que simplemente servir como un calendario de actividades.

El cuerpo estudiantil y el personal docente de la Universidad Jala están en constante crecimiento.
Muchas aplicaciones, como Teach 'n Go, tienen un modelo de ingresos basado en suscripción. Por ejemplo, Teach 'n Go cobra 239 USD mensuales y ofrece soporte para solo 1,000 estudiantes. Otras soluciones de software bien establecidas, como Class365, cobran 1,000 USD mensuales, ofreciendo los módulos principales para 751–1,000 estudiantes. A esos precios, un sistema de planificación se vuelve costoso, y aún más si sus características no resuelven las necesidades del negocio.

\subsection{Justificación}

\subsubsection{Justificación Tecnológica}
La generación de horarios universitarios es una tarea vital para el funcionamiento de la universidad.
Existen muchas variables y procesamiento de información que se deben considerar al momento de elaborar los horarios.
Dado que actualmente esta tarea se hace de forma manual, se producen errores humanos que podrían haberse identificado fácilmente si hubiera un sistema.

Usando machine learning se puede predecir si un estudiante aprobará o no el curso, y al mismo tiempo, aprender de sus errores, mejorando progresivamente con el tiempo.
Esta tecnología reduce el tiempo necesario para evaluar cuántos estudiantes podrían inscribirse en una futura clase y aumenta la precisión en cada iteración.
Su aplicación en la generación automática de horarios significa una mejora significativa a la hora de planear horarios futuros.

\subsubsection{Justificación Social}
La Universidad Jala cree que solo a través de una inversión comprometida en educación, pueden transformar las economías de regiones desfavorecidas ofreciendo programas educativos de clase mundial.
Actualmente, enfrentan un desafío importante con respecto a la gestión de horarios.
Al digitalizar este proceso, pueden ofrecer una mejor experiencia educativa, ya que el sistema proporciona los siguientes beneficios:

\begin{itemize}
    \item La automatización de tareas como la generación de horarios y la gestión de conflictos de programación reduce la cantidad de \textit{tiempo} que la oficina de registros usualmente toma para crear un horario funcional para el módulo.
    \item La oficina de registros puede generar diferentes horarios y seleccionar el que mejor se adapte a sus necesidades, permitiendo una mejor \textit{planificación}.
    \item Dado que JalaU está en constante crecimiento, se vuelve difícil verificar manualmente la calidad del proceso; el sistema automatiza estas verificaciones, ofreciendo así una mayor \textit{calidad} de servicio.
\end{itemize}

\subsubsection{Justificación Económica}
Los sistemas de planificación son costosos porque, la mayoría de las veces, representan un subsistema de un software más grande.
Por ejemplo, Teach 'n Go, como LMS, también ofrece funciones de planificación, entre otras.
La Universidad Jala ya cuenta con un LMS, pero necesita un sistema de planificación que haga más que simplemente servir como un calendario de actividades.

El cuerpo estudiantil y el personal docente de la Universidad Jala están en constante crecimiento.
Muchas aplicaciones, como Teach 'n Go, tienen un modelo de ingresos basado en suscripción\footnote{Un \textbf{modelo de ingresos basado en suscripción} es un modelo de negocio en el que un cliente paga una tarifa recurrente (mensual, anual, etc.) para tener acceso a un producto o servicio.}.
Por ejemplo, Teach 'n Go cobra 239 USD mensuales y ofrece soporte para solo 1,000 estudiantes.
Otras soluciones de software bien establecidas, como Class365, cobran 1,000 USD mensuales, ofreciendo los módulos principales para 751–1,000 estudiantes.
A esos precios, un sistema de planificación se vuelve costoso, y aún más si sus características no resuelven las necesidades del negocio.

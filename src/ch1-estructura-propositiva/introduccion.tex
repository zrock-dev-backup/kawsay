\subsection{Introducción}

La elaboración de horarios universitarios es una tarea compleja que implica la organización de cursos, exámenes y actividades mientras se equilibran múltiples restricciones.
Los principales desafíos incluyen evitar conflictos entre cursos y cumplir con las limitaciones de disponibilidad del profesorado, alinearse con los requisitos de los programas, gestionar eficientemente las aulas y recursos especializados, superar las limitaciones tecnológicas y adaptarse a cambios imprevistos.
Además, los horarios mal diseñados pueden afectar negativamente el bienestar de los estudiantes, crear problemas de equidad para aquellos con diferentes circunstancias y presentar complicaciones en entornos de aprendizaje global y en línea.

El aprendizaje automático (machine learning) es una rama de la inteligencia artificial que se centra en el desarrollo de algoritmos y modelos estadísticos que permiten a los ordenadores aprender y mejorar a partir de la experiencia sin ser programados explícitamente para realizar tareas específicas.
Esta tecnología se puede aplicar para la elaboración de horarios académicos, permitiendo planear actividades futuras.

Este proyecto propone automatizar la elaboración de horarios universitarios. 
La implementación de un sistema web, permitiendo administrar el horario académico así como también, utilizando machine learning, predecir futuros cursos.
También, el poder obtener consejo sobre como actualizar el horario en ciertos escenarios.
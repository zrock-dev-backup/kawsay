\subsection{Planteamiento del problema}
\label{sec:problemDefinition}
La planificación académica de la institución se fundamenta en un proceso manual que depende de un sistema complejo de hojas de cálculo descentralizadas.
Si bien esta metodología fue funcional en el pasado, el crecimiento constante en el volumen de estudiantes, cursos y docentes la ha convertido en una fuente crítica de riesgo operativo y de ineficiencia.
La gestión de datos de esta manera no solo es propensa a errores, sino que ha generado una fractura estructural entre las fases clave del ciclo académico.

\subsubsection{Puntos de falla del proceso manual}

\begin{itemize}
    \item \textbf{Desconexión de datos y de intención:} Una vez que el horario maestro es creado, la lógica de negocio y las restricciones que lo formaron (ej. una clase diseñada para una sección específica) no se transfieren automáticamente al proceso de inscripción.
    El personal del Registro debe actuar como un "puente humano", recordando y aplicando manualmente estas reglas, con el riesgo inherente de error u omisión.

    \item \textbf{Desconexión de proceso y de tiempo:} La transición entre la finalización del horario y el inicio de la inscripción es un evento manual y no transaccional.
    Un cambio solicitado por un docente \textit{después} de que el horario ha sido comunicado como "final" crea un estado inconsistente, donde las inscripciones pueden realizarse sobre la base de información obsoleta, requiriendo costosas correcciones manuales.

    \item \textbf{Desconexión de dominios de validación:} El sistema manual obliga a validar el horario a nivel macro (recursos como aulas y docentes) y las inscripciones a nivel micro (restricciones de un estudiante individual) de forma aislada.
    La resolución de un solo conflicto de inscripción de un estudiante puede invalidar el horario para otros veinte, lo que requiere un ciclo de re-validación manual masivo que es cognitivamente inviable de realizar sin errores.
\end{itemize}

Esta dependencia en hojas de cálculo y en la memoria del operador humano para conectar procesos y validar reglas complejas resulta en un ciclo de trabajo tedioso y de alto riesgo, donde la integridad de la planificación y la inscripción no puede ser garantizada sistémicamente.

La dependencia de la Oficina del Registro en hojas de cálculo descentralizadas para la planificación académica no solo introduce un alto riesgo de errores operativos, sino que también crea una \textit{desconexión fundamental entre el proceso de creación de horarios y el subsecuente ciclo de inscripción estudiantil}.

Esta fractura en el flujo de trabajo obliga a realizar reconciliaciones manuales, impide la aplicación consistente de políticas académicas\footnote{Las \textbf{políticas académicas} se refieren al conjunto de reglas que gobiernan tanto la progresión académica del estudiante (prerrequisitos, cursos obligatorios por cohorte) como las restricciones de recursos de la institución (disponibilidad de docentes, capacidad de las aulas y horarios de funcionamiento).} y limita la capacidad de la institución para planificar de forma proactiva.

\subsubsection{Formulación del problema}

¿Cómo la implementación de un motor de planificación 'dirigido por la demanda'\footnote{Un flujo de trabajo que modela la necesidad de un curso como una 'demanda' antes de crear la clase final.} logra unificar el ciclo académico, desde la creación de horarios hasta la inscripción, para garantizar la resiliencia operativa?

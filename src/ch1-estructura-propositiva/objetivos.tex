\subsection{Objetivos}

\subsubsection{Objetivo general}
Development of academic timetable management web system to generate academic timetables and handle student enrollment.

\subsubsection{Objetivos específicos}

\begin{itemize}
    \item Comprender el proceso actual de generacíon de horarios para mejorarlo.
	% Permitiendo a la oficina de registro producir calendarios de calidad y en menor tiempo.
    \item Analizar el impacto de la predicción de la inscripción de estudiantes utilizando machine learning.
    \item Escribir los manuales de usuario y de desarrollador para el sistema, proporcionando orientación para el uso y mantenimiento del sistema.
\end{itemize}

\subsubsection{Objetivos y acciones}
\begin{longtable}{p{3in}|p{3in}}
\caption{Objetivos y acciones} \label{tab:objectivesNactions} \\
\hline
\textbf{Objetivo Específico} & \textbf{Acciones} \\
\hline
\endfirsthead
\hline
\textbf{Objetivo Específico} & \textbf{Acciones} \\
\hline
\endhead
\hline
\endfoot

Comprender el proceso actual de gestión de horarios para mejorarlo. &
\begin{itemize}
    \item Elaboración de un cuestionario para entrevistar a los roles relacionados con la gestión de horarios.
	\item Levantamiento de requerimientos mediante story mapping, marco i* (i-star) y representación de requerimientos como diagrama de requerimientos y backlog del producto.
\end{itemize}
\\\

&
\begin{itemize}
	\item Establecer KPIs para analizar el proceso de gestión de horarios actual y el proceso mejorado posteriormente.
	\item Diseñar prototipos de interfaz de usuario.
    \item Investigar y documentar candidatos para la arquitectura que se ajusten a los requisitos del sistema.
	\item Elaboración de diagramas C4 para documentar la arquitectura y la infraestructura.
    \item Configurar los entornos del sistema (desarrollo, pruebas, staging y producción) automatizados con flujos de integración y despliegue continuo.
	\item Implementar módulos para la gestión de estudiantes, personal docente y cursos.
	\item Implementar un módulo de gestión de horarios que incluya la creación, modificación y evaluación de conflictos de horarios, asegurando la detección de conflictos en tiempo real.
	\item Implementar un módulo para generar informes que incluyan: requisitos de personal docente y horarios de los cursos, incluidos los informes de horarios.
\end{itemize}
\\\hline

Analizar el impacto de la predicción de la inscripción de estudiantes utilizando aprendizaje automático. &
\begin{itemize}
	\item Elaboración de preguntas de investigación para comprender el proceso actual de predicción de estudiantes y medir cualitativa y, si es posible, cuantitativamente la situación actual.
    \item Realizar una investigación sobre modelos y técnicas de aprendizaje automático relevantes para la predicción.
    \item Desde el contexto de machine learning, evaluar los algoritmos más prometedores y sus lenguajes de programación; documentar la investigación.
    \item Documentar y ejecutar una Prueba de Concepto (PoC) de los algoritmos seleccionados.
	\item Desarrollo del módulo para la predicción de elegibilidad de estudiantes utilizando aprendizaje automático.
	\item Repetir las mismas mediciones iniciales y analizar el impacto de la predicción con aprendizaje automático en la conclusión.
\end{itemize}
\\\hline

Escribir y finalizar los manuales de usuario y de desarrollador para el sistema, proporcionando orientación para el uso y mantenimiento del sistema. &
\begin{itemize}
    \item Elaboración del manual de usuario.
    \item Elaboración del manual de desarrollador.
\end{itemize}
\\\hline
\end{longtable}

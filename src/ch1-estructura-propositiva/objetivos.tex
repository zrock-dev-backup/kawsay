\subsection{Objetivos}
\label{sec:objectives}

\subsubsection{Objetivo general}
Desarrollar un sistema web de gestión de horarios académicos para generar horarios académicos y gestionar la inscripción de estudiantes.

\subsubsection{Objetivos específicos}
Se han identificado los siguientes objetivos específicos:
\begin{itemize}
    \item Comprender el proceso actual de generacíon de horarios para mejorarlo.
    Permitiendo a la oficina de registro producir calendarios de calidad y en menor tiempo.
    \item Analizar el impacto de la predicción de la inscripción de estudiantes utilizando machine learning.
    \item Escribir manuales de usuario y de desarrollador para el sistema, proporcionando orientación para el uso y mantenimiento del sistema.
\end{itemize}

Cada objetivo específico tiene sus acciones detalladas en la Tabla~\ref{tab:actionsObjectives}.
{\small
\begin{longtable}{>{\raggedright}p{2in}>{\raggedright\arraybackslash}p{4in}}
\caption{Detalle de acciones}
\label{tab:actionsObjectives} \\
\toprule
\textbf{Objetivo Específico} & \textbf{Acciones} \\
\midrule
\endfirsthead
\textbf{Objetivo Específico} & \textbf{Acciones} \\
\midrule
\endhead
\bottomrule
\endfoot
\hline
\endlastfoot

Comprender el proceso actual de gestión de horarios para mejorarlo. &
\begin{itemize}[nosep,leftmargin=1em,topsep=0pt]
\item Elaboración de un cuestionario para entrevistar a los roles relacionados con la gestión de horarios.
\item Levantamiento de requerimientos mediante story mapping, marco i* (i-star) y representación de requerimientos como diagrama de requerimientos y backlog del producto.
% \item Establecer KPIs para analizar el proceso de gestión de horarios actual y el proceso mejorado posteriormente.
\item Diseñar prototipos de interfaz de usuario.
\item Investigar y documentar candidatos para la arquitectura que se ajusten a los requisitos del sistema.
\item Elaboración de diagramas C4 para documentar la arquitectura y la infraestructura.
\item Configurar los entornos del sistema (desarrollo, pruebas, staging y producción) automatizados con flujos de integración y despliegue continuo.
\item Implementar módulos para la gestión de estudiantes, personal docente y cursos.
\item Implementar un módulo de gestión de horarios que incluya la creación, modificación y evaluación de conflictos de horarios, asegurando la detección de conflictos en tiempo real.
\item Implementar un módulo para generar informes que incluyan: requisitos de personal docente y horarios de los cursos, incluidos los informes de horarios.
\end{itemize} \\

Analizar el impacto de la predicción de la inscripción de estudiantes utilizando aprendizaje automático. &
\begin{itemize}[nosep,leftmargin=1em,topsep=0pt]
\item Elaboración de preguntas de investigación para comprender el proceso actual de predicción de estudiantes y medir cualitativa y, si es posible, cuantitativamente la situación actual.
\item Realizar una investigación sobre modelos y técnicas de aprendizaje automático relevantes para la predicción.
\item Desde el contexto de machine learning, evaluar los algoritmos más prometedores y sus lenguajes de programación; documentar la investigación.
\item Documentar y ejecutar una Prueba de Concepto (PoC) de los algoritmos seleccionados.
\item Desarrollo del módulo para la predicción de elegibilidad de estudiantes utilizando aprendizaje automático.
\item Repetir las mismas mediciones iniciales y analizar el impacto de la predicción con aprendizaje automático en la conclusión.
\end{itemize} \\

Escribir manuales de usuario y de desarrollador para el sistema, proporcionando orientación para el uso y mantenimiento del sistema. &
\begin{itemize}[nosep,leftmargin=1em,topsep=0pt]
\item Elaboración del manual de usuario.
\item Elaboración del manual de desarrollador.
\end{itemize} \\

\end{longtable}
}

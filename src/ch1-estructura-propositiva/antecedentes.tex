\subsection{Antecedentes}
La programación de horarios puede describirse como la asignación de recursos a actividades colocadas en el espacio-tiempo, de tal manera que se minimice el costo total de un conjunto de recursos utilizados.
En términos prácticos, el problema de la elaboración de horarios puede describirse como la programación de una secuencia de clases entre profesores y estudiantes en un período de tiempo, satisfaciendo un conjunto de restricciones variables.

Las actividades de elaboración de horarios en el pasado se realizaban manualmente y un horario típico, una vez construido, permanecía estático, con solo algunos cambios necesarios para ajustarlo cada semestre o año.
Sin embargo, la naturaleza de la educación ha cambiado sustancialmente a lo largo de los años y, por lo tanto, los requisitos de los horarios se han vuelto mucho más complicados de lo que solían ser. 
En consecuencia, la necesidad de generación automatizada de horarios está aumentando y, por ello, el desarrollo de un sistema de generación de horarios que produzca soluciones válidas es esencial.

La Universidad Jala es una institución educativa comprometida con la educación; busca empoderar a los jóvenes talentos de hoy para que puedan enfrentar la altamente competitiva industria tecnológica. 
Su propuesta garantiza un empleo en esta industria, asegurándose de que lleguen a "Conocer".
Tanto el cuerpo estudiantil como docente están en constante crecimiento; elaborar horarios de forma manual se convierte en una tarea que consume mucho tiempo, donde pueden cometerse equivocaciones humanas. Esto abre una oportunidad para innovar en la digitalización de esta actividad.

La Universidad Jala opera en cinco países (Argentina, Brasil, Bolivia, Colombia y México), tiene un programa académico de 8 semestres en el transcurso de 4 años.
Deben manejar grandes cantidades de personas y personal docente; actualmente cuentan con más de 1000 estudiantes y 275 miembros del personal docente. La entidad encargada de gestionar el calendario académico es la oficina del Registro, una entidad compuesta por varios empleados, responsables de construir una planificación de clases y la asignación de estudiantes a estas clases.

Una tarea importante de la oficina del Registro es proporcionar información sobre cuántos miembros del personal docente contratar para el siguiente periodo.
La producción de este informe consiste en evaluar cuántos estudiantes tomarán sus próximos cursos y cuántos volverán a tomar el mismo curso, lo que actualmente es una predicción manual realizada mediante la evaluación de su desempeño académico. Esta situación crea una oportunidad para la innovación, ya que la tarea de predicción puede ser delegada a la máquina. Permitiendo a la oficina del Registro producir informes más precisos, reducir el margen de error y el tiempo necesario.

Así también existen soluciones de software que proponen una solución a la problemática de la generación de horarios.

Teach N' Go es una herramienta de software para administrar una o varias universidades \ref{app:teach-n-go}
Su propósito principal es integrar la funcionalidad de planificación con otras características necesarias para administrar eficientemente las operaciones escolares.
El software resuelve varios problemas clave en la administración educativa al proporcionar organización de clases con códigos de colores, gestión del tamaño de las aulas, capacidades de inscripción masiva de estudiantes, archivado de clases, paneles de estadísticas rápidas e integración con Google Calendar.
Además, Teach N' Go admite la programación de clases recurrentes (semanales o en fechas personalizadas) con campos personalizables para rastrear datos específicos como temas de clase y códigos de curso, y su integración con la API abierta de HostHub permite la comunicación con otros sistemas, permitiéndole funcionar dentro de un ecosistema de software más amplio.
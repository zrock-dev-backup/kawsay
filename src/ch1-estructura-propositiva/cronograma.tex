\subsection{Cronograma}

% \begin{figure}[H]
%     \centering
%     \rotatebox{90}{\includegraphics[width=\textwidth]{"../resources/chronogram/build/chronogram.png"}}
%     \caption{Cronograma de trabajo Gantt - Fuente: Elaboración propia} \label{fig:chronogram}
% \end{figure}

\begin{longtable}{p{3in}|p{3in}}
\caption{Detalle cronograma} \label{tab:chronogram} \\
\hline
\endfirsthead
\hline
\textbf{Sprint} & \textbf{Descripción} \\
\hline
\endhead
\hline
\endfoot

\hline
Definición de proyecto & Perfil de grado \\
\hline

Sprint 0: Análisis, planeación y diseño & 
En este sprint, se realiza un análisis más profundo de los requisitos del sistema.
Se lleva a cabo la planificación detallada del proyecto, con énfasis en la arquitectura del sistema y el diseño de la interfaz de usuario. Este sprint establece las especificaciones técnicas y de diseño necesarias para el desarrollo de las funcionalidades clave del sistema. \\
\hline

Sprint 1: Gestión de estudiantes, integración SIS & 
Este sprint se enfoca en la integración de los datos de estudiantes y calificaciones desde el Sistema de Información Estudiantil (SIS) en el nuevo sistema.
También incluye el desarrollo de funciones de gestión de estudiantes, como la creación y actualización de registros de estudiantes en el sistema. La integración con el SIS es crucial para asegurar que el sistema pueda acceder a la información actualizada de los estudiantes. \\
\hline

Sprint 2: Gestión de estudiantes & 
En este sprint se profundiza en las funciones de gestión de estudiantes, como el registro, actualización y seguimiento del progreso académico de los estudiantes dentro del sistema.
Además, se implementan funcionalidades para mantener la coherencia de los datos entre el sistema de programación y el SIS. \\
\hline

Sprint 3: Gestión de personal docente & 
Este sprint se centra en la gestión de personal docente, incluyendo la creación de registros de profesores y su disponibilidad para enseñar.
Se establece la capacidad para registrar y gestionar la información del personal académico dentro del sistema, asegurando que los horarios de los profesores sean correctamente registrados para la creación del horario académico. \\
\hline

Sprint 4: Gestión de cursos & 
En este sprint, se desarrolla la funcionalidad para crear y gestionar cursos dentro del sistema.
Esto incluye la creación de cursos, la asignación de prerrequisitos, y la administración de la información relacionada con cada curso, así como la integración con los sistemas de gestión de estudiantes y personal docente. \\
\hline

Sprint 5: Creación de horarios & 
Durante este sprint, el sistema comienza a permitir la creación de horarios académicos.
El enfoque principal es permitir que los administradores registren las asignaturas en un horario, considerando la disponibilidad del personal docente y los cursos disponibles. También se implementan funciones para la asignación de estudiantes a los cursos. \\
\hline

Sprint 6: Migración de horarios & 
En este sprint, se implementan las funcionalidades necesarias para la migración de cursos y horarios, permitiendo que los cursos sean reprogramados si es necesario, por ejemplo, debido a solicitudes del personal docente.
Este proceso permite una mayor flexibilidad en la programación del curso, sin comprometer la integridad de los horarios establecidos. \\
\hline

Sprint 7: Evaluación & 
Este sprint se enfoca en la evaluación del sistema, con énfasis en la identificación de conflictos de horarios, análisis de las zonas horarias, y la validación de la creación de horarios sin errores.
También se evalúa la efectividad del sistema para manejar posibles cambios o conflictos, y se realizan ajustes según sea necesario para optimizar el rendimiento del sistema. \\
\hline

Sprint 8: Reportes & 
En este sprint, se trabaja en la generación de reportes que ayuden a los administradores a revisar y analizar el desempeño de la programación académica.
Esto incluye informes sobre la asignación de cursos, la disponibilidad de personal docente, y la carga de trabajo de los estudiantes. Los reportes permiten tomar decisiones informadas sobre la gestión de los horarios y la planificación de futuros términos académicos. \\
\hline

Sprint 9: QA y corrección de errores & 
El último sprint se dedica a la calidad y la corrección de errores.
Se realiza un análisis exhaustivo del sistema para identificar posibles fallos o áreas de mejora. Este sprint incluye pruebas de calidad, pruebas de integración y la corrección de cualquier error detectado en los sprints anteriores. \\
\hline

Versión Alfa & 
Al final del sprint 9, se produce la versión alfa del sistema, que marca la primera entrega funcional del sistema completo.
La versión alfa es un hito importante, ya que representa el sistema listo para las pruebas finales y la validación del cliente antes del lanzamiento. \\
\hline

\end{longtable}

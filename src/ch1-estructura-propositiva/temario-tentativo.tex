\subsection{Temario tentativo}

CAPÍTULO 1. ESTRUCTURA PROPOSITIVA \\
1.1. Introducción \\
1.2. Antecedentes \\
1.3. Planteamiento del problema \\
1.4. Objetivos \\
1.4.1 Objetivo general\\
1.4.2 Objetivos específicos \\
1.4.3 Objetivos y acciones \\
1.5. Justificación \\
1.5.1 Justificación técnica \\
1.5.2 Justificación económica \\
1.5.3 Justificación social \\
1.6. Alcances y limitaciones \\
1.7. Marco Teórico \\
1.8. Temario tentativo \\
1.9. Cronograma \\

CAPÍTULO 2. MARCO TEÓRICO REFERENCIAL \\
2.1 Caso de estudio\\
2.2 Machine Learning\\
2.3 Algoritmos para la generacion de horarios \\
2.4 Metodologías de trabajo\\
2.5 Metodología de investigación\\
2.6 Arquitectura de Backend\
2.7 Tecnologías para Backend\\
2.8 Tecnologías para Frontend\\
2.9 Diagramas\\
2.10 Tecnología de diagramado\\
2.11 Herramientas para gestionar tareas\\
2.12 Herramientas de versionamiento\\
2.13 Herramientas de diseño\\
2.14 Plan de pruebas\\
2.15 I* Framework\\

CAPÍTULO 3. MARCO PRACTICO \\

3.1 Eleccion de metodología agil \\
3.2 Fase de exploracion \\
3.3 Sprint N \\
3.4 Sprint 2 \\
3.5 Sprint 4 \\
3.6 Sprint 6 \\


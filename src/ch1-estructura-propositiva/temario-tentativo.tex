\subsection{Temario tentativo}

CAPÍTULO 1. ESTRUCTURA PROPOSITIVA \\
1.1. INTRODUCCIÓN \\
1.2. ANTECEDENTES \\
1.3. PLANTEAMIENTO DEL PROBLEMA \\
1.3.1 Formulación del problema \\
1.4. OBJETIVOS \\
1.4.1 Objetivo general\\
1.4.2 Objetivos específicos \\
1.4.3 Objetivos y acciones \\
1.5. JUSTIFICACIÓN \\
1.5.1 Justificación técnica \\
1.5.2 Justificación económica \\
1.5.3 Justificación social \\
1.6. ALCANCES Y LIMITACIONES \\
1.7 MARCO TEÓRICO\\
1.7.1 Universidad\\
1.7.1.1 Proceso de elaboración del horario académico\\
1.7.1.2 Manejo de conflictos\\
1.7.1.3 Restricciones en la elaboración\\
1.7.2 Requirements\\
1.7.2.1 Functional\\
1.7.2.2 Non-functional\\
1.7.3 Machine Learning\\
1.7.3.1 Supervised learning\\
1.7.3.2 Unsupervised learning\\
1.7.4 Timetable generation algorithms\\
1.7.4.1 Evolutionary and genetic algorithms\\
1.7.4.2 Constraint-based reasoning\\
1.7.4.3 Linear programming/Integer programming\\
1.7.5 Backend architecture\\
1.7.5.1 Microservices\\
1.7.5.2 Clean Architecture\\
1.7.5.3 Domain Driven Design\\
1.7.6 Frontend architecture\\
1.7.6.1 MVM\\
1.7.7 Design Methodology\\
1.7.7.1 Lean thinking\\
1.7.7.2 Lean sprint design\\
1.7.8 Metodologías de trabajo\\
1.7.8.1 Scrum\\
1.7.8.2 Scrumban\\
1.7.8.3 Conclusion\\
1.7.9 KPI\\
1.7.10 Diagramas\\
1.7.10.1 Diagrama de requerimientos\\
1.7.10.2 Diagrama C4\\
1.7.10.2.1 System context\\
1.7.10.2.2 Containers\\
1.7.10.2.3 Components\\
1.7.10.2.4 Code\\
1.7.11 Metodología de investigación\\
1.7.13 Tecnologías para Backend\\
1.7.13.1 Lenguajes de programación\\
1.7.13.1.1 C\#\\
1.7.13.1.2 Python\\
1.7.13.2 Databases\\
1.7.13.2.1 Relational\\
1.7.13.2.2 Non-Relational\\
1.7.14 Tecnologías para Frontend\\
1.7.14.1 TypeScript - React\\
1.7.15 Herramientas de diseño\\
1.7.15.1 Figma\\
1.7.15.2 Canvas\\
1.7.15.3 Conclusion\\
1.7.16 Organizador de tareas\\
1.7.16.1 Clickup\\
1.7.16.2 Monday\\
1.7.16.3 Taskwarrior\\
1.7.16.4 Taiga\\
1.7.16.4 Conclusion\\
1.7.17 Herramientas de versionamiento\\
1.7.17.1 GitHub\\
1.7.17.2 Sourcehut\\
1.7.17.3 Nomenclatura de ramas, commits y pull requests\\
1.7.18 Tecnología de diagramado\\
1.7.18.1 Structurizr\\
1.7.18.2 PlantUML\\
1.7.19 Plan de pruebas\\
1.7.19.1 ISO 9126\\
1.8 MARCO PRÁCTICO \\
1.8.1 Sprint 0: Análisis, planeación y diseño \\
1.8.2 Sprint 1: Student management, SIS integration \\
1.8.3 Sprint 2: Student management \\
1.8.4 Sprint 3: Teaching staff management \\
1.8.5 Sprint 4: Courses Management \\
1.8.6 Sprint 5: Schedule creation \\
1.8.7 Sprint 6: Schedule migration \\
1.8.8 Sprint 7: Evaluation \\
1.8.9 Sprint 8: Reports \\
1.8.10 Sprint 9: QA and bug fixing \\
1.9 TEMARIO TENTATIVO \\
1.10. BIBLIOGRAFÍA \\
1.11. CRONOGRAMA \\

\subsection{Temario tentativo}

CAPÍTULO 1. ESTRUCTURA PROPOSITIVA \\
1.1. Introducción \\
1.2. Antecedentes \\
1.3. Planteamiento del problema \\
1.3.1 Formulación del problema \\
1.4. Objetivos \\
1.4.1 Objetivo general\\
1.4.2 Objetivos específicos \\
1.4.3 Objetivos y acciones \\
1.5. Justificación \\
1.5.1 Justificación técnica \\
1.5.2 Justificación económica \\
1.5.3 Justificación social \\
1.6. Alcances y limitaciones \\
1.7 Marco Teórico \\
1.8 Temario tentativo \\
1.9. CRONOGRAMA \\

CAPÍTULO 2. MARCO TEÓRICO REFERENCIAL \\
2.1 Caso de estudio\\
2.1 Machine Learning\\
2.1 Timetable generation algorithms\\
2.1 KPI\\
2.1 Metodologías de trabajo\\
2.1 Metodología de investigación\\
2.1 Backend architecture\\
2.1 Frontend architecture\\
2.1 Tecnologías para Backend\\
2.1 Tecnologías para Frontend\\
2.1 Diagramas\\
2.1 Tecnología de diagramado\\
2.1 Herramientas para gestionar tareas\\
2.1 Herramientas de versionamiento\\
2.1 Herramientas de diseño\\
2.1 Plan de pruebas\\

CAPÍTULO 3. MARCO PRACTICO \\

3.1 Eleccion de metodología agil \\
3.2 Fase de exploracion \\
3.3 Sprint N \\
3.4 Sprint 2 \\
3.5 Sprint 4 \\
3.6 Sprint 6 \\


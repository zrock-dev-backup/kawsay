\section{Introduction}

Jala University is an organization commited to education, that seeks to empower today's young talents so that they can face the highly competitive technology industry. Their proposal guarantees a job in this industry making sure they get to "Knowledge".

% The process starts with building a schedule of classes that helps expand the choice of courses available to the student.
Jala University operates in five countries (Argentina, Brazil, Bolivia, Colombia and Mexico), they have an academic program of 8 terms in the span of 4 years. They have to handle big amounts of people and teaching staff, currrently they have more than a 1000 students and 275 teaching staff. The entity in charge of managing the academic calendar is the Registrar office, an entity composed of several employees, responsible of building a schedule of classes and the assigment of students to this classes.

Scheduling, defined as: "A function in many aspects of industry, commerce and computing in which events are timed to take place at the most opportune time" plays an important factor in the success of any bussines. Academic scheduling is all about the creation of courses and the enrollment of students to them. Registrar office has to consider student grading (GPA, SAP), professors and faculty practitioners availability and courses prerequisites, in order to create the schedule for each cohort \textcite{A cohort is a generation of students} every term, this involves a lot of data processing power and since is done manually it creates an space for human error.

% Mention of schedule forecasting definition
One important task of the Registrar's office is to provide information on how many teaching staff to hire for the next term. The production of this report consists of evaluation on how many students will take their next courses and how many will be retaking the same course, currently is a manual prediction done by asseing their academic performance. This situation creates an opportunity for innovation, the task of prediction can be relegated to the machine. Allowing the registrar to produce more acccourate reports, reduce the error marging and the neccesary time.

Machine learning algorithms give a machine the capacity to predict future outcomes based on historical data, this technology brings innovation to the scheduling process allowing the Registrar Office to predict the outcome (i.e. passed or failed) of a student status regarding an specic course and take action in different dimensions, like notifying Student Services so that they can attempt to change this action or produce tentative academic schedules.

% An academic calendar management system can pro
